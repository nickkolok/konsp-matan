
\fXR, $f$ непрерывна на $X$. Из теоремы Ферма вытекает, что точки локального экстремума следует искать среди корней производной и точек, принадлежащих $X$, в которых не существует конечная производная (т. е. производная не определена или бесконечна).

\opred
Корни производной функции называются стационарными точками этой функции.

\opred

Стационарные точки и точки, в которых не существует конечной производной, называются критическими точками первого рода или точками, подозрительными на экстремум.

\subsubsection{Замечание}
Условие $f'(x)=0$, являясь необходимым условием внутреннего локального экстремума дифференцируемой функции, не является достаточным. Классический пример -- функция $f(x)=x^3$ в точке $x=0$ имеет нулевую производную, но не имеет экстремума.

\opred 

Говорят, что при переходе через $x_0$ производная функции $f$ меняет знак с + на -, если
$$
\exists(\delta>0)(\forall(x\in(x_0-\delta;x_0))[f'(x)>0] \cap \forall(x\in(x_0;x_0+\delta))[f'(x)<0])
$$

Определения смены знака производной с - на + и отсутствия смены знака производной аналогичны; сформулировать их оставляем читателю.

\subsubsection{Теорема о смене знака производной}

\fXR, $x_0$ - критическая точка первого рода функции $f$ и функция $f$ дифференцируема в любой внутренней точке $X$, кроме, быть может, точки $x_0$.

Если при переходе через $x_0$ производная меняет знак с + на -, то $x_0$ -- точка локального максимума $f$, если с - на +, то $x_0$ -- точка локального минимума $f$, а если смены знака нет, то в точке $x_0$ нет и экстремума.

\dokvo (Для случая смены знака с + на -; случай смены знака с - на + предоставляем читателю.)
Возьмём $\forall(x \in U_\delta(x_0))$ и рассмотрим отрезок $A$ с концами $x$ и $x_0$. По теореме Лагранжа 
$$
\exists(c\in A)[f(x)-f(x_0)=f'(c)(x-x_0)].
$$

Если $x<x_0$, то $f'(c)>0, x-x_0<0$, откуда $f(x)-f(x_0)<0$.
Если $x>x_0$, то $f'(c)<0, x-x_0>0$, откуда $f(x)-f(x_0)<0$.
Имеем:
$$
\exists(\delta>0)\forall(x\in \mathring{U}_\delta(x_0))[f(x)<f(x_0)].
$$

Это в точности определение локального максимума.

\dokvo (Для случая постоянства знака производной.)
Знак разности $f(x)-f(x_0)$ будет зависеть от знака разности $x-x_0$, т. е. положения точки $x$ слева или справа от точки $x_0$, следовательно, в $x_0$ экстремума нет.

\dokno

\subsubsection{Теорема}
\fXR и функция $f$ имеет в точке $x_0\in X$ производные до n-ого порядка включительно, причём $f'(x_0)=f''(x_0)=...=f^{(n-1)}(x_0)=0$, $f^{(n)}(x_0) \neq 0$.
Тогда:

1) Если $n$ чётно, то в точке $x_0$ функция $f$ имеет экстремум, причём если $f^{(n)}(x_0)<0$, то это максимум, а если $f^{(n)}(x_0)>0$, то минимум.

2) Если $n$ нечётно, то в $x_0$ экстремума функции $f$ нет.

\dokvo (Для случая $f^{(n)}(x_0)>0$; случай $f^{(n)}(x_0)<0$ предоставляем читателю.)

Разложим $f(x)$ по формуле Тейлора в $x_0$ с остаточным членом в форме Пеано:

$$
f(x)=f(x_0)+\frac{f'(x_0)}{1!}(x-x_0)+...+\frac{f^{(n)}(x_0)}{n!}(x-x_0)^n+o(|x-x_0|^n)
$$

Так как $f'(x_0)=f''(x_0)=...=f^{(n-1)}(x_0)$ по условию теоремы, имеем:

$$
f(x)=f(x_0)+\frac{f^{(n)}(x_0)}{n!}(x-x_0)^n+o(|x-x_0|^n)
$$


$$
f(x)-f(x_0)=\frac{f^{(n)}(x_0)}{n!}(x-x_0)^n+o(|x-x_0|^n)
$$

При $x$, достаточно близких к $x_0$,

$$
\sgn(f(x)-f(x_0))=\sgn(f^{(n)}(x_0)(x-x_0)^n).
$$

Так как $f^{(n)}(x_0)>0$, то

$$
\sgn(f(x)-f(x_0))=\sgn((x-x_0)^n).
$$

Если $n$ чётно, то $\sgn((x-x_0)^n)=1$, т. е. $f(x)-f(x_0)>0$, что означает, что $x_0$ - точка минимума.

Если $n$ нечётно, то из последнего равенства имеем

$$
\sgn(f(x)-f(x_0))=\sgn(x-x_0),
$$

т. е. в любой сколь угодно малой окрестности $x_0$ разность $(x)-f(x_0)$ меняет знак, и экстремума функции нет.

\subsubsection{Замечание}

Для того, чтобы найти наибольшее (или наименьшее) значение непрерывной функции $f:[a;b] \to \R$, нужно найти её локальные максимумы (или минимумы) и сравнить значения функции в них со значениями функции на концах отрезка.

Впрочем, иногда просто вычисляют значения функции во всех критических точках.
