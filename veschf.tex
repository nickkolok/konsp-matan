\section{Предел вещественной функции вещественного аргумента}
\subsection{Определение предела функции по Коши, примеры}
\subsection{Определение предела функции по Гейне, примеры, эквивалентность определений}
\subsection{Обобщение понятия предела функции на расширенную числовую ось}

\section{Свойства пределов функции и функций, имеющих предел}
\subsection{Свойства, связанные с неравенствами}
\subsection{Свойства, связанные с арифметическими  операциями}

\section{Односторонние пределы функции}
\subsection{Определение односторонних пределов, связь между существованием предела и односторонних пределов функции}
\subsection{Теорема о существовании односторонних пределов у монотонной функции и ее следств}

\section{Критерий Коши, замечательные пределы, бесконечно малые функции}
\subsection{Критерий Коши существования предела функции}
\subsection{Первый замечательный предел}
\subsection{Второй замечательный предел}
\subsection{Бесконечно малые функции и их классификация}

\section{Непрерывные функции. Общие свойства}
\subsection{Понятие непрерывности функции в точке}
\subsection{Непрерывность функции на множестве}
\subsection{Понятие колебания функции на множестве и в точке. Необходимое и достаточное условие непрерывности функции в точке}
\subsection{Односторонняя непрерывность}
\subsection{Классификация точек разрыва}
\subsection{Локальные свойства непрерывных функций}
\section{Функции, непрерывные на отрезке}

\subsection{Теорема Больцано-Коши и следствия из неё}
\subsection{Первая теорема Вейерштрасса}
\subsection{Вторая теорема Вейерштрасса}
\subsection{Понятие равномерной непрерывности функции. Теорема Кантора, следствия из неё}
\subsection{Свойства монотонных функций. Теорема об обратной функции}
\subsection{Непрерывность элементарных функций}

