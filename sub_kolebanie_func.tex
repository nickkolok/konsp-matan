\opred

\fXR, $E \subset X$, $\alpha_E=\inf_E f(x)$, $\beta_E=\sup_E f(x)$.
Тогда разность $\alpha_E-\beta_E$ называется колебанием функции $f$ на множестве $E$:

$$
\omega(f,E)=\alpha_E-\beta_E=\sup_E f(x) - \inf_E f(x)
$$

Или, что то же самое, 

$$
\omega(f,E)=\sup_{a,b \in E}(f(a)-f(b))
$$

\subsubsection{Примеры.}

$\omega(x^2,[-2;4])=16$

$\omega(\sgn x,[0;4])=1$

$\omega(\sgn x,(0;4])=0$

$\omega(\sgn x,[-1;4])=2$

\opred

\fXRx.
Величина $\lim_{\delta \to 0+}\omega(f,U_{\delta}(x_0))$ называется колебанием функции $f$ в точке $x_0$:

$$
\omega(f,x_0)=\lim_{\delta \to 0+}\omega(f,U_{\delta}(x_0))
$$

\subsubsection{Теорема.}

Пусть $f:X\to\R$.
Функция $f$ непрерывна в точке $x_0\in X$ тогда и только тогда, когда $\omega(f,x_0)=0$.
