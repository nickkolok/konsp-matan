\opred
Дифференциальным биномом (или биномиальным дифференциалом) называется выражение вида
$$x^m(a+bx^n)^p dx$$

Рассмотрим вопрос об интегрировании дифференциального бинома, т.~е. об отыскании интеграла вида
\begin{equation}\label{integral_ot_diff_binoma}
\int x^m(a+bx^n)^p dx
\end{equation}
Сделаем замену $t=x^n$, тогда $x=t^{\frac{1}{n}}$, $dx=\frac{1}{n}t^{\frac{1}{n}-1}$, и 
$$
\int x^m(a+bx^n)^p dx=
\int t^{\frac{m}{n}}(a+bt)^p\cdot \frac{1}{n}\cdot t^{\frac{1}{n}-1}dt=
\frac{1}{n}\int t^{\frac{m+1}{n}-1}(a+bt)^p dt
$$
Положив $q=\frac{m+1}{n}-1$, интеграл (\ref{integral_ot_diff_binoma}) мы представим в виде
$$\varphi(p,q)=\frac{1}{n}\int t^q(a+bt)^p$$

\subsubsection{Теорема.}
Если хотя бы одно из чисел $p$, $q$ или $p+q$ является целым, то интеграл $\varphi(p,q)$ рационализируется.

\dokvo

1. Пусть $p\in\Z$. Тогда $\varphi(p,q)=\int R(t,t^q)dt$. Интегралы такого вида уже были рассмотрены нами ранее.

2. Пусть $q\in\Z$. Тогда $\varphi(p,q)=\int R((a+bt)^p,t)dt$. Интегралы такого вида уже были рассмотрены нами ранее.

3. Пусть, наконец, $p+q\in\Z$. Тогда $\varphi(p,q)=\int R\left(\left(\frac{a+bt}{t}\right)^p,t^{p+q}\right)dt$. И снова получили интеграл уже изученного вида.

\dokno

\subsubsection{Пример.}
$$\int x^2 \sqrt{x}(1-x^2)dx\begin{zamena}m=\frac{5}{2},~n=2,~p=1\in\Z\\x=t^2,~dx=2tdt\end{zamena}$$$$=
\int t^5(1-t^4)2tdt=2\int (t^6-t^{10}) dt=...$$

Завершить вычисление интеграла предоставляем читателю самостоятельно.

\subsubsection{Замечание.}
Великий русский математик Пафнутий Львович Чебышев доказал, что в случае, когда условие  доказанной теоремы не выполнено, интеграл не представим через элементарные функции, т. е. является неберущимся. О неберущихся интегралах читатель узнает буквально на следующей странице.
