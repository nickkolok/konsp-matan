\section{Неопределенный интеграл}
\subsection{Первообразная и неопределенный интеграл}
Основной задачей дифференциального исчисления является нахождение производной функции. Интегральное же исчисление решает обратную задачу -- находит функцию по её производной. Например, если дан пройденный путь в каждый момент времени (зависимость пройденного пути от времени), а нужно найти скорость в каждый момент времени -- это задача дифференциального исчисления; если дана скорость в каждый момент времени, а нужно найти путь -- это задача интегрального.

Заметим, что интегрирование, в отличие от дифференцирования функции, является неоднозначной операцией.

\opred
Функция $F(x)$ называется первообразной функции $f(x)$ на некотором промежутке $X \subset \R$, если $F$ дифференцируема на этом промежутке и 

$$
\forall(x\in X)[F'(x)=f(x)]
$$

\paragraph{Пример.}
Пусть $f(x)=\sin 3x$. Тогда одна из первообразных $F(x)=\frac{-\cos 3x}{3}$.

\subsubsection{Свойство 1.}
Если $F(x)$ -- первообразная функции $f(x)$, то $\forall(C\in\R)[F(x)+C$ -- также первообразная $f(x)]$.

\dokvo
$F'(x)=f(x)$

$(F(x)+C)'=F'(x)+C'=F'(x)+0=F'(x)=f(x)$

\dokno

\subsubsection{Свойство 2.}
Любые две первообразные $F_1(x)$ и $F_2(x)$ функции $f(x)$ отличаются на постоянную.

\dokvo

По определению $F_1'(x)=f(x)$, $F_2'(x)=f(x)$.
Докажем, что $F_1(x)-F_2(x)=const$.
Пусть $\phi(x)=F_1(x)-F_2(x)$.
Тогда $\phi'(x)=f(x)-f(x)=0$.
Значит, $\phi(x)=const$, т. е. $F_1(x)-F_2(x)=const$.

\dokno

Таким образом, по производной можно восстановить функцию с точностью до постоянного слагаемого (его называют произвольной аддитивной постоянной и обозначают $C$).

\opred
Совокупность всех первообразных функции $f$ называется неопределённым интегралом функции $f$ и обозначается $\int f(x) dx$.

$\int$ - знак интеграла. Введён в печать Яковом Бернулли в 1690 году. Значок $\int$ произошёл от латинской буквы $S$ - сокращения ``summa'', а название ``интеграл'' -- от латинского слова ``integro'' -- ``восстанавливать, объединять''

В записи $$\int f(x) dx$$
$x$, стоящая под знаком дифференциала $d$, называется переменной интегрирования;

$f(x)$ называется подынтегральной функцией;

$f(x)dx$ называется подынтегральным выражением.


Если известна одна из первообразных функции $f(x)$, то, поскольку первообразные отличаются на постоянную, известна и вся совокупность первообразных, т. е. неопределённый интеграл.

\subsubsection{Пример.}

$$\int \sin 3x dx=-\frac{1}{3} \cos 3x +C$$


\subsection{Свойства неопределенного интеграла}
\subsubsection{Свойство 1.}
Приозводная непределённого интеграла равна подынтегральной функции:

$$\left(\int f(x) dx \right)'=f(x)$$
$$d\left(\int f(x) dx \right)=f(x)d(x)$$

\subsubsection{Свойство 2.}
Интеграл от производной функции равен этой функции с точностью до постоянной:

$$\int f'(x)dx=f(x)+C$$
$$\int df(x)=f(x)+C$$

Эти два свойства вытектают из определения.

\subsubsection{Свойство 3.}
Если функции $f(x)$ и  $g(x)$ имеют первообразную на $X$, то их линейная комбинация тоже имеет первообразную на $X$ и 

$$\int(\alpha f(x) + \beta g(x))dx=\alpha \int f(x)dx+ \beta \int g(x)dx$$

Доказать это равенство несложно -- достаточно продифференцировать правую и левую часть.
Таким образом, неопределённый интеграл линеен.

\subsubsection{Замечание.}

При последовательных преобразованиях выражения, содержащего неопределённые интегралы, произвольную аддитивную постоянную $C$, возникающую при взятии интеграла, пишут только в тех частях равенства, где нет других интегралов, и опускают в тех частях, где интегралы есть.

\subsubsection{Замечание.}

Знак интеграла $\int$ никогда не используется отдельно от указания переменной интегрирования, наример, $dx$.

Сформулируем также следующую теорему, которая будет доказана позже:

\subsubsection{Теорема}

Если функция непрерывна на промежутке, то она интегрируема на этом промежутке.

\subsection{Таблица интегралов}
Все приведённые равенства устанавливаются дифференцированием правой части и верны на общей области определения правой и левой частей.

Формулы, являющиеся следствием таблицы производных:

\newcounter{N1} % для создания списков, маркированных со стилями, нужен счётчик
\begin{list}{\arabic{N1}.}{\usecounter{N1}}

\item
$$
\int x^\alpha dx= \frac{x^{\alpha+1}}{\alpha+1}+C, \alpha \neq -1
$$

\item
$$
\int \frac{dx}{x}= \ln|x|+C, x\neq 0
$$

\item
$$
\int a^x dx= \frac{a^x}{\ln a}+C
$$

В частности,

$$
\int e^x dx= e^x +C
$$

\item
$$
\int \sin x dx= -\cos x+C
$$

\item
$$
\int \cos x dx= \sin x+C
$$

\item
$$
\int \frac{dx}{\cos^2 x}= \tg x +C
$$

\item
$$
\int \frac{dx}{\sin^2 x}= -\ctg x +C
$$

\item
$$
\int \frac{dx}{\sqrt{1-x^2}}= \arcsin x+C
$$

Обобщение:

$$
\int \frac{dx}{\sqrt{a^2-x^2}}= \arcsin \frac{x}{a}+C
$$

\item
$$
\int \frac{dx}{1+x^2}= \arctg x+C
$$

Обобщение:

$$
\int \frac{dx}{a^2+x^2}= \frac{1}{a} \arctg \frac{x}{a}+C
$$


\item

``Логарифм длинный''
$$
\int \frac{dx}{\sqrt{x^2 \pm 1}}= \ln|x+\sqrt{x^2 \pm 1}|+C
$$

Обобщение:

$$
\int \frac{dx}{\sqrt{x^2 \pm a^2}}= \ln|x+\sqrt{x^2 \pm a^2}|+C
$$

\item

``Логарифм высокий''

$$
\int \frac{dx}{1-x^2}= \frac{1}{2} \ln \left| \frac{1+x}{1-x}\right|+C
$$

Обобщение:
$$
\int \frac{dx}{a^2-x^2}= \frac{1}{2a} \ln \left| \frac{a+x}{a-x}\right|+C
$$

%\end{list}

Напомним теперь читателю определение гиперболических функций.
Вопрос об их интегрировании целесообразно рассмотреть ввиду того, что при интегрировании других функций часто используется т. наз. гиперболическая замена.

\opred

Гиперболический синус $$\sh x=\frac{e^x - e^{-x}}{2}$$

\opred

Гиперболический косинус $$\ch x=\frac{e^x + e^{-x}}{2}$$

\opred

Гиперболический тангенс $$\th x=\frac{\sh x}{\ch x}$$

\opred

Гиперболический котангенс $$\cth x=\frac{\ch x}{\sh x}$$

Продолжим таблицу интегралов:

%\begin{list}{\arabic{N1}.}{}

\item
$$
\int \sh x dx= \ch x+C
$$

\item
$$
\int \ch x dx= \sh x+C
$$

\item
$$
\int \frac{dx}{\ch^2 x}= \th x+C
$$

\item
$$
\int \frac{dx}{\sh^2 x}= -\cth x+C
$$

\end{list}

\subsubsection{Замечание}

При записи результатов интегрирования произвольные аддитивные постоянные объединяют:
$$\int(x^2 + \sin x + 2)dx=\frac{x^3}{3}-\cos x + 2x +C$$


\subsection{Интегрирование по частям}
\subsubsection{Метод.}
Пусть $u(x)$ и $v(x)$ на некотором промежутке $X$ -- диффернецируемые функции. Тогда
$$\int u(x)\cdot v'(x)dx=u(x)\cdot v(x) - \int v(x) \cdot u'(x)dx$$

Т. е., перейдя к дифференциалам функций,
$$\int udv=uv- \int vdu$$

\dokvo
Нам известна формула дифференцирования произведения:
$$(u(x)\cdot v(x))'=u'(x)v(x)+v'(x)u(x)$$
Интегрируем её:
$$u(x)\cdot v(x)'=\int u'(x)v(x) dx +\int v'(x)u(x) dx$$
И переносим один из интегралов в левую часть:
$$u(x)\cdot v(x) - \int v(x) \cdot u'(x)dx=\int u(x)\cdot v'(x)dx$$

\dokno

\subsubsection{Замечание 1.}

При использовании формулы интегрирования по частям подынтегральную функцию нужно представить в виде произведения одной функции на дифференциал другой.
Делают так, чтобы интеграл $\int vdu$ оказался проще, чем интеграл $\int udv$.
Иногда формулу интегрирования по частям приходится применять несколько раз.

\subsubsection{Замечание 2.}

Функция $v$ по $dv$ восстанавливается, вообще говоря, неоднозначно, с точностью до постоянного слагаемого. Его можно считать равным нулю.

\dokvo
Пусть по дифференциалу $dv$ нашлись функции $v_0$ и $v_0+C$. На левую часть, т. е. $\int udv$, $C$ не влияет, т. к. $d(v_0)=d(v_0+C)$. Рассмотрим правую часть:
$$
u\cdot (v_0+C) - \int (v_0+C)du=
uv_0+uC-\int v_0 du - C\int du=$$$$=
uv_0+uC-\int v_0 du - Cu=
uv_0-\int v_0 du
$$
\dokno

\subsubsection{Замечание 3.}
Интегрирование по частям особенно эффективно при интегрировании, если:

а) $u(x)=P_n(x)$, т. е. многочлен от $x$, а $v'(x) \in \{e^x,\sin x,\cos x\}$

б) $u(x) \in \{\ln x, \arctg x \}$, $v'(x)=P_n(x)$

\subsubsection{Пример.}

$$\int x^2 e^x dx = \int \left(\frac{x^3}{3}\right)'e^x dx=$$ $$=
\left<\begin{array}{c|c}
u=x^2 & du=2xdx \\
dv=e^x dx & v=e^x
\end{array}\right>=$$ $$=
x^2 e^x - 2\int e^x \cdot x dx=$$ $$=
\left<\begin{array}{c|c}
u=x & du=dx \\
dv=e^x dx & v=e^x
\end{array}\right>=$$ $$=
x^2 e^x-2\left(e^x \cdot x - \int e^x dx \right)=x^2 e^x - 2 x e^x + 2 e^x +C
$$



\subsection{Замена переменной}
\subsubsection{Теорема.}
Пусть $F$ -- первообразная для $f$ -- непрерывной функции на промежутке $T$, т. е.
$$\int f(t)dt=F(t)+C$$
и на промежутке $X$ задано $\varphi:X\to T$ -- непрерывное дифференцируемое отображение.

Тогда на промежутке $X$
$$\int f(\varphi(x))\cdot \varphi '(x) dx=F(\varphi(x))+C$$
Т. е.
$$\int f(\varphi(x))\cdot d\varphi(x)=F(\varphi(x))+C$$

\dokvo
$$(F(\varphi(x))+C)'=f(\varphi(x))\cdot \varphi'(x)$$
\dokno

\subsubsection{Пример.}

$$\int x e^{x^2} dx = 
\left<\begin{array}{c}
t=x^2 \\
dt=2xdx
\end{array}\right>= %$$ $$=
\frac{1}{2}\int e^t dt = \frac{1}{2} e^t +C = \frac{1}{2}e^{x^2}+C
$$

\subsubsection{Пример.}

$$\int \cos^2 x \sin x dx = 
\left<\begin{array}{c}
t=\sin x \\
dt=-\cos x
\end{array}\right>=$$ $$=
-\int t^2 dt = -\frac{t^3}{3}+C = -\frac{\cos^3 x}{3} +C
$$

\subsubsection{Следствие.}
Если $F'(x)=f(x)$ и $\{a;b\}\in\R$, то
$$\int f(ax+b)dx=\frac{1}{a}F(ax+b)+C$$

\subsubsection{Пример.}

$$\int\cos(7x+3)dx=-\frac{1}{7}\sin(7x+3)+C$$

\subsubsection{Замечание 1.}
Полезно помнить следующие интегралы:

$$\int \frac{g'(x)}{g(x)} dx = 
\left<\begin{array}{c}
t=g(x) \\
dt=g'(x)dx
\end{array}\right>=$$ $$=
\int \frac{dt}{t}=\ln|g(x)|+C
$$

$$\int \frac{g'(x)}{\sqrt{g(x)}} dx = 
\left<\begin{array}{c}
t=g(x) \\
dt=g'(x)dx
\end{array}\right>=$$ $$=
\int \frac{dt}{\sqrt{t}}=2\sqrt{g(x)}+C
$$

\subsubsection{Замечание 2.}
Замену переменной под знаком неопределённого интеграла часто производят иначе: вместо того, чтобы принимать за новую переменную $t$ некоторую функцию $f(x)$, рассматривают $x$ как дифференцируемую функцию от $z$, т. е. $x=\psi(z)$. Тогда
$$\int f(x)dx=\int f(\psi(x))\psi'(z)dz$$
Однако при применении этого метода нужно убедиться, что существует обратная функция $\psi^{-1}(x)=z$, позволяющая вернуться от $z$ к исходной переменной $x$.

\subsubsection{Пример.}
$$\int \sqrt{1-x^2}dx =
\left<\begin{array}{c}
t=\sin z \\
|x|\leq 1; |z|\leq \frac{\pi}{2}
dx=\cos z dz
\end{array}\right>=$$ $$=
\int\sqrt{1-\sin^2 z} \cos z dz=$$$$=
\int\frac{1+\cos 2z}{2}=\frac{1}{2}\int dz +\frac{1}{2}\cdot \frac{1}{2} \sin 2z +C=$$$$=
\frac{\arcsin x}{2}+\frac{\sin(2\arcsin x)}{4}+C=\frac{\arcsin x + x\sqrt{1-x^2}}{2}+C
$$

\subsection{Интегрирование рациональных функций}
\subsection{Интегралы от тригонометрических выражений}
\subsection{Подстановки Эйлера}
\subsection{Интегралы от иррациональных выражений}
\subsection{Интегралы от дифференциальных биномов} 
\subsection{Неберущиеся интегралы}
\opred

Интеграл, не выражающийся через элементарные функции, называется неберущимся.

\subsubsection{Примеры.}
$$\int x^m(a+bx^n)^p dx$$
если $q=\frac{m+1}{n}, p \notin \Z, q \notin\Z, p+q\notin \Z$

$$\int \frac{e^x}{x^n}dx$$

$$\int \frac{\sin x}{x^n}dx$$

$$\int \frac{\cos x}{x^n}dx$$

$$\int \frac{e^{-x^2}}{x^n}dx$$

Часто в приложениях возникает интеграл вида $\int R(x,\sqrt{P_n(x)})dx$. Случаи, когда $n=1$ или $n=2$, исследованы нами ранее. В случае $n \geq 3$, вообще говоря, такой интеграл может быть неберущимся.

С помощью неберущихся интегралов определяются некоторые новые классы трансцендентных функций. Например, эллиптическими интергалами I, II и III рода называются соответственно:
$$\int \frac{dx}{\sqrt{(1-x^2)(1-k^2 x^2)}}$$
$$\int \frac{x^2 dx}{\sqrt{(1-x^2)(1-k^2 x^2)}}$$
$$\int \frac{dx}{(1+hx^2)\sqrt{(1-x^2)(1-k^2 x^2)}}$$

Здесь $0<k<1$.



\section{Определенный интеграл Римана}
\subsection{Задача о вычислении площади криволинейной трапеции}
\subsection{Определение определенного интеграла}
\subsection{Необходимое условие интегрируемости функции}
\subsection{Критерий Коши интегрируемости функции}
\subsection{Необходимое и достаточное условие интегрируемости}
\subsection{Интегралы Дарбу}
\subsection{Признак Дарбу существования интеграла}
\subsection{Свойства интеграла Римана}
\subsection{Первая теорема о среднем}
\subsection{Вторая теорема о среднем} 
\subsection{Формула Ньютона-Лейбница}
\subsection{Формула интегрирования по частям для определенного интеграла}
\subsection{Замена переменной в определенном интеграле}
\subsection{Понятия о приближенных методах вычисления определенных интегралов}
...

\section{Приложения определенного интеграла}
\subsection{Аддитивная функция промежутка}
\subsection{Длина параметризованной кривой} 
\subsection{Площадь поверхности вращения}
\subsection{Площадь фигуры}
\subsection{Объем тела вращения}
\subsection{Понятие о несобственных интегралах}

