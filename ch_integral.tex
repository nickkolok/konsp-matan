\section{Неопределенный интеграл}
\subsection{Первообразная и неопределенный интеграл}
Основной задачей дифференциального исчисления является нахождение производной функции. Интегральное же исчисление решает обратную задачу -- находит функцию по её производной. Например, если дан пройденный путь в каждый момент времени (зависимость пройденного пути от времени), а нужно найти скорость в каждый момент времени -- это задача дифференциального исчисления; если дана скорость в каждый момент времени, а нужно найти путь -- это задача интегрального.

Заметим, что интегрирование, в отличие от дифференцирования функции, является неоднозначной операцией.

\opred
Функция $F(x)$ называется первообразной функции $f(x)$ на некотором промежутке $X \subset \R$, если $F$ дифференцируема на этом промежутке и 

$$
\forall(x\in X)[F'(x)=f(x)]
$$

\paragraph{Пример.}
Пусть $f(x)=\sin 3x$. Тогда одна из первообразных $F(x)=\frac{-\cos 3x}{3}$.

\subsubsection{Свойство 1.}
Если $F(x)$ -- первообразная функции $f(x)$, то $\forall(C\in\R)[F(x)+C$ -- также первообразная $f(x)]$.

\dokvo
$F'(x)=f(x)$

$(F(x)+C)'=F'(x)+C'=F'(x)+0=F'(x)=f(x)$

\dokno

\subsubsection{Свойство 2.}
Любые две первообразные $F_1(x)$ и $F_2(x)$ функции $f(x)$ отличаются на постоянную.

\dokvo

По определению $F_1'(x)=f(x)$, $F_2'(x)=f(x)$.
Докажем, что $F_1(x)-F_2(x)=const$.
Пусть $\phi(x)=F_1(x)-F_2(x)$.
Тогда $\phi'(x)=f(x)-f(x)=0$.
Значит, $\phi(x)=const$, т. е. $F_1(x)-F_2(x)=const$.

\dokno

Таким образом, по производной можно восстановить функцию с точностью до постоянного слагаемого (его называют произвольной аддитивной постоянной и обозначают $C$).

\opred
Совокупность всех первообразных функции $f$ называется неопределённым интегралом функции $f$ и обозначается $\int f(x) dx$.

$\int$ - знак интеграла. Введён в печать Яковом Бернулли в 1690 году. Значок $\int$ произошёл от латинской буквы $S$ - сокращения ``summa'', а название ``интеграл'' -- от латинского слова ``integro'' -- ``восстанавливать, объединять''

В записи $$\int f(x) dx$$
$x$, стоящая под знаком дифференциала $d$, называется переменной интегрирования;

$f(x)$ называется подынтегральной функцией;

$f(x)dx$ называется подынтегральным выражением.


Если известна одна из первообразных функции $f(x)$, то, поскольку первообразные отличаются на постоянную, известна и вся совокупность первообразных, т. е. неопределённый интеграл.

\subsubsection{Пример.}

$$\int \sin 3x dx=-\frac{1}{3} \cos 3x +C$$


\subsection{Свойства неопределенного интеграла}
\subsection{Таблица интегралов}
\subsection{Интегрирование по частям}
\subsection{Замена переменной}
\subsection{Интегрирование рациональных функций}
\subsection{Интегралы от тригонометрических выражений}
\subsection{Подстановки Эйлера}
\subsection{Интегралы от иррациональных выражений}
\subsection{Интегралы от дифференциальных биномов} 
\subsection{Неберущиеся интегралы}
...

\section{Определенный интеграл Римана}
\subsection{Задача о вычислении площади криволинейной трапеции}
\subsection{Определение определенного интеграла}
\subsection{Необходимое условие интегрируемости функции}
\subsection{Критерий Коши интегрируемости функции}
\subsection{Необходимое и достаточное условие интегрируемости}
\subsection{Интегралы Дарбу}
\subsection{Признак Дарбу существования интеграла}
\subsection{Свойства интеграла Римана}
\subsection{Первая теорема о среднем}
\subsection{Вторая теорема о среднем} 
\subsection{Формула Ньютона-Лейбница}
\subsection{Формула интегрирования по частям для определенного интеграла}
\subsection{Замена переменной в определенном интеграле}
\subsection{Понятия о приближенных методах вычисления определенных интегралов}
...

\section{Приложения определенного интеграла}
\subsection{Аддитивная функция промежутка}
\subsection{Длина параметризованной кривой} 
\subsection{Площадь поверхности вращения}
\subsection{Площадь фигуры}
\subsection{Объем тела вращения}
\subsection{Понятие о несобственных интегралах}

