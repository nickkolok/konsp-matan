\section{Неопределённый интеграл}
\subsection{Первообразная и неопределенный интеграл}
Основной задачей дифференциального исчисления является нахождение производной функции. Интегральное же исчисление решает обратную задачу -- находит функцию по её производной. Например, если дан пройденный путь в каждый момент времени (зависимость пройденного пути от времени), а нужно найти скорость в каждый момент времени -- это задача дифференциального исчисления; если дана скорость в каждый момент времени, а нужно найти путь -- это задача интегрального.

Заметим, что интегрирование, в отличие от дифференцирования функции, является неоднозначной операцией.

\opred
Функция $F(x)$ называется первообразной функции $f(x)$ на некотором промежутке $X \subset \R$, если $F$ дифференцируема на этом промежутке и 

$$
\forall(x\in X)[F'(x)=f(x)]
$$

\paragraph{Пример.}
Пусть $f(x)=\sin 3x$. Тогда одна из первообразных $F(x)=\frac{-\cos 3x}{3}$.

\subsubsection{Свойство 1.}
Если $F(x)$ -- первообразная функции $f(x)$, то $\forall(C\in\R)[F(x)+C$ -- также первообразная $f(x)]$.

\dokvo
$F'(x)=f(x)$

$(F(x)+C)'=F'(x)+C'=F'(x)+0=F'(x)=f(x)$

\dokno

\subsubsection{Свойство 2.}
Любые две первообразные $F_1(x)$ и $F_2(x)$ функции $f(x)$ отличаются на постоянную.

\dokvo

По определению $F_1'(x)=f(x)$, $F_2'(x)=f(x)$.
Докажем, что $F_1(x)-F_2(x)=const$.
Пусть $\phi(x)=F_1(x)-F_2(x)$.
Тогда $\phi'(x)=f(x)-f(x)=0$.
Значит, $\phi(x)=const$, т. е. $F_1(x)-F_2(x)=const$.

\dokno

Таким образом, по производной можно восстановить функцию с точностью до постоянного слагаемого (его называют произвольной аддитивной постоянной и обозначают $C$).

\opred
Совокупность всех первообразных функции $f$ называется неопределённым интегралом функции $f$ и обозначается $\int f(x) dx$.

$\int$ - знак интеграла. Введён в печать Яковом Бернулли в 1690 году. Значок $\int$ произошёл от латинской буквы $S$ - сокращения ``summa'', а название ``интеграл'' -- от латинского слова ``integro'' -- ``восстанавливать, объединять''

В записи $$\int f(x) dx$$
$x$, стоящая под знаком дифференциала $d$, называется переменной интегрирования;

$f(x)$ называется подынтегральной функцией;

$f(x)dx$ называется подынтегральным выражением.


Если известна одна из первообразных функции $f(x)$, то, поскольку первообразные отличаются на постоянную, известна и вся совокупность первообразных, т. е. неопределённый интеграл.

\subsubsection{Пример.}

$$\int \sin 3x dx=-\frac{1}{3} \cos 3x +C$$


\subsection{Свойства неопределенного интеграла}
\subsubsection{Свойство 1.}
Приозводная непределённого интеграла равна подынтегральной функции:

$$\left(\int f(x) dx \right)'=f(x)$$
$$d\left(\int f(x) dx \right)=f(x)d(x)$$

\subsubsection{Свойство 2.}
Интеграл от производной функции равен этой функции с точностью до постоянной:

$$\int f'(x)dx=f(x)+C$$
$$\int df(x)=f(x)+C$$

Эти два свойства вытектают из определения.

\subsubsection{Свойство 3.}
Если функции $f(x)$ и  $g(x)$ имеют первообразную на $X$, то их линейная комбинация тоже имеет первообразную на $X$ и 

$$\int(\alpha f(x) + \beta g(x))dx=\alpha \int f(x)dx+ \beta \int g(x)dx$$

Доказать это равенство несложно -- достаточно продифференцировать правую и левую часть.
Таким образом, неопределённый интеграл линеен.

\subsubsection{Замечание.}

При последовательных преобразованиях выражения, содержащего неопределённые интегралы, произвольную аддитивную постоянную $C$, возникающую при взятии интеграла, пишут только в тех частях равенства, где нет других интегралов, и опускают в тех частях, где интегралы есть.

\subsubsection{Замечание.}

Знак интеграла $\int$ никогда не используется отдельно от указания переменной интегрирования, наример, $dx$.

Сформулируем также следующую теорему, которая будет доказана позже:

\subsubsection{Теорема}

Если функция непрерывна на промежутке, то она интегрируема на этом промежутке.

\subsection{Таблица интегралов}
Все приведённые равенства устанавливаются дифференцированием правой части и верны на общей области определения правой и левой частей.

Формулы, являющиеся следствием таблицы производных:

\newcounter{N1} % для создания списков, маркированных со стилями, нужен счётчик
\begin{list}{\arabic{N1}.}{\usecounter{N1}}

\item
$$
\int x^\alpha dx= \frac{x^{\alpha+1}}{\alpha+1}+C, \alpha \neq -1
$$

\item
$$
\int \frac{dx}{x}= \ln|x|+C, x\neq 0
$$

\item
$$
\int a^x dx= \frac{a^x}{\ln a}+C
$$

В частности,

$$
\int e^x dx= e^x +C
$$

\item
$$
\int \sin x dx= -\cos x+C
$$

\item
$$
\int \cos x dx= \sin x+C
$$

\item
$$
\int \frac{dx}{\cos^2 x}= \tg x +C
$$

\item
$$
\int \frac{dx}{\sin^2 x}= -\ctg x +C
$$

\item
$$
\int \frac{dx}{\sqrt{1-x^2}}= \arcsin x+C
$$

Обобщение:

$$
\int \frac{dx}{\sqrt{a^2-x^2}}= \arcsin \frac{x}{a}+C
$$

\item
$$
\int \frac{dx}{1+x^2}= \arctg x+C
$$

Обобщение:

$$
\int \frac{dx}{a^2+x^2}= \frac{1}{a} \arctg \frac{x}{a}+C
$$


\item

``Логарифм длинный''
$$
\int \frac{dx}{\sqrt{x^2 \pm 1}}= \ln|x+\sqrt{x^2 \pm 1}|+C
$$

Обобщение:

$$
\int \frac{dx}{\sqrt{x^2 \pm a^2}}= \ln|x+\sqrt{x^2 \pm a^2}|+C
$$

\item

``Логарифм высокий''

$$
\int \frac{dx}{1-x^2}= \frac{1}{2} \ln \left| \frac{1+x}{1-x}\right|+C
$$

Обобщение:
$$
\int \frac{dx}{a^2-x^2}= \frac{1}{2a} \ln \left| \frac{a+x}{a-x}\right|+C
$$

%\end{list}

Напомним теперь читателю определение гиперболических функций.
Вопрос об их интегрировании целесообразно рассмотреть ввиду того, что при интегрировании других функций часто используется т. наз. гиперболическая замена.

\opred

Гиперболический синус $$\sh x=\frac{e^x - e^{-x}}{2}$$

\opred

Гиперболический косинус $$\ch x=\frac{e^x + e^{-x}}{2}$$

\opred

Гиперболический тангенс $$\th x=\frac{\sh x}{\ch x}$$

\opred

Гиперболический котангенс $$\cth x=\frac{\ch x}{\sh x}$$

Продолжим таблицу интегралов:

%\begin{list}{\arabic{N1}.}{}

\item
$$
\int \sh x dx= \ch x+C
$$

\item
$$
\int \ch x dx= \sh x+C
$$

\item
$$
\int \frac{dx}{\ch^2 x}= \th x+C
$$

\item
$$
\int \frac{dx}{\sh^2 x}= -\cth x+C
$$

\end{list}

\subsubsection{Замечание}

При записи результатов интегрирования произвольные аддитивные постоянные объединяют:
$$\int(x^2 + \sin x + 2)dx=\frac{x^3}{3}-\cos x + 2x +C$$


\subsection{Интегрирование по частям}
\subsubsection{Метод.}
Пусть $u(x)$ и $v(x)$ на некотором промежутке $X$ -- диффернецируемые функции. Тогда
$$\int u(x)\cdot v'(x)dx=u(x)\cdot v(x) - \int v(x) \cdot u'(x)dx$$

Т. е., перейдя к дифференциалам функций,
$$\int udv=uv- \int vdu$$

\dokvo
Нам известна формула дифференцирования произведения:
$$(u(x)\cdot v(x))'=u'(x)v(x)+v'(x)u(x)$$
Интегрируем её:
$$u(x)\cdot v(x)'=\int u'(x)v(x) dx +\int v'(x)u(x) dx$$
И переносим один из интегралов в левую часть:
$$u(x)\cdot v(x) - \int v(x) \cdot u'(x)dx=\int u(x)\cdot v'(x)dx$$

\dokno

\subsubsection{Замечание 1.}

При использовании формулы интегрирования по частям подынтегральную функцию нужно представить в виде произведения одной функции на дифференциал другой.
Делают так, чтобы интеграл $\int vdu$ оказался проще, чем интеграл $\int udv$.
Иногда формулу интегрирования по частям приходится применять несколько раз.

\subsubsection{Замечание 2.}

Функция $v$ по $dv$ восстанавливается, вообще говоря, неоднозначно, с точностью до постоянного слагаемого. Его можно считать равным нулю.

\dokvo
Пусть по дифференциалу $dv$ нашлись функции $v_0$ и $v_0+C$. На левую часть, т. е. $\int udv$, $C$ не влияет, т. к. $d(v_0)=d(v_0+C)$. Рассмотрим правую часть:
$$
u\cdot (v_0+C) - \int (v_0+C)du=
uv_0+uC-\int v_0 du - C\int du=$$$$=
uv_0+uC-\int v_0 du - Cu=
uv_0-\int v_0 du
$$
\dokno

\subsubsection{Замечание 3.}
Интегрирование по частям особенно эффективно при интегрировании, если:

а) $u(x)=P_n(x)$, т. е. многочлен от $x$, а $v'(x) \in \{e^x,\sin x,\cos x\}$

б) $u(x) \in \{\ln x, \arctg x \}$, $v'(x)=P_n(x)$

\subsubsection{Пример.}

$$\int x^2 e^x dx = \int \left(\frac{x^3}{3}\right)'e^x dx=$$ $$=
\left<\begin{array}{c|c}
u=x^2 & du=2xdx \\
dv=e^x dx & v=e^x
\end{array}\right>=$$ $$=
x^2 e^x - 2\int e^x \cdot x dx=$$ $$=
\left<\begin{array}{c|c}
u=x & du=dx \\
dv=e^x dx & v=e^x
\end{array}\right>=$$ $$=
x^2 e^x-2\left(e^x \cdot x - \int e^x dx \right)=x^2 e^x - 2 x e^x + 2 e^x +C
$$



\subsection{Замена переменной}
\subsubsection{Теорема.}
Пусть $F$ -- первообразная для $f$ -- непрерывной функции на промежутке $T$, т. е.
$$\int f(t)dt=F(t)+C$$
и на промежутке $X$ задано $\varphi:X\to T$ -- непрерывное дифференцируемое отображение.

Тогда на промежутке $X$
$$\int f(\varphi(x))\cdot \varphi '(x) dx=F(\varphi(x))+C$$
Т. е.
$$\int f(\varphi(x))\cdot d\varphi(x)=F(\varphi(x))+C$$

\dokvo
$$(F(\varphi(x))+C)'=f(\varphi(x))\cdot \varphi'(x)$$
\dokno

\subsubsection{Пример.}

$$\int x e^{x^2} dx = 
\left<\begin{array}{c}
t=x^2 \\
dt=2xdx
\end{array}\right>= %$$ $$=
\frac{1}{2}\int e^t dt = \frac{1}{2} e^t +C = \frac{1}{2}e^{x^2}+C
$$

\subsubsection{Пример.}

$$\int \cos^2 x \sin x dx = 
\left<\begin{array}{c}
t=\sin x \\
dt=-\cos x
\end{array}\right>=$$ $$=
-\int t^2 dt = -\frac{t^3}{3}+C = -\frac{\cos^3 x}{3} +C
$$

\subsubsection{Следствие.}
Если $F'(x)=f(x)$ и $\{a;b\}\in\R$, то
$$\int f(ax+b)dx=\frac{1}{a}F(ax+b)+C$$

\subsubsection{Пример.}

$$\int\cos(7x+3)dx=-\frac{1}{7}\sin(7x+3)+C$$

\subsubsection{Замечание 1.}
Полезно помнить следующие интегралы:

$$\int \frac{g'(x)}{g(x)} dx = 
\left<\begin{array}{c}
t=g(x) \\
dt=g'(x)dx
\end{array}\right>=$$ $$=
\int \frac{dt}{t}=\ln|g(x)|+C
$$

$$\int \frac{g'(x)}{\sqrt{g(x)}} dx = 
\left<\begin{array}{c}
t=g(x) \\
dt=g'(x)dx
\end{array}\right>=$$ $$=
\int \frac{dt}{\sqrt{t}}=2\sqrt{g(x)}+C
$$

\subsubsection{Замечание 2.}
Замену переменной под знаком неопределённого интеграла часто производят иначе: вместо того, чтобы принимать за новую переменную $t$ некоторую функцию $f(x)$, рассматривают $x$ как дифференцируемую функцию от $z$, т. е. $x=\psi(z)$. Тогда
$$\int f(x)dx=\int f(\psi(x))\psi'(z)dz$$
Однако при применении этого метода нужно убедиться, что существует обратная функция $\psi^{-1}(x)=z$, позволяющая вернуться от $z$ к исходной переменной $x$.

\subsubsection{Пример.}
$$\int \sqrt{1-x^2}dx =
\left<\begin{array}{c}
t=\sin z \\
|x|\leq 1; |z|\leq \frac{\pi}{2}
dx=\cos z dz
\end{array}\right>=$$ $$=
\int\sqrt{1-\sin^2 z} \cos z dz=$$$$=
\int\frac{1+\cos 2z}{2}=\frac{1}{2}\int dz +\frac{1}{2}\cdot \frac{1}{2} \sin 2z +C=$$$$=
\frac{\arcsin x}{2}+\frac{\sin(2\arcsin x)}{4}+C=\frac{\arcsin x + x\sqrt{1-x^2}}{2}+C
$$

\subsection{Интегрирование элементарных дробей}
Рассмотрим вопрос об интегрировании четырёх типов дробей, называемых элементарными.

\opred
Элементарной дробью I типа называется дробь вида
\begin{equation}\label{elem_drob_I}
\frac{a}{x+p}
\end{equation}

Такая дробь интегрируется очевидным образом:
$$
\int\frac{a}{x+p}dx=a\int\frac{d(x+p)}{x+p}=a\ln|x+p|+C
$$

\opred
Элементарной дробью II типа называется дробь вида
\begin{equation}\label{elem_drob_II}
\frac{a}{(x+p)^n},~n\geq 2
\end{equation}

Такая дробь тоже легко интегрируется:
$$
\int\frac{a}{(x+p)^n}dx=a\int\frac{d(x+p)}{(x+p)^n}=\frac{a}{1-n}(x+p)^{-n+1}+C
$$

\opred
Элементарной дробью III типа называется дробь вида
\begin{equation}\label{elem_drob_III}
\frac{ax+b}{x^2+px+q},~ D=p^2-4q<0
\end{equation}

Такая дробь интегрируется с помощью замены
\begin{equation}
t=x+\frac{p}{2},~ dt=dx, ~ \alpha^2=\frac{-D}{4}, ~\beta=b-\frac{ap}{2}
\end{equation}
Имеем:
$$
\int \frac{ax+b}{x^2+px+q}=\int\frac{a\left(x+\frac{p}{2}\right)+b-\frac{ap}{2}}{x^2+2\frac{p}{2}+\frac{p^2}{4}+q-\frac{p^2}{4}}dx=$$$$=
\int\frac{at+\beta}{t^2+\alpha^2}dt=\frac{a}{2}\int\frac{d(t^2)}{t^2+\alpha^2}+\beta\int\frac{dt}{t^2+\alpha^2}=$$$$=\frac{a}{2}\ln|t^2+\alpha^2|+\frac{\beta}{\alpha}\arctg\frac{t}{\alpha}+C
$$
Возвращение к исходным переменной и параметрам предоставляем читателю.
\opred
Элементарной дробью IV типа называется дробь вида
\begin{equation}\label{elem_drob_IV}
\frac{ax+b}{(x^2+px+q)^k},~ D=p^2-4q<0,~k\geq 2
\end{equation}

Такая дробь тоже интегрируется с помощью замены (которая, вообще говоря, часто применяется при интегрировании выражений, содержащих квадратный трёхчлен)
\begin{equation}
t=x+\frac{p}{2},~ dt=dx, ~ \alpha^2=\frac{-D}{4}, ~\beta=b-\frac{ap}{2}
\end{equation}
Имеем:
$$
\int \frac{ax+b}{(x^2+px+q)^k}=
\int
	\frac	{a\left(x+\frac{p}{2}\right)+b-\frac{ap}{2}}
		{\left(x^2+2\frac{p}{2}+\frac{p^2}{4}+q-\frac{p^2}{4}\right)^k}
dx=
$$$$=
\int\frac{at+\beta}{(t^2+\alpha^2)^k}dt=
\frac{a}{2}\int\frac{d(t^2)}{(t^2+\alpha^2)^k}+\beta\int\frac{dt}{(t^2+\alpha^2)^k}=
$$$$=
\frac{a}{2(1-k)}(t^2+\alpha^2)^{1-k}+\beta\int\frac{dt}{(t^2+\alpha^2)^k}
$$

Рассмотрим теперь интеграл
$$J_k=\int\frac{dt}{(t^2+\alpha^2)^k}$$
Преобразуем его:
$$J_k=\frac{1}{\alpha^2}\int\frac{t^2+\alpha^2-t^2}{(t^2+\alpha^2)^k}dt=
$$$$=
\frac{1}{\alpha^2}\int\frac{dt}{(t^2+\alpha^2)^{k-1}}-\frac{1}{\alpha^2}\int\frac{t^2}{(t^2+\alpha^2)^k}dt=
\frac{1}{\alpha^2}J_{k-1}-\frac{1}{\alpha^2}\int\frac{t^2}{(t^2+\alpha^2)^k}dt
$$
Первое слагаемое вычисляется рекуррентно (помним, что $J_1$ -- интеграл от элементарной дроби III типа), займёмся вторым слагаемым:
$$
\int\frac{t^2}{(t^2+\alpha^2)^k}dt=
$$$$
\begin{zamena}[c|c]
u=t & du = dt\\
dv=\frac{t dt}{(t^2+\alpha^2)^k} & v=\int \frac{tdt}{(t^2+\alpha^2)^k}=\frac{1}{2}\int\frac{d(t^2)}{(t^2+\alpha^2)^k}=\frac{1}{2}\cdot\frac{1}{1-k}(t^2+\alpha^2)^{1-k}
\end{zamena}
$$$$
\frac{t}{2}\cdot\frac{1}{1-k}(t^2+\alpha^2)^{1-k}-\int \frac{1}{2}\cdot\frac{1}{1-k}\cdot\frac{tdt}{(t^2+\alpha^2)^{k-1}}
$$
Как вычисляется последний интеграл, мы уже знаем.
Таким образом, интегрирование элементарной дроби IV типа со знаменателем степени $k$ рекуррентно сводится к интегрированию элементарной дроби IV типа со знаменателем степени $k-1$, а, значит, на некотором шаге к интегрированию элементарной дроби III типа.

\subsection{Интегрирование рациональных функций}
Здесь и далее будем обозначать рациональные функции (они же рациональных дроби), т. е. частное двух многочленов, буквой $R$, иногда с некоторыми индексами и диакритиками, а сами многочлены -- буквами $P$, $Q$, $S$, при этом нижний индекс, подобно курсу алгебры, отводится для указания наибольшей возможной степени многочлена. Обратим внимание на то, что некоторые термины и утверждения будут заимствоваться из курса алгебры без отдельного предупреждения.

Итак, рассмотрим вопрос об интегрировании рациональной дроби $R(x)=\frac{P(x)}{Q(x)}$.
Если эта дробь неправильная, то её легко разложить на сумму многочлена и правильной дроби, которые затем интегрировать по отдельности.
Рассмотрим вопрос об интегрировании правильной рациональной дроби ${ R(x)=\frac{P_m(x)}{Q_n(x)} }$, где $m<n$.

Как известно, любой многочлен $Q_n$ представим в виде
\begin{equation}\label{integr_rac_func_mnogoch_razlozh}
Q_n(x)=a_n(x-x_1)^{v_1}+...+(x-x_k)^{v_k}+(x^2+p_{k+1}x+q_{k+1})^{v_{k+1}}+...+(x^2+p_m x+q_m)^{v_m}
,\end{equation}
где $v_1+...+v_k + 2(v_{k+1}+...+v_m)=n$.

Более того, в курсе алгебры доказывается теорема, что для рациональной дроби $R(x)=\frac{P_m(x)}{Q_n(x)}$ со знаменателем, представленным в виде (\ref{integr_rac_func_mnogoch_razlozh}), существует представление
$$
R(x)=S(x)+\sum_{j=1}^k\sum_{l=1}^{v_j}\frac{a_{j,l}}{(x-x_j)^l}+\sum_{j=k+1}^m\sum_{l=1}^{v_j}\frac{b_{j,l}x+c_{j,l}}{(x^2+p_j x + q_j)^l}
$$

При этом $a_{j,l}$, $b_{j,l}$ и $c_{j,l}$ ищутся методом неопределённых коэффициентов: выписывается разложение правильной рациональной дроби на сумму элементарных дробей, элементарные дроби приводятся к общему знаменателю, коэффициенты при одинаковых степенях переменной интегрирования приравниваются. Возникает СЛУ, в которой число уравнений равно числу неизвестных. После её решения и определяются требуемые значения $a_{j,l}$,$b_{j,l}$ и $c_{j,l}$.

Итак, разложение рациональной дроби позволяет нам сформулировать следующую (фактически, уже доказанную) теорему:

\begin{teorema}
Интеграл от любой рациональной функции выражается через рациональную функцию, логарифм и арктангенс.
\end{teorema}

\subsection{Интегралы от тригонометрических выражений}
Рассмотрим интегралы вида 
$$\int R(\sin x,\cos x)dx$$

\subsubsection{Универсальная тригонометрическая подстановка.}
Пусть $t=\tg\frac{x}{2}$, тогда 
$$x=2\arctg t, dx=\frac{2dt}{1+t^2}$$
$$\sin x=\frac{2t}{1+t^2}$$
$$\cos x=\frac{1-t^2}{1+t^2}$$

Таким образом, эта подстановка (известная читателю ещё из курса средней школы, где она применялась для решения тригонометрических уравнений) позволяет гарантированно рационализировать искомый интеграл:

$$\int R(\sin x,\cos x)dx=\int R(\frac{2t}{1+t^2},\frac{1-t^2}{1+t^2})\cdot \frac{2dt}{1+t^2}=\int R_1(t)dt$$

\subsubsection{Пример.}

$$\int \frac{dt}{3+\cos x}=
\left<\begin{array}{c}
t=\tg\frac{x}{2}
\end{array}\right>=
\int\left(\frac{2dt}{1+t^2}\cdot\frac{1}{3+\frac{1-t^2}{1+t^2}}\right)=$$
$$=2\int\frac{dt}{3t^2+3+1-t^2}=\int\frac{dt}{t^2+2}=
\frac{1}{\sqrt{2}}\arctg\frac{t}{\sqrt{2}}+C=$$$$=
\frac{1}{\sqrt{2}}\arctg\left(\frac{1}{\sqrt{2}}\cdot \tg\frac{x}{2}\right)+C$$

Однако неудобство этого метода заключается в том, что степень знаменателя рациональной функции $R_1$ получается сравнительно большой, поэтому применяются и другие, менее универсальные приёмы.

\subsubsection{Приём.}
$$\int R(\sin x)\cdot \cos x dx \begin{zamena}t=\sin x\\dt=\cos x dx\end{zamena}\int R_1(t)dt$$
Для $\int R(\cos x) \cdot \sin x dx$ - аналогично.

\subsubsection{Приём.}
$$\int R(\sin^2 x, \cos^2 x) dx \begin{zamena}t=\tg x\\x=\arctg t, dx=\frac{dt}{1+t^2}\\cos^2 x=\frac{1}{1+t^2}\\cos^2 x=\frac{t^2}{1+t^2}\end{zamena}\int R_1(t^2)dt$$

\subsubsection{Приём.}
$$\int R(\sin^2 x, \cos^2 x) dx =\int R(\frac{1-\cos 2x}{2},\frac{1+\cos 2x}{2})dx=\int R_1(\cos 2x)dx$$

\subsubsection{Замечание.}
Кроме того, при интегрировании произведения тригонометрических функций от линейной функции от $x$ удобно применить представление произведения тригонометрических функций в виде полусуммы.

\subsubsection{Пример.}
$$\int \sin (2x+3) \cdot \cos(3x+2) dx=\frac{1}{2}\int(\sin(2x+3+3x+2) \cdot \sin(2x+3-(3x+2))dx=...$$




\subsection{Интегралы от иррациональных выражений}
Рассмотрим интегралы вида
$$\int R(x,y(x))dx$$
Чтобы свести такой интеграл к интегралу от рациональной функции, нужно найти подстановку $x=x(t)$ такую, чтобы $x(t)$ (а, значит, и $x'(t)$) и $y(x(t))$ были рациональными функциями от $t$:

$$\int R(x,y(x))dx=\int R(x(t),y(x(t)))x'(t)dt=\int R_1(t)dt$$

Рассмотрим сначала случай $y=\sqrt[n]{\frac{\alpha x+ \beta}{\gamma x+ \delta}}$. Пусть
$$t^n=\frac{\alpha x+ \beta}{\gamma x+ \delta}$$
Тогда $$(\gamma x + \delta)t^n=\alpha x + \beta$$
Отсюда $$ (\gamma t^n - \alpha)x = \beta - \delta t^n$$
Т. е. $$x = \frac{\beta - \delta t^n}{\gamma t^n - \alpha}=R_x(t)$$

Интеграл рационализирован.

\subsubsection{Пример.}

$$\int\frac{\sqrt{x}}{1+x}dx\begin{zamena}t=\sqrt{x}\\x=t^2\\dx=2tdt\end{zamena}2\int\frac{t^2 dt}{1+t^2}=$$$$=
2\int\left(1-\frac{1}{1+t^2}\right)dt=2t-2\arctg t+C=2\sqrt{x}-2 \arctg\sqrt{x}+C$$

Обобщим теперь наш опыт на случай интеграла

$$\int R\left( x, \left(\frac{\alpha x + \beta}{\gamma x + \delta}\right)^{r_1},...,\left(\frac{\alpha x + \beta}{\gamma x + \delta}\right)^{r_k}\right)dx,$$

где $r_1,...,r_n \in \Q$. Тогда $r_i = \frac{p_i}{q_i}$. Пусть $m$ - наименьшее общее кратное чисел $q_1,...,q_n$. Введём замену
$$t^m=\frac{\alpha x+ \beta}{\gamma x+ \delta}$$
Легко видеть, что в этом случае интеграл рационализируется.

\subsubsection{Пример.}

$$\int\frac{dx}{\sqrt{x}+\sqrt[3]{x}}=\int\frac{dx}{x^\frac{3}{6}+x^\frac{2}{6}}
\begin{zamena}x=t^6,~~t=x^\frac{1}{6}\\dx=6t^5 dt\end{zamena}6\int\frac{t^5 dt}{t^3+t^2}=$$$$=
6\int\frac{t^3}{t+1}dt=6\left(\int\frac{t^3+1}{t+1}dt-\int\frac{1}{t+1}dt\right)=$$$$=
6\left(\int(t^2-t+1)dt-\ln|t+1|\right)=2t^3-3t^2+6t-\ln|t+1|+C=$$$$=
2\sqrt{x}-3\sqrt[3]{x}+6\sqrt[6]{x}-\ln|\sqrt[6]{x}+1|+C
$$



\subsection{Подстановки Эйлера}
Перейдём теперь к вопросу об интегрировании функции 

\begin{equation}\label{integral_podst_Eilera}
\int R(x,\sqrt{ax^2+bx+c})dx
\end{equation}

Случай, когда $a=0$, фактически рассмотрен нами ранее и потому интереса не представляет.
Введём стандартное обозначение дискриминанта: $D=b^2-4ac$.
Рассмотрим теперь случаи, когда $D=0$.
Если $a<0$, то функция определена лишь в одной точке, и говорить об интеграле нет смысла (т. к. интеграл определяется на промежутке).
Если же $a>0$, то корень извлекается, и задача сводится к взятию интеграла вида $\int R (x,|x-x_0|)dx$, что не представляет особой сложности.

Пусть теперь $a>0$, $D>0$.
Тогда
\begin{equation}\label{vydel_poln_kvadr}
ax^2+bx+c=a\left(x+\frac{b}{2a}\right)^2+\left(c-\frac{b^2}{4a}\right)=a\left(x+\frac{b}{2a}\right)^2-\frac{D}{4a}
\end{equation}

Положим теперь 
\begin{equation}\label{zamena_pered_podst_Eilera}
\tau = \sqrt{a}\left(x+\frac{b}{2a}\right),
\alpha^2=\frac{D}{4a}, ~ 
\text{тогда} ~ 
x=\frac{\tau}{\sqrt{a}}-\frac{b}{2a}, ~ 
dx=\frac{1}{\sqrt{a}}d\tau
\end{equation}
Выражение (\ref{vydel_poln_kvadr}) примет вид $\tau^2-\alpha^2$, а исследуемый интеграл (\ref{integral_podst_Eilera}) преобразуется в:

$$
\int R\left(\frac{\tau}{\sqrt{a}}-\frac{b}{2a},\sqrt{\tau^2-\alpha^2}\right)\cdot\frac{1}{\sqrt{a}}d\tau
$$

Теперь рассмотрим случай, когда $a>0$, $D<0$. Замена будет аналогична замене (\ref{zamena_pered_podst_Eilera}), за исключением того, что $\alpha^2=-\frac{D}{4a}$. Интеграл (\ref{integral_podst_Eilera}) примет вид

$$
\int R\left(\frac{\tau}{\sqrt{a}}-\frac{b}{2a},\sqrt{\tau^2+\alpha^2}\right)\cdot\frac{1}{\sqrt{a}}d\tau
$$

В случае, если $a<0$, $D>0$, замена снова будет аналогична (\ref{zamena_pered_podst_Eilera}), за исключением того, что $\tau = \sqrt{a}\left(x+\frac{b}{2a}\right)$. Интеграл (\ref{integral_podst_Eilera}) примет вид

$$
\int R\left(\frac{\tau}{\sqrt{a}}-\frac{b}{2a},\sqrt{\tau^2-\alpha^2}\right)\cdot\frac{1}{\sqrt{-a}}d\tau
$$

И, наконец, если $D<0$, $a<0$, то подынтегральная функция не имеет смысла.

Таким образом, задача отыскания интеграла (\ref{integral_podst_Eilera}) сведась к отысканию следующих интегралов (здесь $t=\frac{\tau}{\alpha}$, постоянные множители вынесены за знак интеграла):

$$
\int\hat R(t,\sqrt{1-t^2})dt
$$$$
\int\hat R(t,\sqrt{1+t^2})dt
$$$$
\int\hat R(t,\sqrt{t^2-1})dt
$$

Проницательный читатель заметит, что в первых двух случаях можно применить гиперболическую замену, а в третьем - тригонометрическую, но существуют подстановки, позволяющие свести взятие этих интегралов непосредственно к интегрированию рациональной функции. Эти подстановки названы в честь первооткрывателя -- Эйлера.

Для взятия интеграла вида
$$\int\hat R(t,\sqrt{t^2-1})dt$$
применяют замену
$$\sqrt{t^2-1}=u(t\pm 1)$$
или
$$\sqrt{t^2-1}=\pm(t-u)$$

Для взятия интеграла вида
$$\int\hat R(t,\sqrt{t^2+1})dt$$
применяют замену
$$\sqrt{t^2+1}=tu\pm 1$$
или
$$\sqrt{t^2+1}=\pm(t-u)$$

Для взятия интеграла вида
$$\int\hat R(t,\sqrt{1-t^2})dt$$
применяют замену
$$\sqrt{1-t^2}=u(1\pm t)$$
или
$$\sqrt{1-t^2}=tu\pm1$$

Поясним на примере последней, как они работают:
$$\sqrt{1-t^2}=tu-1$$
$$1-t^2=t^2 u^2 -2tu+1$$
$$2tu=(1+u^2)t^2$$
$$2u=(1+u^2)t$$
$$t=\frac{2u}{(1+u^2)}$$
$$\sqrt{1-t^2}=tu-1=\frac{2u^2}{(1+u^2)}$$

Дифференциал $u'(t)du$ также будет рациональной функцией; выписать его предоставляем читателю. Таким образом, интеграл рационализировался.

\subsection{Интегралы от дифференциальных биномов}
\opred
Дифференциальным биномом (или биномиальным дифференциалом) называется выражение вида
$$x^m(a+bx^n)^p dx$$

Рассмотрим вопрос об интегрировании дифференциального бинома, т.~е. об отыскании интеграла вида
\begin{equation}\label{integral_ot_diff_binoma}
\int x^m(a+bx^n)^p dx
\end{equation}
Сделаем замену $t=x^n$, тогда $x=t^{\frac{1}{n}}$, $dx=\frac{1}{n}t^{\frac{1}{n}-1}$, и 
$$
\int x^m(a+bx^n)^p dx=
\int t^{\frac{m}{n}}(a+bt)^p\cdot \frac{1}{n}\cdot t^{\frac{1}{n}-1}dt=
\frac{1}{n}\int t^{\frac{m+1}{n}-1}(a+bt)^p dt
$$
Положив $q=\frac{m+1}{n}-1$, интеграл (\ref{integral_ot_diff_binoma}) мы представим в виде
$$\varphi(p,q)=\frac{1}{n}\int t^q(a+bt)^p$$

\subsubsection{Теорема.}
Если хотя бы одно из чисел $p$, $q$ или $p+q$ является целым, то интеграл $\varphi(p,q)$ рационализируется.

\dokvo

1. Пусть $p\in\Z$. Тогда $\varphi(p,q)=\int R(t,t^q)dt$. Интегралы такого вида уже были рассмотрены нами ранее.

2. Пусть $q\in\Z$. Тогда $\varphi(p,q)=\int R((a+bt)^p,t)dt$. Интегралы такого вида уже были рассмотрены нами ранее.

3. Пусть, наконец, $p+q\in\Z$. Тогда $\varphi(p,q)=\int R\left(\left(\frac{a+bt}{t}\right)^p,t^{p+q}\right)dt$. И снова получили интеграл уже изученного вида.

\dokno

\subsubsection{Пример.}
$$\int x^2 \sqrt{x}(1-x^2)dx\begin{zamena}m=\frac{5}{2},~n=2,~p=1\in\Z\\x=t^2,~dx=2tdt\end{zamena}$$$$=
\int t^5(1-t^4)2tdt=2\int (t^6-t^{10}) dt=...$$

Завершить вычисление интеграла предоставляем читателю самостоятельно.

\subsubsection{Замечание.}
Великий русский математик Пафнутий Львович Чебышев доказал, что в случае, когда условие  доказанной теоремы не выполнено, интеграл не представим через элементарные функции, т. е. является неберущимся. О неберущихся интегралах читатель узнает буквально на следующей странице.
 
\subsection{Неберущиеся интегралы}
\opred

Интеграл, не выражающийся через элементарные функции, называется неберущимся.

\subsubsection{Примеры.}
$$\int x^m(a+bx^n)^p dx$$
если $q=\frac{m+1}{n}, p \notin \Z, q \notin\Z, p+q\notin \Z$

$$\int \frac{e^x}{x^n}dx$$

$$\int \frac{\sin x}{x^n}dx$$

$$\int \frac{\cos x}{x^n}dx$$

$$\int \frac{e^{-x^2}}{x^n}dx$$

Часто в приложениях возникает интеграл вида $\int R(x,\sqrt{P_n(x)})dx$. Случаи, когда $n=1$ или $n=2$, исследованы нами ранее. В случае $n \geq 3$, вообще говоря, такой интеграл может быть неберущимся.

С помощью неберущихся интегралов определяются некоторые новые классы трансцендентных функций. Например, эллиптическими интергалами I, II и III рода называются соответственно:
$$\int \frac{dx}{\sqrt{(1-x^2)(1-k^2 x^2)}}$$
$$\int \frac{x^2 dx}{\sqrt{(1-x^2)(1-k^2 x^2)}}$$
$$\int \frac{dx}{(1+hx^2)\sqrt{(1-x^2)(1-k^2 x^2)}}$$

Здесь $0<k<1$.



\section{Определённый интеграл Римана}
\subsection{Задача о вычислении площади криволинейной трапеции}
К понятию определённого интеграла привела задача о площади криволинейной трапеции.

\opred
Криволинейной трапецией называется фигура на координатной плоскости, ограниченная осью абсцисс, некоторыми прямыми $x=a$ и $x=b$ ($a<b$) и графиком некоторой непрерывной и неотрицательной на $[a;b]$ функции $f$.

\opred 
Разбиением $T$ отрезка $[a;b]$ называется совокупность точек ${x_0,...,x_n}$, таких, что 
$$a=x_0<x_1<x_2<...<x_{n-1}<x_n=b$$

В дальнейшем, говоря о разбиениях, слова ``на отрезке $[a;b]$'' мы будем почти всегда опускать, предполагая, что этот отрезок нам известен.

\opred
Если разбиение $T$ состоит из точек ${x_0,...,x_n}$, то эти точки называются точками деления разбиения $T$.

\opred
Отрезки $[x_{j-1};x_j]$, где $j=1...n$, называются подотрезками разбиения $T$ и обозначаются $\Delta_j$, а их длины обозначаются $\Delta x_j=x_j-x_{j-1}$.

\opred
Наибольшая из длин подотрезков разбиения $T$ называется диаметром разбиения $T$ и обозначается $d(T)=\max\limits_j \Delta_j$

\opred
Если на каждом подотрезке $\Delta_j$ разбиения $T$ выбрать произвольную точку $\xi_j$, то разбиение $T$ называется разбиением с отмеченными точками и обозначается $(T,\xi)$.

Чтобы найти площадь $S_T$ криволинейной трапеции, на отрезке $[a;b]$ строят некоторое разбиение $(T,\xi)$ и затем суммируют площади прямоугольников с шириной $\Delta_j$ и высотой $f(\xi_j)$:
$$S_T \approx \sum_{j=1}^{n}f(\xi_j)\cdot \Delta x_j$$
Здесь $n$ - количество подотрезков разбиения $T$.

Интуитивно ясно, что чем меньше диаметр разбиения, тем лучше приближена площадь трапеции. Строгое математическое доказательство этому будет дано ниже.


\subsection{Определение определённого интеграла}
Введём сначала несколько вспомогательных определений.

\opred
Разбиение $T_2$, получающееся из разбиения $T_1$ путём добавления новых точек деления, называется измельчением разбиения $T_1$.
Пишут $T_2 \supset T_1$.

Часто вместо сквозной нумерации точек измельчения используют двойную, т. е. на отрезке $[x_{j-1};x_j]$ точки нумеруются как $x_{j-1,0}, ..., x_{j-1,m}$.
Заметим, что $x_{j-1,0}=x_{j-1}$, но $x_{j-1,m}<x_j=x_{j,0}$.

\opred
Пусть даны два разбиения $T_1$ и $T_2$.
Их объединением $T=T_1 \cup T_2$ называется разбиение, составленное как из точек $T_1$, так и из точек $T_2$.

Заметим, что в таком случае $T\supset T_1$, $T\supset T_2$.

\opred
Пусть $f:[a;b]\to \R$ и $(T, \xi)$ -- некоторое разбиение отрезка $[a;b]$ на $n$ подотрезков.
Интегральной суммой функции $f$ с разбиением $T$ называется сумма произведений значений функции $f$ в выбранных точках $\xi_j$ на длины соответствующих отрезков разбиения:
$$S(f,(T,\xi))=\sum_{j=1}^n f(\xi_j)\Delta x_j$$

\opred
Функция $f$ называется интегрируемой по Риману на отрезке $[a;b]$, если
\begin{equation}\label{def_opred_integral_1}
\exists(J\in\R)\forall(\varepsilon>0)\exists(\delta>0)\forall((T,\xi):d(T)<\delta)[|S(f,(T,\xi))-J|<\varepsilon]
\end{equation}
Число $J$ в этом случае называют определённым интегралом (или интегралом Римана) функции $f$ на отрезке $[a;b]$ и пишут:
$$
J=\intl_a^b f(x) dx
$$

Здесь:

$f(x)$ -- подынтегральная функция

$f(x)dx$ -- подынтегральное выражение

$[a;b]$ -- промежуток интегрирования

$a$ -- нижний предел интегрирования

$b$ -- верхний предел интегрирования

Иногда определение (\ref{def_opred_integral_1}) пишут так:
$$
\intl_a^b f(x) dx = \lim_{d(T)\to 0} S(f,(T,\xi))
$$

Но следует иметь в виду, что запись предела здесь -- символическая, а не буквальная.
Заметим вскользь, что определение (\ref{def_opred_integral_1}) можно записать в виде, очень похожем на определение предела функции по Коши:
$$\exists(J\in\R)\forall(\varepsilon>0)\exists(\delta>0)\forall((T,\xi))[d(T)<\delta \Rightarrow |S(f,(T,\xi))-J|<\varepsilon]
$$

Тот факт, что функция $f$ является интегрируемой по Риману на отрезке $[a;b]$, сокращённо записывают так:
$$f\in R[a;b]$$


\subsection{Эквивалентное определение определённого интеграла}
И снова начнём со вспомогательного определения:

\opred
Последовательность разбиений $\{T_n\}$ отрезка $[a;b]$ называется неограниченно измельчающейся, если 
$$\lim_{n\to\infty}d(T_n) = 0$$

Проницательный читатель наверняка предположил, что раз существует определение определённого интеграла (\ref{def_opred_integral_1}), аналогичное определению предела функции по Коши, то существует и определение, аналогичное определению предела по Гейне. Сформулируем его:

\begin{opr}\label{eqiv_opr_opr_intl}
Пусть $f:[a;b]\to \R$. Функция $f$ называется интегрируемой по Риману, если
\begin{equation}\label{def_opred_integral_2}
\exists(J\in\R)\forall\left(\left\{\left(T_n,\xi^{(n)}\right)\right\}\right)
\left[\lim_{n\to\infty}d(T_n) = 0 \Rightarrow \lim_{n\to\infty}S\left(f,\left(T_n,\xi^{(n)}\right)\right)=J\right]
\end{equation}
\end{opr}

\subsubsection{Теорема.}
Определения (\ref{def_opred_integral_1}) и (\ref{def_opred_integral_2}) эквивалентны.

Докажем сначала, что (\ref{def_opred_integral_1}) $\Rightarrow$ (\ref{def_opred_integral_2})

\dokvo
Пусть $$J=\intl_a^b f(x) dx $$ в смысле определения (\ref{def_opred_integral_1}).

Зафиксируем любую бесконечно измельчающуюся последовательность разбиений $\left\{\left( T_n, \xi^{(n)}\right)\right\}$. Тогда $d(T_n)\to 0$, и, следовательно,
\begin{equation}\label{dokvo_eqiuv_def_opred_int_1}
\forall(\delta>0)\exists(n_0\in\N)\forall(n\geq n_0)[d(T_n)<\delta]
\end{equation}

С другой стороны, по определению (\ref{def_opred_integral_1}),
$$
\forall(\varepsilon>0)\exists(\delta>0)\forall((T,\xi))[d(T)<\delta \Rightarrow |S(f,(T,\xi))-J|<\varepsilon]
$$

Зафиксировав $\varepsilon$ и найдя из этого условия $\delta$, с учётом (\ref{dokvo_eqiuv_def_opred_int_1}) получим:
$$
\forall(\varepsilon>0)\exists(n_0\in\N)\forall(n\geq n_0)[d(T_n)<\delta]
$$
Следовательно,
$$
\forall(\varepsilon>0)\exists(n_0\in\N)\forall(n\geq n_0)\left[\left|S\left(f,\left(T_n,\xi^{(n)}\right)\right)\right|<\varepsilon\right]
$$
Из этого условия непосредственно следует, что $J=\intl_a^b f(x) dx$ в смысле определения (\ref{def_opred_integral_2})

Докажем теперь, что из выполнения определения (\ref{def_opred_integral_2}) следует выполнение (\ref{def_opred_integral_1})

\dokvo
\pp: пусть определение (\ref{def_opred_integral_2}) выполнено, а определение (\ref{def_opred_integral_1}) - нет, т. е. 
$$
\exists(\varepsilon>0)\forall(\delta>0)\exists((T,\xi))[d(T)<\delta \cap |S(f,(T,\xi))-J|\geq \varepsilon]
$$
Зафиксируем найденное $\varepsilon$ и будем брать $\delta$ из последовательности $\left\{\frac{1}{n}\right\}$. Тогда разбиения $\left(T_n,\xi^{(n)}\right)$ образуют бесконечно измельчающуюся последовательность. Но эта последовательность не сходится к $J$, т. к.
$$\left|S\left(f,\left(T_n,\xi^{(n)}\right)\right)-J\right|\geq \varepsilon$$
Таким образом, определение (\ref{def_opred_integral_2}) не выполнено. Получили противоречие, следовательно, наше допущение о том, что определение (\ref{def_opred_integral_1}) не выполнено -- неверно.
Эквивалентность определений доказана.

\dokno

\subsection{Необходимое условие интегрируемости функции}
\begin{teorema}
Если $f\in R[a;b]$, то $f$ ограничена на $[a;b]$.
\end{teorema}

\dokvo

Идея доказательства заключается в том, что если функция неограничена, то при любом, сколь угодно мелком разбиении найдётся подотрезок, на котором она неограничена, и, двигая по этому отрезку отмеченную точку, можно добиться сколь угодно большой разницы интегральных сумм.

Итак, строгое доказательство.

Так как $f\in R[a;b]$, то $\exists(J\in \R)\left[J=\intl_a^b f(x) dx\right]$.

\pp: $f$ неограничена на $[a;b]$.
Рассмотрим некоторое разбиение $(T,\xi)$.
Тогда $\exists(i)[f \mbox{~неограничена на~}\Delta_i]$, то есть
$$
\exists(i)\forall(M>0)\exists(\xi_M \in \Delta_i)[|f(\xi_M)|>M]
$$

Обозначим
$$
S_i=\sum_{j=1,j\neq i}^n f(\xi_j)\Delta x_j
$$

Тогда
$$
|S(f,(T,\xi))|=|S_i+f(\xi_i)\Delta x_i|\geq
|f(\xi_i)\Delta x_i|-|S_i|
$$

Положим теперь
$$
M=\frac{|J|+1+|S_i|}{\Delta x_i}
$$

Тогда 
$$
|S(f,(T,\xi))|\geq |J|+1+|S_i|-|S_i|=|J|+1
$$

То есть
$$
|S(f,(T,\xi))-J|\geq|S(f,(T,\xi))|-|J|\geq 1
$$

А это означает, что для $\epsilon=1$ определение \ref{opr_opred_integral_1} не выполнено.
Мы пришли к противоречию, следовательно, наше допущение неверно, и функция $f$ ограничена на $[a;b]$.

\dokno

\begin{zamech}
Обратное неверное.
Так, функция Дирихле ограничена на любом отрезке, но не интегрируема на нём.
\end{zamech}



\subsection{Критерий Коши интегрируемости функции}
\begin{teorema}
$f\in R[a;b] \Leftrightarrow $
\begin{multline}\label{krit_Koshi_opr_int}
\forall(\epsilon>0)\exists(\delta>0)\forall((T',\xi'),(T'',\xi''))[d(T')<\delta, d(T'')<\delta \Rightarrow
\\ \Rightarrow
 |S(f,(T',\xi'))-S(T'',\xi'')|<\epsilon]
\end{multline}
\end{teorema}

\dokvo
\neobh
Идея доказательства: просто применить определение интеграла и свойства модуля.

Действительно, по определению определённого интеграла
$$
\exists(J\in \R)\forall(\epsilon>0)\exists(\delta>0)\forall((T',\xi'):d(T')<\delta)[|S(f,(T',\xi'))-J|<\frac\epsilon{2}]
$$
$$
\exists(J\in \R)\forall(\epsilon>0)\exists(\delta>0)\forall((T'',\xi''):d(T'')<\delta)[|S(f,(T'',\xi''))-J|<\frac\epsilon{2}]
$$

Значит, 
\begin{multline*}
|S(f,(T',\xi'))-S(T'',\xi'')|= |S(f,(T',\xi'))-J+J-S(T'',\xi'')|\leq
\\ \leq
|S(f,(T',\xi'))-J|+|S(T'',\xi'')-J|<\frac\epsilon{2}+\frac\epsilon{2}=\epsilon
\end{multline*}
\dokno

\dost

Рассмотрим неограниченно измельчающуюся последовательность разбиений $\left\{\left(T_n,\xi^{(n)}\right)\right\}$.
Так как $d(T)\to 0$, то
$$
\forall(\epsilon>0)\exists(n_0\in\N)\forall(n\geq n_0)[d(T_n)<\delta]
$$

В силу условия (\ref{krit_Koshi_opr_int})
$$
\forall(n,p\geq n_0)\left[\left|S\left(f,\left(T_n,\xi^{(n)}\right)\right)-S\left(f,\left(T_p,\xi^{(p)}\right)\right)\right|<\epsilon\right]
$$

Таким образом, последовательность интегральных сумм $\left\{S\left(f,\left(T_n,\xi^{(n)}\right)\right)\right\}$ -- фундаментальная числовая последовательность и имеет некоторый предел, обозначим его $J$.

Казалось бы, на этом можно остановиться и считать критерий доказанным, но что, если различные последовательности разбиений дадут разные пределы?
Докажем, что такого не произойдёт.

Нужно доказать, что
$$
\forall(\epsilon>0)\exists(\delta>0)\forall((T,\xi))[d(T)<\delta \Rightarrow |S(f,(T,\xi))-J|<\epsilon]
$$

Зафиксируем $\epsilon$.
Найдём по нему $\delta$ из (\ref{krit_Koshi_opr_int}).
Возьмём любое разбиение $T$, не принадлежащее выбранной последовательности, такое, что $d(T)<\delta$.
При достаточно большом $n$ (то есть $n\geq n_0$) $d(T_n)<\delta$.
Применяем условие (\ref{krit_Koshi_opr_int}):
$$
\left|S\left(f,\left(T_n,\xi^{(n)}\right)\right)-S\left(f,\left(T,\xi\right)\right)\right|<\frac\epsilon{2}
$$

При достаточно большом $n$ эти суммы отличаются на сколь угодно малую величину.

\dokno





\subsection{Необходимое и достаточное условие интегрируемости}
\begin{opred}
Пусть f(x) определена на множестве E. Тогда колебанием функции f на множестве E называется
$$
\omega(f,E)=\sup_{x',x''\in E}|f(x')-f(x'')|=\sup_{x\in E}f(x)-\inf_{x\in E}f(x)
$$
если f(x) - ограничена на E, то $\omega(f,E) \neq\infty$
\end{opred}
\\
Рассмотрим промежуток [a;b], где $\vartriangle x_i = x_i - x_{i-1}; \vartriangle_i=[x_{i-1};x_i];$
$$
\vartriangle_{ij}=[x_{i-1;j-1};x_{ij}],
$$
Если $T_1$ - измельчение T.

\begin{teorema}
Для того, чтобы ограниченная функция f(x) была интегрируема по Риману в [a;b] НиД
$$
\forall(\epsilon>0)\exists(\sigma>0)\forall(T)[d(T)<\sigma\Rightarrow\sum_{i=1}^{n}\omega(f,\vartriangle_i)\cdot\vartriangle x_i<\epsilon]
$$
(если $\omega(f,\vartriangle_i)=\omega_i ,$ то $\sum_{i=1}^{n} \omega(f,\vartriangle_i)\cdot x_i= \sum_{i=1}^{n}\omega_i \vartriangle x_i$)
\end{teorema}

\dokvo
\subsubsection{Необходимость}
Дано: $f\in R[a;b]$
\\
Возьмем $\forall(\epsilon>0).$ Т.к. $f\in R[a;b],$ то выполняется критерий Коши:

$$
\forall(\epsilon>0)\exists(\sigma>0)\forall((T',\varphi'),(T'',\varphi''))[d(T')<\sigma_1 d(T')<\sigma\Rightarrow|S(f,(T',\varphi'))-S(f,(T'',\varphi''))|<\frac{\epsilon}{3}]
$$
\\
Пусть $T'=T''=T$, а $\varphi'' \neq \varphi'$.
\\
$\varphi'=(\varphi_1',...,\varphi_n')$ выберем так, чтобы:
$$
f(\varphi_i')>\sup_{x\in\vartriangle_i}f(x)-\frac{\epsilon}{3(b-a)}=M_i-\frac{\epsilon}{3(b-a)}
$$
где $M_i=\sup_{x\in\vartriangle_i}f(x)$
\\
$\varphi''=(\varphi_1'',...,\varphi_n'')$ выберем так чтобы:
$$
f(\varphi_i')>\inf_{x\in\vartriangle_i}f(x)-\frac{\epsilon}{3(b-a)}=m_i-\frac{\epsilon}{3(b-a)}
$$
Преобразуем условие:
$$
\sum_{i=1}^{n}\omega_i\vartriangle x_i=\sum_{i=1}^{n}(M_i-m_i)\vartriangle x_i=
$$

$$
=\sum_{i=1}^{n}(M_i-f(\varphi_i')+f(\varphi_i')f(\varphi_i'')+f(\varphi_i'')-m_i)\cdot\vartriangle x_i=
$$

$$
=\sum_{i=1}^{n}(M_i-f(\varphi_i'))\vartriangle x_i + \sum_{i=1}^{n}(f(\varphi_i')-f(\varphi_i''))\vartriangle x_i + \sum_{i=1}^{n}(f(\varphi_i'')-m_i)\vartriangle x_i<
$$

$$
\sum_{i=1}^{n}(\frac{\epsilon}{3(b-a)}\cdot\vartriangle x_i) + \sum_{i=1}^{n}f(\varphi_i')\vartriangle x_i + \sum_{i=1}^{n}f(\varphi_i'')\vartriangle x_i + \sum_{i=1}^{n}\frac{\epsilon}{3(b-a)}\cdot \vartriangle x_i =
$$

$$
=\frac{2\epsilon}{3(b-a)}\cdot\sum_{i=1}^{n}\vartriangle x_i + S(f,(T,\varphi')) - S(f,(T,\varphi''))<
$$

$$
<\frac{2\epsilon}{3(b-a)}\cdot (b-a) + \frac{\epsilon}{3}=\epsilon
$$
\dokno

\begin{teorema}
$f\in R[a;b] \Leftrightarrow$
\begin{equation}\label{nid_usl_int_koleb}
	\Leftrightarrow \forall(\epsilon>0)\exists(\delta>0)\forall(T:d(T)<\delta)
	\left[\sum_{i=1}^m\omega(f,\Delta_i) \Delta x_i <\epsilon\right]
\end{equation}
\end{teorema}

\dokvo

\subsubsection{Достаточность}
Дано: усл. Доказать: $f\in R[a;b]$
\\
В силу критерия Коши надо доказать что выполняется условие (*) (см.2.2).
\\
Возьмем два разбиения: ($T,\varphi$) и ($\tilde{T},\tilde{\varphi}$):$T \subset \tilde{T}$, т.е. $\tilde{T}$ - измельчение T. Пусть T имеет n  точек разбиения, и в каждом i-том подотрезке будет $n_i$ разбиений: значит, $\sum_{j=1}^{n}\vartriangle x_{ij}=\vartriangle x_i.$ Тогда
$$
|S(f,(T,\varphi))-S(f(\tilde{T},\tilde{\varphi}))|=|\sum_{i=1}^{n}f(\varphi_i)\vartriangle x_i - \sum_{i=1}^{n}\sum_{j=1}^{n}f(\varphi_{ij})\vartriangle x_{ij}|
$$

$$
|\sum_{i=1}^{n}\sum_{j=1}^{n}(f(\varphi_i)-f(\varphi_{i,j}))\vartriangle x_{ij}|\leq
$$

\subsubsection{Пояснение:}
почему $\sum_{i=1}^{n}f(\varphi_i)\vartriangle x_i$ можно заменить на $\sum_{i=1}^{n}\sum_{j=1}^{n}f(\varphi_{i})\vartriangle x_{ij}$?
\\
Вспомним, что такое $\sum_{i=1}^{n}f(\varphi_i)\vartriangle x_i$;
\\
$\sum_{i=1}^{n}\sum_{j=1}^{n}f(\varphi_{i})\vartriangle x_{ij}$
(та же самая площадь, только разбитая на более мелкие кусочки)
\\
Или алгебраически:
$\sum_{i=1}^{n}\sum_{j=1}^{n}f(\varphi_{i})\vartriangle x_{ij} = \sum_{i=1}^{n}f(\varphi_i)\sum_{j=1}^{n_i}\vartriangle x_{ij}=\sum_{i=1}^{n}f(\varphi_i)\vartriangle x_i$

$$
\leq |f(\varphi_i)-f(\varphi_{ij})\leq\omega(f,\vartriangle_i)=\omega_i|\leq
$$

$$
\leq\sum_{i=1}^{n}\sum_{j=1}^{n}\omega_i\vartriangle x_{ij} = \sum_{i=1}^{n}\omega_i\sum_{j=1}^{n}\vartriangle x_{ij}=
$$

$$
=\sum_{i=1}^{n}\omega_i \vartriangle x_i < \frac{\epsilon}{2}
$$
\\
Возьмём $\forall(\epsilon>0)$. Тогда $\exists(\sigma>0)[d(T)<\sigma\Rightarrow\sum_{i=1}^{n}\omega_i \vartriangle x_i < \frac{\epsilon}{2}]$
\\
Но!Возьмём 2 разбиения: $\forall$($(T',\varphi')$ и $(T'',\varphi'')$).
\\
Пусть $T=T'\cup T'' \Rightarrow T-$ измельчение T' и T'' $\Rightarrow d(T)<\sigma\Rightarrow$

$$
\Rightarrow|S(f,(T',\varphi'))-S(f,(T'',\varphi''))|=\leq
$$

$$
\leq|S(f,(T,\varphi))+S(f,(T',\varphi'))+S(f,(T,\varphi))-S(f,(T'',\varphi''))|\leq
$$

$$
\leq ||S(f,(T,\varphi))-S(f,(T',\varphi'))| + |S(f,(T,\varphi))-S(f,(T'',\varphi''))|<\frac{\epsilon}{2} + \frac{\epsilon}{2} = \epsilon
$$
Значит выполняется (*) из критерия Коши следует что $f\in R[a;b]$
\dokno

\subsection{Интегралы Дарбу}
\begin{opred}
Пусть есть $f:[a;b] \to R$, f(x) - ограничена.
Есть $\forall(T:T=\{x_0,...,x_n\}); M_i=\sup_{x\in\vartriangle_i}f(x); m_i=\inf_{x\in\vartriangle_i}f(x).$
\\
Тогда верхняя ($\overline{S}(f,T)$)/нижняя ($\underline{S}(f,T)$) сумма Дарбу функции f(x) по отрезку [a;b] и разбиению T - это:
$$
\overline{S}(f,T)=\sum_{i=1}^{n}M_i \vartriangle x_i
$$
$$
\underline{S}(f,T)=\sum_{i=1}^{n}m_i \vartriangle x_i
$$
где $\vartriangle x_i = x_i - x_{i-1}; \vartriangle_i=[x_{i-1};x_i]$
\end{opred}

\subsubsection{Свойства сумм Дарбу}
\subsubsection{(1)}
$\forall(T)$ справедливо неравенство:
$$
m(b-a) \leq \underline{S}(f,T)\leq S(f,T)\leq \overline{S}(f,T) \leq M(b-a)
$$
$$ где m=\inf_{x \in [a;b]}f(x), M=\sup_{x\in[a;b]}f(x)$$
$$S(f,(T,\varphi))=\sum_{i=1}^{n}f(\varphi_i)\vartriangle x_i  (\varphi_i \in \vartriangle_i)$$
$$m\leq m_i\leq f(\varphi_i)\leq M_i\leq M$$
\subsubsection{(2)}
$\forall(T)[\sup S(f,(T,\varphi))=\overline{S}(f,T);\inf S(f,(T,\varphi))=\underline{S}(f,T)]$
\\
\dokvo
Пусть $\varphi=(\varphi_1,...,\varphi_n)$
\\
$\sup S(f,(T,\varphi))=\sup \sum_{i=1}^{n}f(\varphi_i)\vartriangle x_i=\sup \sum_{i=1}^{n}\sup f(\varphi_i)\vartriangle x_i=$
\\
$=\sum_{i=1}^{n}M_i\vartriangle x_i = \overline{S}(f,T)$
\\
$\inf S(f,(T,\varphi))=\inf\sum_{i=1}^{n}(f(\varphi_i)\vartriangle x_i)=\inf\sum_{i=1}^{n}\inf f(\varphi_i)\vartriangle x_i = \sum_{i=1}^{n}m_i\vartriangle x_i = \underline{S}(f,T)$
\subsubsection{(3)}
$\forall(T,\tilde{T}:T\subset\tilde{T})[\overline{S}(f,\tilde{T})\leq \overline{S}(f,T); \underline{S}(f,\tilde{T})\geq \underline{S}(f,T)]$
\dokvo
$$\overline{S}(f,\tilde{T})=\sum_{i=1}^{n}\sum_{j=1}^{n_i}M_{ij}\vartriangle x_{ij} \leq \sum_{i=1}^{n}M_i\sum_{j=1}^{n_i}\vartriangle x_{ij}= \sum_{i=1}^{n}M_i\vartriangle x_i = \overline{S}(f,T)
$$
$$
M_{ij}=\sup_{x\in \vartriangle_{ij}}
$$

$$\underline{S}(f,\tilde{T})=\sum_{i=1}^{n}\sum_{j=1}^{n_i}m_{ij}\vartriangle x_{ij} \geq \sum_{i=1}^{n}m_i\sum_{j=1}^{n_i}\vartriangle x_{ij}= \sum_{i=1}^{n}m_i\vartriangle x_i = \underline{S}(f,T)
$$
$$
M_{ij}=\inf_{x\in \vartriangle_{ij}}
$$
\dokno

\subsubsection{(4)}
Пусть $\forall(T_1,T_2)\exists(T)[T=T_1 \cup T_2].$ Тогда $T_1 \subset T, T_2 \subset T.$
$$\underline{S}(f,T_1) \leq \underline{S}(f,T) \leq \overline{S}(f,T) \leq \overline{S}(f,T_2)$$
\dokno

\subsubsection{Верхние и нижние интегралы Дарбу}
Из свойства и сумм Дарбу следует, что ${\overline{S}(f,T)}$ ограничена снизу, т.е. $\exists(\inf_{T}\overline{S}(f,T))$ и $\underline{S}(f,T)$ ограничена сверху, т.е. $\exists(\sup_{T}(f,T))$
\\
\begin{opred}
Эти числа называются соответственно верхний($\overline{\overline{s}}$) и нижний ($\underline{\overline{s}}$) интегралы Дарбу.
$$
\overline{\overline{s}}=\inf_{T}\overline{S}(f,T);\underline{\overline{s}}=\inf_{T}\underline{S}(f,T)
$$
\end{opred}

\begin{teorema}
$$
\lim_{d(T)\to 0}\overline{S}(f,T)=\overline{\overline{s}};
\lim_{d(T)\to 0}\underline{S}(f,T)=\underline{\overline{s}}
$$
\end{teorema}
\dokvo
Надо доказать что:
$$
\forall(\epsilon>0)\exists(\sigma>0)\forall(T:d(T)<\sigma)[|\overline{S}(f,T)-\overline{s}|<\epsilon].
$$
$$
\overline{\overline{s}}=\inf\overline{S}(f,T)\Rightarrow\overline{S}(f,T)-\overline{\overline{s}}\geq 0\Rightarrow|\overline{S}(f,T)-\overline{s}|=\overline{S}(f,T)-
\overline{s}
$$
\\
Значит теперь надо доказать:
$$
\forall(\epsilon>0)\exists(\sigma>0)\forall(T:d(T)<\sigma)[\overline{S}(f,T)<\overline{s}+\epsilon]
$$
Возьмем $\forall(\epsilon>0)$ т.к. $\overline{s}=\inf \overline{S}(f,T)$, то по определению $\inf:$
$$\exists(T_1 отрезка [a;b]=\{x_0,...,x_n\})[\overline{S}(f,T)<\overline{\overline{s}}+\frac{\epsilon}{2}]$$
Положим $\exists(\sigma=\frac{\epsilon}{2\Omega p})$, где $\Omega=\omega(f,[a;b])$ p - количество точек в $T_1.$
\\
Возьмем произвольное разбиение $\forall(T:d(T)<\sigma).$
Пусть $T_2=T \cup T_1;$ тогда $T \subset T_2, T_1 \subset T.$
\\
1). Если $\exists(\vartriangle_i \subset T)[$в $\vartriangle_i$ нет $x_i].$ Тогда $M_i \vartriangle x_i = 0.$
\\
2).Если в $\vartriangle_i$ из T есть точки разбиения $T_1(n_j)$, тогда посчитаем разность:
$$
\overline{S}(f,T)-\overline{S}(f,T_2)=M_i\vartriangle x_i - \sum_{j=1}^{n_i}M_{ij}\vartriangle x_{ij} = \sum_{j=1}^{n_i}(M_i - M_{ij})\vartriangle x_{ij} \leq 
$$
$$
\leq (M_i - M_{ij}=\sup_{[a;b]}f(x)-\sup_{[x_i;x_{i+1}]_ni}f(x) \leq\omega(f,[a;b])=\Omega) \leq
$$
$$
\leq\Omega\sum_{j=1}^{n}\vartriangle x_{ij} =\Omega\vartriangle x_i<\Omega d(T)<\Omega\sigma=\frac{\epsilon}{2p}
$$
На каждом из подотрезков, рассматриваемого типа, внутри которых есть точки из $T_1$, меньше p.
\\
Поэтому
$$
\overline{S}(f,T)-\overline{S}(f,T_2)<p\cdot\frac{\epsilon}{2p}=\frac{\epsilon}{2}
$$
Т.к. $T_2$ - измельчение разбиения $T_1$, то по свойству №3 верхних сумм Дарбу:
$$
\overline{S}(f,T_2)\leq\overline{S}(f,T_1)<\overline{s}+\frac{\epsilon}{2}
$$
Отсюда: $\overline{S}(f,T_1)-\overline{s}<\frac{\epsilon}{2}.$
$$\overline{S}(f,T)<\frac{\epsilon}{2}+\overline{S}(f,T_2)<\frac{\epsilon}{2}+\overline{s}+\frac{\epsilon}{2}<\overline{s}+\epsilon$$
\dokno




\subsection{Признак Дарбу существования интеграла}
\begin{teorema}
Для того, чтобы $f \in R[a;b]$ НиД чтобы $\underline{\overline{s}}=\overline{\overline{s}}$
\end{teorema}

\dokvo
\subsubsection{Необходимость}
дано: $f\in R[a;b]$ доказать $\underline{\overline{s}}=\overline{\overline{s}}$
\\
$$f\in R[a;b] \Rightarrow \exists(\overline{s}=\int_{a}^{b}f(x)dx) \Rightarrow \forall(\epsilon>0)\exists(\sigma>0)\forall(T)[d(T)<\sigma \Rightarrow \overline{s}-\epsilon<S(f,(T,\varphi))<\overline{s}+\epsilon]$$ 
По свойству №2 сумм Дарбу:
$$
\overline{S}(f,T)=\sup_{T}S(f,(T,\varphi)); \underline{S}(f,T)=\inf_{T}S(f,(t,\varphi))
$$
Поэтому из того, что 
$$
[\overline{s}-\epsilon<S(f,(T,\varphi))<\overline{S}+\epsilon] \Rightarrow \overline{S}-\epsilon \leq \underline{S}(f,T) \leq S(f,(T,\varphi)) \leq \overline{S}(f,T) \leq \overline{s}+\epsilon
$$
По определению: $\overline{\underline{s}}=\sup \underline{S}(f,T); \overline{\overline{s}}=\inf \overline{S}(f,T)$
Значит:
$
\underline{I}=\sup \underline{S}=\int_{a}^{b}f(x)dx
$
$
\overline{I}=\inf \overline{S}=\int_{a}^{b}f(x)dx
$
\dokno

\subsubsection{Достаточность}
дано:$\tilde{\underline{s}}=\overline{\tilde{s}}$
доказать: $f\in R[a;b]$
Пусть $\tilde{\underline{s}}=\overline{\tilde{s}}=\tilde{s}$
Докажем, что $\tilde{s}=\int_{a}^{b}f(x)dx$
Теорема Дарбу: $\lim_{d(T) \to 0} \underline{S}(f,T)=\underline{I}=I$
\\
$\lim_{d(T) \to 0} \overline{S}(f,T)=\overline{I}=I$
$$
\forall(\epsilon>0)\exists(\sigma_1>0)\forall(T)[d(T)<\sigma_1|\Rightarrow \overline{s}-\epsilon < \underline{S}(f,T)<\overline{s}+\epsilon]
$$
$$
\forall(\epsilon>0)\exists(\sigma_2>0)\forall(T)[d(T)<\sigma_2|\Rightarrow \overline{s}-\epsilon < \overline{S}(f,T)<\overline{s}+\epsilon]
$$
Пусть $\sigma=min(\sigma_1 и \sigma_2).$ Тогда:
$$
\forall(T:d(T)<\sigma)[(\overline{s}-\epsilon<\underline{S}(f,T)<\overline{s}+\epsilon) \wedge (\overline{s}-\epsilon<\overline{S}(f,T)<\overline{s}+\epsilon)]
$$
По свойству №1 сумм Дабу:
$$
\overline{s}-\epsilon < \underline{S}(f,T)\leq S(f,(T,\varphi))\leq \overline{S}(f,T) < \overline{s}+\epsilon|\Rightarrow
$$
$$
\Rightarrow \forall(\epsilon>0)\exists(\sigma>0)\forall(T:d(T)<\sigma)[
\overline{s}-\epsilon < S(f,(T,\varphi)) < \overline{s} + \epsilon]\Rightarrow
$$
$$
\Rightarrow [|S(f,(T,\varphi))-\overline{s}| < \epsilon] \Rightarrow
 $$
$$
 \Rightarrow f\in R[a;b]
$$
\dokno

\subsubsection{Следствие}
$$
f\in R[a;b]\Leftrightarrow \forall(\epsilon>0)\exists(\sigma>0)\forall(T)[d(T)<\sigma \Rightarrow \sum_{i=1}^{n(T)}\omega(f_i,\vartriangle_i)\vartriangle x_i < \epsilon]
$$
или используя определение предела, переформулируем:
$$
f \in R[a;b] \Leftrightarrow \lim_{d(T) \to 0} \sum_{i=1}^{n(T)}\omega(f,\vartriangle_i)\vartriangle x_i = 0
$$


\subsection{Свойства интеграла Римана}
Следующие два свойства фактически дополняют определение:

\begin{equation}
\intl_a^a f(x) dx = 0
\end{equation}

\begin{equation}
\intl_a^b f(x) dx = - \intl_b^a f(x) dx
\end{equation}

Ещё два свойства характеризуют интеграл как линейный оператор на пространстве интегрируемых функций.
Аддитивность:

\begin{multline}
\intl_a^b (f+g)(x) dx = \lim_{d(T)\to 0}S(f+g,(T,\xi))= 
\\=
\lim_{d(T)\to 0}S(f,(T,\xi))+\lim_{d(T)\to 0}S(g,(T,\xi))=\intl_a^b f(x) dx+\intl_a^b g(x) dx
\end{multline}

Однородность ($c$ -- константа):

$$
\intl_a^b (cf)(x)dx = \lim_{d(T)\to 0}S(cf,(T,\xi))=c\lim_{d(T)\to 0}S(f,(T,\xi))=c\intl_a^b f(x) dx
$$

Предостережём читателя: столь же красивой формулы для интеграла от произведения функций нет.

Введём вспомогательное определение.

\begin{opr}
Сужением разбиения $(T,\xi)$, содержащего среди точек деления $c$ и $d$, отрезка $[a;b]$ на подотрезок $[c;d]\subset[a;b]$ называется разбиение $(T_1,\xi)$, точками деления которого являются точки деления $T$, лежащие на отрезке $[c;d]$, а отмеченными точками -- соответствующие отмеченные точки разбиения $T$.
\end{opr}

Если функция интегрируема на отрезке, то она интегрируема и на любом подотрезке.

\dokvo

Пусть $f\in R[a;b]$, $[\alpha;\beta]\subset[a;b]$.
Рассмотрим те разбиения, в которые входят точки $\alpha$ и $\beta$, и положим $a=\xi_1$, $\alpha=\xi_p$, $\beta=\xi_q$, $b=\xi_n$
Тогда по необходимому и достаточному условию интегрируемости
$$
\forall(\epsilon>0)\exists(\delta>0)\forall((T,\xi):d(T)<\delta)\left[\sum_{i=1}^n \omega(f,\Delta_i)\Delta x_i < \epsilon\right]
$$
Так как колебание функции на отрезке есть величина положительная, а $1<p<q<n$ то
$$
\sum_{i=p}^q \omega(f,\Delta_i)\Delta x_i < \sum_{i=1}^n \omega(f,\Delta_i)\Delta x_i < \epsilon
$$
Пусть $T_1$ -- сужение разбиения $T$ на отрезок $[\alpha,\beta]$.
Но ${\sum\limits_{i=p}^q \omega(f,\Delta_i)\Delta x_i<\epsilon}$ -- сумма колебаний функции $f$, соответствующая разбиению $T_1$.
Значит, выполнено необходимое и достаточное условие интегрируемости для функции $f$ на отрезке $[a;b]$.

\dokno

Если функция интегрируема на отрезке, то этот отрезок можно разбить на две части, и сумма интегралов на частях будет равна интегралу на отрезке:
$$
f\in R[a;b], c\in [a;b] \Rightarrow \intl_a^b f(x)dx = \intl_a^c f(x) dx + \intl_c^b f(x) dx
$$

\dokvo

Тот факт, что интегралы на подотрезках существуют, вытекает из предыдущего свойства.

Рассмотри теперь бесконечно измельчающуюся последовательность разбиений $\{(T_n,\xi^{(n)})\}$, таких, что $c$ -- одна из точек деления.

По определению \ref{eqiv_opr_opr_intl} 
$$
\{S(f,(T,\xi))\} \to \intl_a^b f(x) dx
$$

Если мы обозначим через $T_n'$ и $T_n''$ сужения $T_n$ на $[a;c]$ и $[c;b]$ соответственно, то получим
$$
S(f,(T_n,\xi^{(n)}))=S(f,(T_n',\xi'^{(n)}))+S(f,(T_n'',\xi''^{(n)}))\to \intl_a^c f(x) dx + \intl_c^b f(x) dx
$$

\dokno

Вернёмся теперь к вопросу об интегрировании произведения функций.
Здесь имеет место лишь неконструктивное утверждение: произведение двух интегрируемых на отрезке функций интегрируемо на этом отрезке, т. е.
$$
\{f,g\}\subset R[a;b] \Rightarrow (f\cdot g)\in R[a;b]
$$

\dokvo

Так как $f$ и $g$ интегрируемы на $[a;b]$, то они ограничены на $[a;b]$.
Значит,
$$
\exists(M>0)\forall(x\in[a;b])[f(x)\leq M, g(x)\leq M]
$$
Оценим теперь колебание произведения функций $fg$ на $\Delta_i$, положив $\{x',x''\}\subset[a;b]$:
\begin{multline*}
|f(x')g(x')-f(x'')g(x'')|=
|g(x')(f(x')-f(x'')+f(x'')(g(x')-g(x''))|\leq
\\\leq
|g(x')|\cdot|f(x')-f(x'')|+|f(x'')|\cdot|g(x')-g(x'')|\leq
M\omega(f,\Delta_i)+M\omega(g,\Delta_i)
\end{multline*}

Значит,
$$
\sum_{i=1}^n M\omega(fg,\Delta_i) \Delta x_i \leq M \sum_{i=1}^n \left( \omega(f,\Delta_i) + \omega(g,\Delta_i) \right)\Delta x_i 
$$

Но выражение справа сколь угодно мало по необходимому и достаточному условию интегрируемости, значит, и выражение слева сколь угодно мало (сумма колебаний неотрицательна), значит, снова применив необходимое и достаточное условие интегрируемости, получим, что $(f\cdot g)\in R[a;b]$.

\dokno

Введём теперь определение неотрицательной и неположительной части функций:

\begin{opr}
$$
f_+(x)=\left\{
\begin{array}{l}
f(x), \mbox{~если~} f(x) > 0 \\
0,    \mbox{~если~} f(x) \leq 0
\end{array}
\right.
$$
\end{opr}


\begin{opr}
$$
f_-(x)=\left\{
\begin{array}{l}
f(x), \mbox{~если~} f(x) < 0 \\
0,    \mbox{~если~} f(x) \leq 0
\end{array}
\right.
$$
\end{opr}

Легко убедиться, что
$$
f_+(x)+f_-(x)=f(x)
$$

$$
f_+(x)-f_-(x)=|f(x)|
$$

Неотрицательная и неположительная части интегрируемой функции интегрируемы, т. е.
$$
f\in R[a;b] \Rightarrow \{f_+,f_-\}\in R[a;b]
$$

\dokvo

Заметим, что колебание неотрицательной (неположительной) части функции на некотором отрезке не превосходит колебания самой функции на данном отрезке.
Пусть $T$ - разбиение отрезка $[a;b]$ и ${\sum\limits_{i=1}^n \omega(f,\Delta_i)<\epsilon}$.
Тогда
$$\sum\limits_{i=1}^n \omega(f_+,\Delta_i)\Delta x_i <\sum\limits_{i=1}^n \omega(f,\Delta_i)\Delta x_i <\epsilon$$
$$\sum\limits_{i=1}^n \omega(f_-,\Delta_i)\Delta x_i <\sum\limits_{i=1}^n \omega(f,\Delta_i)\Delta x_i <\epsilon$$

Применив необходимое у достаточное условие интегрируемости функции, получим, что $\{f_+,f_-\}\in R[a;b]$

Как следствие, модуль интегрируемой функции сам является интегрируемым:
$$
f \in R[a;b] \Rightarrow |f| \in R[a;b] 
$$

Обратное, однако, неверно.
Пример -- функция $f(x)=\frac{1}{2}-D(x)$, где $D(x)$ -- функция Дирихле.

Наконец, докажем следующее свойство:
$$
\{f,g\}\subset R[a;b], \forall(x\in[a;b])[f(x)\leq g(x)] \Rightarrow \intl_a^b f(x)dx < \intl_a^b g(x) dx
$$

\dokvo

Интерграл есть предел интегральных сумм.
Но, так как $f(x)\leq g(x)$, то
$$
S(f,(T,\xi))\leq S(g,(T,\xi))
$$
Переходя к пределу при $d(T)\to 0$, получим требумое неравенство.

\dokno

Как следствие, интеграл любой непрерывной положительной функции положителен:
$$
f(x)\in R[a;b], \forall(x\in[a;b])[f(x)>0] \Rightarrow \intl_a^b f(x) dx > 0
$$

Более того,
$$
f\in R[a;b] \Rightarrow \intl_a^b f(x) dx \leq \intl_a^b |f(x)| dx 
$$

\subsection{Первая теорема о среднем}
\begin{teorema}
Пусть $\{f,\varphi\}\subset R[a;b]$, $\varphi$ сохраняет знак на $[a;b]$, $m=\inf\limits_{[a;b]}f(x)$, $M=\sup\limits_{[a;b]}f(x)$.
Тогда 
$$
\exists(\mu\in[m;M])\left[
\intl_a^b f(x)\varphi(x) dx=\mu\intl_a^b \varphi(x) dx
\right]
$$
\end{teorema}

\dokvo
Не теряя общности, будем доказывать для случая, когда $\varphi(x)$ положительна на $[a;b]$.
(В противном случае -- просто вынести минус единицу за знак интеграла.)

Так как
$$
\forall(x\in[a;b])[m\leq f(x)\leq M]
$$

умножив на $\varphi(x)$, имеем

$$
m\varphi(x)\leq f(x)\varphi(x) \leq M\varphi(x)
$$

Интегрируем (помним свойства интеграла Римана!)

$$
\intl_a^b m\varphi(x) dx \leq \intl_a^b f(x)\varphi(x) dx \leq \intl_a^b M\varphi(x)
$$

Если $\intl_a^b\varphi(x) dx =0$, то $\varphi(x)\equiv 0$ на $[a;b]$, следовательно, $\intl_a^b\varphi(x)f(x) dx=0$ и $\mu$ можно брать любым.

В противном случае на $\intl_a^b\varphi(x)f(x)dx$ можно разделить:
$$
m\leq\frac{\intl_a^b\varphi(x)dx }{\intl_a^b\varphi(x)f(x)dx }\leq M
$$

Условию теоремы удовлетворяет
$$
\mu=\frac{\intl_a^b\varphi(x)dx}{\intl_a^b\varphi(x)f(x)dx}
$$

\dokno

Как следствие, если $f$ непрерывна на $[a;b]$, то, в силу теоремы о промежуточном значении, 
$$
\exists(\xi\in[a;b])[f(\xi)=\mu]
$$
то есть 

$$
\exists(\xi\in[a;b])\left[
\intl_a^b f(x)\varphi(x) dx=f(\xi)\intl_a^b \varphi(x) dx
\right]
$$

Если пойти дальше и положить $\varphi(x)\equiv 1$, получим

$$
\exists(\xi\in[a;b])\left[
\intl_a^b f(x) dx=f(\xi)(b-a)
\right]
$$


\subsection{Вторая теорема о среднем}
\begin{teorema}
Пусть $\varphi:[a;b]\to\R$, $\varphi$ монотонна, $f\in R[a;b]$.
Тогда
$$
\exists(\xi\in[a;b])\left[\intl_a^b f(x)\varphi(x)dx=\varphi(a)\intl_a^\xi f(x) dx+\varphi(b)\intl_\xi^b f(x) dx\right]
$$
\end{teorema}

\subsection{Простейшие классы интегрируемых функций}
\subsubsection{1.Клас непрерывных функций}
\begin{teorema}
Непрерывная на отрезке функция всегда интегрируема на нём.
\end{teorema}

\dokvo
По следствию из теоремы Кантора:
$$
\forall(\epsilon>0)\exists(\sigma>0)\forall(T)[d(T)<\sigma|\Rightarrow\omega_i<\frac{\epsilon}{b-a}]
$$
Поэтому:
$$
\sum_{i=1}^{n}\omega_i\cdot\vartriangle x_i < \sum_{i=1}^{n} \frac{\epsilon}{b-a}\vartriangle x_i = \frac{\epsilon}{b-a}\sum_{i=1}^{n}\vartriangle x_i = \frac{\epsilon}{b-a}\cdot (b-a)=\epsilon
$$
\\
Значит, $\forall(\epsilon>0)\exists(\sigma>0)\forall(T)[d(T)<\sigma|\Rightarrow\sum_{i=1}^{n}\omega_i\vartriangle x_i < \epsilon$, т.е. выполняется НиД уловие интегрируемости функции. Значит, любая непрерывная функция - интегрируема.
\dokno

\subsubsection{2.Класс кусочно-непрерывных функций}
\begin{opred}
Функция кусочно-непрерывна на отрезке, если она непрерывна во всех точках этого отрезка, за исключением конечного числа точек.
$PC[a;b]=Peace Continues=$[кусочно-непрерывная]
\end{opred}

\begin{teorema}
Если $f\in PC[a;b]$ f(x) - ограничена, то $f \in R[a;b]$
\end{teorema}

\dokvo
т.к. f(x) - ограничена, то $\exists(M>0)\forall(x\in[a;b])[|f(x)|\leq M]$. Пусть $A=a_1,...,a_k$ - точки разрыва функции f(x).
\\
Возьмем $\forall(\epsilon>0).$
\\
Пусть $\mu=\min_{i\neq j}|a_i-a_j|$ и $(\sigma_1>0)[\sigma_1<\min(\mu,\frac{\epsilon}{8mk})]$
\\
Окружим точки разрыва интервалом радиуса $\sigma_1$. Эти окрестности пересекаться не будут!!!
\\
Пусть G-объединение этих окрестностей $K=[a;b]\ G.$ G-промежутки $\Rightarrow$ K-компакт. В силу теоремы Кантора f(x) - равномерно непрерывна на компакте K, если:
$$
\exists(\sigma_2>0)\forall(x',x''\in K)[|x'-x''|<\sigma_2|\Rightarrow|f(x')-f(x'')|<\frac{\epsilon}{2(b-a)}]
$$
Без ограничения общности будем считать, что $\sigma_2<\frac{1}{2}\cdot \sigma_1.$
\\
Возьмём произвольные разбиение $T:d(T)<\sigma_2$
$$
\sum_{i=1}^{n}\omega_i\vartriangle x_i = \sum_{i=1;\vartriangle_i\notin G}^{n}\omega_i\vartriangle x_i + \sum_{i=1;\vartriangle_i \cap G \neq 0}^{n}
\omega_i\vartriangle x_i$$
\\
Оценим каждое слагаемое отдельно:
$$
1) \sum_{i=1;\vartriangle_i\in K}^{n}\omega_i\vartriangle x_i<(\omega_i<\frac{\epsilon}{2(b-a)})<\frac{\epsilon}{2(b-a)}+\sum_{i=1}^{n}\vartriangle x_i = \frac{\epsilon}{2(b-a)}\cdot (b-a)=\frac{\epsilon}{2}
$$

$$
2) \omega_i < 2\mu; (\omega_i=|\sup-\inf|\leq |\sup|+|\inf| < \mu+\mu=2\mu)
$$
тогда:
$$
\sum_{\vartriangle_i\in G}\omega_i\vartriangle x_i \leq k(\sigma_1+2\sigma_2)\cdot 2\mu < (\sigma_2<\frac{1}{2}\cdot\sigma_1) <
$$
$$
< k\cdot\sigma_1\cdot2\mu < 4\mu k \min(\mu,\frac{\epsilon}{8mk})=\frac{\epsilon}{2}
$$
Тогда,
$$
\sum_{i=1}^{n}\omega_i\vartriangle x_i=\sum_{\vartriangle_i\in k}^{n}\omega_i\vartriangle x_i +\sum_{\vartriangle_i \in G}^{n}\omega_i\vartriangle x_i<\frac{\epsilon}{2}+\frac{\epsilon}{2} = \epsilon,
$$
т.е. f(x) удовлетворяет определению $\Rightarrow f(x)\in R[a;b]$
\dokno

\subsubsection{3.Класс монотонных функций}
\begin{teorema}
Если $f:[a;b]\to \R$ - монотонна, то $f\in R[a;b]$
\end{teorema}
\dokvo
Возьмем $\forall(\epsilon>0).$ Пусть $\sigma=\frac{\epsilon}{|f(b)-f(a)|},(f(a)\neq f(b))$
\\
Если $f(a)=f(b)$, то, в силу монотонности, $f(x)=\cos t.$ Возьмем $\forall(T):d(T)<\sigma$
$$
\sum_{i=1}^{n}\omega(f,\vartriangle_i)\vartriangle x_i =(1)=\sum_{i=1}^{n}|f(x_i)-f(x_{i-1})|\vartriangle x_i \leq (2) \leq d(T)\sum_{i=1}^{n}|f(x_i)-f(x_{i-1})|=
$$
$$
=d(T)|\sum_{i=1}^{n}(f(x_i)-f(x_{i-1}))|=d(T)|f(b)=f(a)|<
$$
$$
<\sigma\cdot|f(b)-f(a)|=\frac{\epsilon}{|f(b)-f(a)|}\cdot|f(b)-f(a)|=\epsilon
$$
\subsubsection{Пояснение(1)}
$\omega(f,\vartriangle_i)=|f(x_i)-f(x_{i-1})|$, в силу монотонности f(x).
\subsubsection{Пояснение(2)}
$\vartriangle x_i\leq d(T),$ т.к. $d(T)\max\vartriangle x_i$
\\
Получили, что f(x) подходит под определение, следовательно $f(x)\in R[a;b]$
\dokno

\subsubsection{Замечание}
Если PC - функции могут иметь лишь конечное число точек разрыва, то монотонные функции могут иметь бесконечное число точек разрыва.
\\
Например:
$
f(x)\left\{
\begin{array}{l}
0,x=0 \\
\frac{1}{2^n},\frac{1}{2^n}< x < \frac{1}{2^{n+1}}
\end{array}
\right. (n =1,2...)
$

\subsubsection{Упражнение}
Доказать, что монотонная функция может иметь не более чем счетное число точек разрыва.
\\
(см. Соболев, Покорный, Аносов, краткий курс матана ч.1 стр. 112-113)




\subsection{Формула Ньютона-Лейбница}
Связывает определение определенных и неопределенных интегралов.
\begin{opred}
F-обобщённая первообразная для f(x) на некотором промежутке, если F(x) - дифференцируема на этом промежутке, исключая некоторое конечное число точек, и всюду, где F(x) диффренцируема $[F'(x)=f(x)]$
\end{opred}
\subsubsection{Лемма 1}
$f\in R[a;b]$ - непрерывна в $x_0\in[a;b].$ Тогда $F(x)=\int_{a}^{b}f(t)dt$ - дифференцируема в $x_0$ и $F'(x_0)=f(x_0)$
\dokvo
$x_0,\vartriangle x: x_0+\vartriangle x\in[a;b]$

$$
\vartriangle F=F(x_0+\vartriangle x)-F(x_0)=\int_{a}^{x_0+\vartriangle x}f(t)dt - \int_{a}^{x_0}f(t)dt=
$$

$$
=\int_{x_0}^{x_0+\vartriangle x}f(t)dt = \mu(\vartriangle x)\cdot\vartriangle x
$$
\\
(по теореме о среднем),где $m(\vartriangle x)\leq\mu(\vartriangle x)\leq M(\vartriangle x),$
\\
$(M=\sup_{x\in\vartriangle_i}f(x),m=\inf_{x\in\vartriangle_i}f(x)),$
\\
где $\vartriangle_i=[x_0;x_0+\vartriangle x]$
\\
$\frac{\vartriangle F}{\vartriangle x}=\mu(\vartriangle x)$
\\
При $\vartriangle_i \to x_0$  $\vartriangle x \to 0 |\Rightarrow \vartriangle_i \to x_0,$ т.е. $[x_0;x_0+\vartriangle x]\to x_0,$ где $x_0$-просто точка.
\\
f(x) - непрерывна в $x_0$, $\mu(\vartriangle x)\in[\inf f(x),\sup f(x)]$
\\
$\forall(\epsilon>0)\exists(\sigma>0)[|\vartriangle x|<\sigma\Rightarrow|\mu \vartriangle x - f(x_0)|<\epsilon]$
\\
Перейдём к $\lim (\vartriangle x \to 0).$ Получим:
$$
\lim_{\vartriangle x\to 0}\mu(\vartriangle x) = f(x_0)
$$

$$
\lim_{\vartriangle x\to 0}\frac{\vartriangle F}{\vartriangle x}=\lim_{\vartriangle x\to 0}\mu(\vartriangle x)=f(x_0)\Rightarrow F'(x)=f(x_0)
$$
\dokno

\subsubsection{Замечание}
если f(x) непрерывна на всём отрезке [a;b], то эта лемма верна для $\forall(x\in[a;b])$

\subsubsection{Следствие 1}
если f(x) - непрерывна на [a;b], то $F(x)=\int_{a}^{x}f(t)dt$ - первообразная для f(t) на [a;b]. Поскольку любые 2 первообразные отличаются лишь на константу, то множество первообразных можно представить в виде $\int_{a}^{x}f(t)dt+C$

\subsubsection{Следствие 2}
Если $f\in PC[a;b],$ то $\int_{a}^{x}f(t)dt$ - обобщенная первообразная от функции f(t).

\subsubsection{Теорема}
$ f(x) \in PC[a;b]\models>\int\limits_a^b f(x)$, $dx= f(b)- f(a) $
где $F$-обобщенная первообразная для $f(t)$
($ f(x)$-$f(a)$ $= \Bigl. F(x) \Bigr|_a^b$
$-f(x)$ в подстановке от а до в)
Доказательство:

в силу следствия 2 из леммы1 обобщенная первообразная $ f(x)$ выражается

Пусть

$ F(x)= $ $\int_a^b f(t)\,dt+C$ 

тогда

$ f(a)= $ $\int_a^b f(t)\,dt+C=C$

$f(x)=$ $\int_a^b f(t)\,dt+C=$$\int_a^b f(t)\,dt+$$f(x)$ 

Отсюда:

$\int_a^b f(t) dt = $ $f(b)-$ $f(x)$

Th1-основная Th интегрального исчисления.

Замечание:

Не следует думать, что любой определённый интеграл можно вычислить подобным оброзом фуции,первообразные которых не выражаються через элементарные функции.

Например:

$\int_a^b f(x) dx-$
не выражается через формулу Ньютона - Лейбница, но он ведь существует!Т.е. их как-то можно выразить в числах,посчитать
 
\subsection{Формула интегрирования по частям для определённого интеграла}
Взятие определённого интеграла по частям применяют в тех же случаях, что и неопределённого.
Сформулируем теорему, являющуюся следствием из формулы Ньютона-Лейбница.

\begin{teorema}
Пусть функции $u$ и $v$ непрерывно дифференцируемы на $[a;b]$.
Тогда
$$
\intl_a^b (uv')(x) dx=(uv)(x)\Bigl.\Bigr|_a^b-\intl_a^b(u'v)(x)dx
$$
\end{teorema}

\dokvo
$$
(uv)'(x)=(u'v)(x)+(uv')(x)
$$

Проинтегрировав на $[a;b]$, получаем:

$$
\intl_a^b (uv)'(x)dx=\intl_a^b (u'v)(x) dx + \intl_a^b (uv')(x) dx
$$
То есть

$$
\intl_a^b (uv)'(x)dx-\intl_a^b (u'v)(x) dx = \intl_a^b (uv')(x) dx
$$


По формуле Ньютона-Лейбница
$$
\intl_a^b (uv)'(x)dx = (uv)(x)\Bigl.\Bigr|_a^b
$$

\dokno


\subsection{Замена переменной в определенном интеграле}
\subsection{Понятие о приближенных методах вычисления определённых интегралов}
В технических задачах вычислять определённый интеграл по формуле Ньютона-Лейбница или, тем более, по определению часто бывает очень сложно и не нужно: требуется только определённая точность.
В этих случаях подынтегральную функцию заменяют функцией более простой, как правило, кусочно-непрерывной.

Пусть $f$ -- функция, которую надо численно проинтегрировать на $[a;b]$, через $g$ обозначим функцию, которой будем заменять $f$.

Метод первый -- метод прямоуголника -- фактически повторяет определение, но "идёт не до конца":
строится разбиение $T$ с точками деления $x_0, ..., x_n$; эти же точки принимаются за отмеченные на отрезках, левыми (правыми) концами которых они являются (одна точка, конечно, остаётся лишней).
Таким образом, при методе прямоугольника функция $g$ принимает вид (в случае правых концов):
$$
g(x)=f(x_i), \mbox{~где~} x_{i-1}<x\leq x_i
$$

Метод трапеции предполагает замену функции на ломаную с вершинами $(x_i,f(x_i))$.

Однако на практике наиболее часто используется метод Симпсона, или метод парабол.
Он основан на том, что неизвестные коэффициенты функции $g_i(x)=a_i x^2 + b_i x + c_i$ можно восстановить по трём точкам, принадлежащим грайику этой функции.
Отрезок $[a;b]$ разбивают на $n=2m$ частей, а затем на отрезках $[x_{2i};x_{2i+2}]$ заменяют параболами, проходящими через точки $(x_{2i},f(x_{2i})$, $(x_{2i+1},f(x_{2i+1})$, $(x_{2i+2},f(x_{2i+2})$.





\section{Приложения определённого интеграла}
\subsection{Аддитивная функция промежутка}
\begin{opred}
Пусть $F:[a;b]^2 \to \R$. $F$ называется аддитивной, если
\begin{equation}\label{opr_additive}
\forall(\{\alpha,\beta,\gamma\}\subset[a;b])[F(\alpha,\beta)=F(\alpha,\gamma)+F(\gamma,\beta)]
\end{equation}
\end{opred}
Заметим, что аддитивной функцией промежутка такую функцию называют потому, что часто удобно считать $(\alpha,\beta)$ промежутком.

\begin{zamech}\label{zamech_additive_func_promezh}
$f(\alpha,\alpha)=0$, так как $F(\alpha,\beta)=F(\alpha,\alpha)+F(\alpha,\beta)$.
Аналогично доказывается, что $F(\beta,\alpha)=-F(\alpha,\beta)$.
\end{zamech}

Покажем теперь, что с аддитивной функцией можно связать некоторую обычную функцию.
Это сделать очень легко -- достаточно зафиксировать $\alpha=a$:
$$
f(x)=F(a,x)
$$
Тогда приращение $f(\beta)-f(\alpha)$ запишется в виде:
$$
f(\beta)-f(\alpha)=F(a,\beta)-F(a,\alpha)=F(\alpha,\beta)
$$

Найденная связь обратима: аддитивную функцию можно определить через разность приращений.

\begin{primer}
Пусть $F(x)=\intl_a^x f(t)dt, f\in R[a;b]$ и 
$$
\forall(\{\alpha,\beta\}\subset[a;b])[\Phi(\alpha,\beta)=F(\beta)-F(\alpha)]
$$.
Тогда $\Phi(\alpha,\beta)=\intl_a^\beta f(t)dt-\intl_a^\alpha f(t)dt=\intl_\alpha^\beta f(t)dt$
\end{primer}

\begin{teorema}
Пусть дана аддитивная функция промежутка $[a;b]$ $F(\alpha,\beta)$, $\{\alpha,\beta\}\subset[a;b]$,
и функция $f\in R[a;b]$ такая, что
\begin{equation}\label{usl_teorema_ob_additive}
\forall(\{\alpha,\beta\}\subset[a,b]:\alpha<\beta)
[(\beta-\alpha)\inf\limits_{[\alpha;\beta]}f(x)\leq F(\alpha,\beta)\leq(\beta-\alpha)\sup\limits_{[\alpha;\beta]}f(x)]
\end{equation}
Тогда $F(\alpha,\beta)=\intl_\alpha^\beta f(t)dt$
\end{teorema}
\dokvo
Возьмём разбиение $T$ отрезка $[a;b]$ и обозначим $m_i=\inf\limits_{\Delta_i}f(x)$, $M_i=\sup\limits_{\Delta_i}f(x)$
Тогда из (\ref{usl_teorema_ob_additive}) следует, что
$$
m_i\Delta x_i\leq F(x_{i-1},x_i)\leq M_i\Delta x_i
$$
Суммируем:
$$
\sum_{i=1}^n m_i\Delta x_i\leq \sum_{i=1}^n F(x_{i-1},x_i)\leq \sum_{i=1}^n M_i\Delta x_i
$$
Слева и справа в этом равенстве -- нижняя и верхняя суммы Дарбу соответственно.
Так как $F$ -- аддитивная функция промежутка, то 
$$
\sum_{i=1}^n F(x_{i-1},x_i)=F(\alpha,\beta)
$$
Отсюда немедленно следует, что 
$$
F(\alpha,\beta)=\intl_\alpha^\beta f(t)dt
$$
\dokno


\subsection{Длина параметризованной кривой} 
\begin{opred}
Простая кривая на плоскости в $\R^2$ - образ непрерывного взаимнооднозначного отображения $ \varGamma:[a;b] \to \R^2$ ($\varGamma$ - "гамма") т.е. множество точек (x;y):{(x;y)=$\varGamma(t),f\in[a;b],(x;y)\in \R^2$}
\end{opred}
Часто удобно задавать $\varGamma$ покоординатно:
$$
\left\{
\begin{array}{l}
x=x(t) \\
y=y(t)
\end{array}
\right. \mbox{~- это параметрическое задание функции~}
$$

$
\left\{
\begin{array}{l}
x=x(t) \\
y=y(t)
\end{array}
\right.$ задает параметризованную кривую, если отрезок [a;b] допускает разбение на конечное число подотрезков 
$[t_0;t_1],...,[t_{n-1};t_n], где t_0=a, t_n=b$
образ каждого из которых при отображении $\varGamma$ является простой кривой.

\begin{opred}
Возьмем разбиение $T=\{t_0=a;t_1;...;t_n=b\}$.
Пусть $\varGamma_k=\varGamma(t_k)$. Соединим $\varGamma_{k-1}$ и $\varGamma_k$ отрезком. $\forall(k=\{1;n\})$
Получим ломаную, вписанную в параметризованную кривую, которая определяется уравнениями $x=x(t), y=y(t)$.
\end{opred}

\begin{opred}
	Параметризованная кривая $\varGamma$ - распремляемая, если множество длинн $\{l(L)\}$ ломанных, вписанных в $\varGamma$ - ограничено. В этом случае sup всех этих длинн называют длина кривой.
	$des: l(L)=\sup \{l(L)\}$
\end{opred}

\begin{teorema}
Пусть $T_1$- измельчений T соответственной ломаной. Тогда $l(L_1) \geqslant l(L)$
\end{teorema}
	
\dokvo
$l(L)=\sum x_i$, где $x_i$ - длина ломаной на $[t_{i-1};t_i]$, где $t_{i-1},t_i$ - точки разбиения.

Пусть $t_i$ - точки разбиения $T_1$.
Если на ($t_{i-1},t_i$) нет $t'$, тогда $x_i-{x_i}' = 0$
Если на ($t_{i-1},t_i$) есть $t'$, тогда ${x_i}'-x_i > 0$
Т.к. $l(L_1)=\sum {x_i'}$, тогда $l(L_1)-l(L)=\sum x_i - \sum {x_i}' \geqslant 0$
\dokno

\begin{opred}
	Если в определении кривеой x(t) и y(t) - непрерывно дифференцируемы, то такая кривая называется гладкой.
\end{opred}

\begin{teorema}
	Любая гладкая на [a;b] кривая спрямляема и длина её вычисляется по формуле:
	$l(\varGamma)= \sum \sqrt{(x'(t))^2+(y'(t))^2} dt$
\end{teorema}

\dokvo
Возьмем произвольное разбиение $T=\{t_0,t_1,...,t_n\}$ где $t_0=a, t_n=b$. Пусть $L$ - ломаная, соответсвующая этому разбиению, вписанная в кривую $\varGamma$.
Найдем длину ломаной:
$l(L)=\sum \sqrt{(x(t_i)-x(t_{i-1}))^2+(y(t_i)-y(t_{i-1}))^2}$ , где $(x(t_i);y(t_i))$ - координаты $t_i$
 $(x(t_{i-1});y(t_{i-1}))$ - координаты $t_{i-1}$.
 В силу непрерывной дифференцируемости x(t) и y(t) используем теорему Лагранжа:
$\exists (\eta_i , \zeta_i \in(t_{i-1};t_i))[x(t_i)-x(t_{i-1})=x'(\zeta_i)\vartriangle t_i \\
y(t_i)-y(t_{i-1})=y'(\eta_i) \vartriangle t_i]$
Тогда $l(L)= \sum \sqrt{(x'(\zeta_i))^2 \cdot (\vartriangle t_i)^2 + (y'(\eta_i))^2 \cdot (\vartriangle t_i)^2} = \vartriangle t_i \cdot \sum_{i=1}^{n} \sqrt{(x'(\zeta_i))^2+(y'(\eta))^2}$ т.к. x'(t) и y'(t) - непрерывны то $\exists(M)[|x'(t)\leqslant M|y'(t)\leqslant M]$.
Тогда $l(\varGamma) \leqslant \sqrt{2} M(b-a)$. Т.е. все ломаные, вписаные в кривую, ограничены по длине, что означает спрямленность кривой $\varGamma$.
\\
Пусть S(T,$\zeta$) - интегральная сумма, где $\zeta_i(t_{i-1},t_i) \cdot [x(t_i)-x(t_{i-1})=x'(\zeta_i)\vartriangle t_i]$ (по теореме Лагранжа) $S(f(T,\zeta)) = \sum_{i=1}^{n} \sqrt{(x'(\zeta_i))^2+(y'(\zeta_i))^2} \vartriangle t_i$.
\\
$\forall (\epsilon > 0) \exists (\delta>0) \forall(T)[d(T)<\delta \Rightarrow |l(L)-I|<\frac{\epsilon}{2}]$, где $I=\int_{a}^{b} \sqrt{(x'(t))^2+(y'(t))^2}dt$, т.к. $|\sqrt{a^2+b_1^2}-\sqrt{a^2+b^2}|\leqslant|b_1 - b|$, то $|\sqrt{(x'(\zeta_i))^2+(y'(\zeta_i))^2}-\sqrt{(x'(\zeta_i))^2+(y'(\eta_i))^2}| \leqslant |y'(\zeta_i) - y'(\eta_i)|\leqslant \omega(y',\eta_i)$.
\\
Тогда $|l(L)-S(f(T,\zeta))|=|\sum_{i=1}^{n}\sqrt{(x'(\zeta_i))^2+(y'(\zeta_i))^2} \vartriangle t_i - \sum_{i=1}^{n}\frac{a \cdot \vartriangle t_i}{\sqrt{(x'(\zeta_i))^2+(y'(\zeta_i))^2}}| \leqslant \sum_{i=1}^{n} \omega(y',\vartriangle_i)\vartriangle t_i$ т.к. y'-непрерывна $|\Rightarrow y'\in R[a;b]|\Rightarrow$ выполняет НиД условие интегрируемости при бесконечно малом $d(T)[\sum_{i=1}^{n}\omega(y',\vartriangle_i)\vartriangle t_i < \frac{\epsilon}{4}]$.
\\Кроме того $|S(T,\zeta)-I|<\frac{\epsilon}{4}$ при $d(T)\rightarrow 0$
\\
Отсюда:
\\
$|l(L)-I|\leqslant |l(L)-S(T,\zeta)|+S(T,\zeta)-I| < \frac{\epsilon}{4}+\frac{\epsilon}{4}=\frac{\epsilon}{2}$
\\
Среди всех ломаных, удовлетворяющих последнему неравенству, найдется ломаная с длиной l(L), которая отличается от длины кривой $l(\varGamma)$ на величину, меньшую, чем $\frac{\epsilon}{2}$.
В самом деле:
$l(\varGamma)=sup$ длинн ломаных, поэтому $\exists(T^*)$ $[0 \leqslant l(\varGamma)-l(L^*)<\frac{\epsilon}{2}]$.
\\
Измельчим $T^*$ так, чтобы $d(T^{**})<\delta$.
\\
Для соответвующей ломаной, в силу доказанного: $[|l(L^{**})-I|<\frac{\epsilon}{2}]$. Отсюда (при $d(T^{**})$):$|l(\varGamma)-I|\leqslant |l(\varGamma)-l(L^{**})|+|l(L^{**})-I|<\frac{\epsilon}{2}+\frac{\epsilon}{2}=\epsilon$.
В силу произвольности $\epsilon$ получим, что $l(\varGamma)=I=\int_{a}^{b}\sqrt{(x'(t))^2+(y'(t))^2}$
\dokno

\subsubsection{Следствие.}
Когда имеется простая кривая, заданная как график функции y=f(x), то её можно параметрищовать с помощью отображения $\varGamma:x \rightarrow(x,f(x))$. Полученная гладкая параметризованная кривая имеет в силу теоремы длину, равную $\int_{a}^{b}\sqrt{(x'(t))^2+(y'(t))^2}dt$

\subsubsection{Замечание 1.}
Если кривая параметризована двумя....
\subsection{Площадь поверхности вращения}
Пусть L - простая прямая плоскости, которая не пересекается с $O_X$.

\begin{opred}
	Поверхность вращения F - множество точек в $\R^3$, описываемое кривой L при вращении содержащей её плоскости $O_y$ вокруг $O_x$
\end{opred}
Прямую L будем считать параметризованной (с учётом того, что третья координата z(t) в точках кривой L равна 0:z=0)
\\
Параметризующее отображение $\varGamma$ имеет вид $\varGamma(t)=(x(t)); y(t);0), t \in [a;b]$.
Пусть $T=\{a=t_0;t_1;...;t_n=b\}$ - разбиение [a;b] и K - соответствующая разбиению T ломаная, вписаная в кривую L с помощью отображения $\varGamma$, т.е. вершины ломаной - точки $$\varGamma(t_k), k=\{0;n\}$$.
Обозначим через Ф - поверхность, которая получается из K вращения вокруг $O_x$.
\\
Пусть $S(\CYRF)$ - площадь поверхности.
Поверхность Ф определяется выбранным разбиением T. Будем говорить, что Ф вписана в поверхность вращения F.
\begin{opred}
F имеет площадь, если множество площадей, вписанных в неё поверхностей, построенных вышеописанным способом по всем разбиениям T имеет предел. Этот 
$$
\lim_{d(T)\to 0}S(\Phi)
$$
называется площадью поверхности F(des;S(F)).
\end{opred}
\\
Вращение каждого подотрезка создаёт усечённый конус. Площадь поверхности Ф-суммы площадей боковой поверхности усеченных конусов каждого подотрезка:
\\
$$S(\phi)=\pi\sum_{i=1}^{n}(y(t_{i-1})+y(t_i))\sqrt{(x(t_i)-x(t_{i-1}))^2+(y(t_i)-y(t_{i-1})^2},$$ 
\\
где $ y(t_{i-1}) и y(t_i)$ - радиусы оснований.
\\
Пусть $$x_i=x(t_i), y_i=y(t_i).$$ Тогда:
\\
 $$S(\phi)=\pi\sum_{i=1}^{n}(y_{i-1}+y_i)\sqrt{(x_i-x{i-1})^2+(y_i-y{i-1})^2};$$
$$\forall(t)[y(t)\geqslant0]$$

\begin{teorema}
Пусть F-поверхность, отвечающая гладкой простой кривой L(гладкая: x(t) и y(t) - непрерывно дифференцируемы). Тогда
$$
S(F)=2\pi\int_{a}^{b}y(t)\sqrt{(x'(t))^2+(y'(t))^2}dt
$$
\end{teorema}
\dokvo
(аналогично теореме о длине кривой)

\subsubsection{Упражнение:}
Доказать, что для эквивалентных параметризаций кривой L получим одинаковые площади поверхностей вращения.

\subsubsection{Замечание 1.}
Если кривая L не $\cap$ с $O_y$, то для поверхности вращения F, полученной из L при вращении плоскости $xO_y$ вокруг $O_y$ имеет место формула:
$$
S(F)=2\pi\int_{a}^{b}x(t)\sqrt{(x'(t))^2+(y'(t))^2}dt
$$

\subsubsection{Замечание 2.}
Пусть L - график неотрицательной непрерывной дифференцируемой функции, то отображение $\varGamma(x)=(x;y(x);0)$ задает гладкую параметризацию L. Тогда 
$$
S(F)=2\pi\int_{a}^{b}y(x)\sqrt{(x'(x))^2+(y'(x))^2}dx=2\pi\int_{a}^{b}y(x)\sqrt{1+(y'(x))^2}dx
$$
\subsection{Площадь фигуры}
\begin{opred}
Фигура на плоскости - любое множество точек на плоскости.
\end{opred}

\begin{opred}
Ограниченная фигура - фигура, которая целиком содержится в некотором круге ограниченного радиуса.
\\
Мы будем рассматривать только ограниченные фигуры.
\end{opred}

\begin{opred}
Говорят, что фигура $F_1$ вписана в $F_2$, если все точки $F_1 \in F_2$. Тогда $F_2$ описана вокруг $F_1$.	
\end{opred}
\\
Чтобы определить понятие площади, возьмем точки на $O_x$ и проведем прямые || $O_y$. Аналогично - на $O_y$. Получим плоскость разбитую на квадраты со стороной = 1. S каждого такого квадрата равна 1. Это квадраты ранга 1.
\\
Обозначим через $\sigma_1$ фигуру, составленную из квадратов ранга 1, полностью лежащих в F.
\\
Пусть $s_1$ - площадь $\sigma_1$.
Обозначим через $\Sigma_1$ фигуру, составленную из квадратов, имеющих с F непустое пересечение. $S_1$ - площадь $\Sigma_1$.
\\
Затем делим каждый квадрат 1 ранга на 100 маленьких квадратов со стороной 0,1; S каждого такого квадрата равна 0,01 (это квадраты второго ранга).
\\
Пусть $\sigma_2$ - фигура, состоящая из квадратов ранга 2, полностью лежащих в $F(s_2=S(\sigma_2));$ $\Sigma_2$-фигура, состоящая их квадратов ранга 2, имеющих с F непустое пересечение $(S_2=S(\Sigma_2))$
\\
В итоге, уменьшая площадь квадратов, получим последовательность фигур $\{\Sigma_n\}$ и $\{\sigma_n\}$. Очевидно, что при $n \to \infty$ $S_{k+1} \leqslant S_k$, а $s_{k+1}\geqslant s_k,$ при том 
$$\forall (k,m \in \{1;n\})[s_k \leqslant S_m].$$
Отсюда $\{S_n\}$ и $\{s_n\}$ - ограничены (т.к. $s_1 \leq s_n \leq S_1 \leq S_n): \{S_n\}$ ограничена снизу, $\{s_n\}$ сверху.
\\
Обозначим:
$$
S=\lim_{n\to \infty}S_n, s=\lim_{n \to \infty}s_n
$$

\begin{opred}
Фигура F - квадрируема, если S=s, и
$$
S(F)=\lim_{n\to \infty}\{S_n\},\lim_{n \to \infty}\{s_n\}
$$
\end{opred}

\subsubsection{Упражнение (монотонность площади)}
Доказать, что если $F_1$ и $F_2$ - квадрируемы, причем $F_1 \leq F_2$, то $[S(F_1) \leq S(F_2)]$

\subsubsection{Лемма 1.}
Пусть $F_1,F_2$ - квадрируемые фигуры, $F_1\cap F_2$ - квадрируемо и $S(F_1 \cap F_2)=0,$ тогда $F_1 \cup F_2$ - квадрируемо и $S(F_1+F_2)=S(F_1)+S(F_2).$
\dokvo
Рассмотрим  $F_1\cap F_2$ и $\{\Sigma\}$, где $\{\Sigma\}$ - квадраты ранга n, имеющие непустое пересечение с $F_1\cap F_2$.
\\
По условию: $F_1\cap F_2$ - квадрируема и $S(F_1\cap F_2)=0$ $\Rightarrow$ при достаточно большом n $[S(\Sigma_n)<\epsilon].$
\\
$\Sigma_n$ расширим до $\Sigma_n^1$ - фигуры, состоящей из квадратов ранга n, покрывающих $F^1.$
\\
$\Sigma_n$ расширим до $\Sigma_n^2$ - фигуры, состоящей из квадратов ранга n, покрывающей $F^2$.
\\
Можем считать(при необходимости увеличивая n), что:
$$
S(\Sigma_n^1)-S(F^1)<\frac{\epsilon}{2};
$$

$$
S(\Sigma_n^2)-S(F^2)<\frac{\epsilon}{2};
$$

$$
S(\Sigma_n^1 \cup \Sigma_n^2)=S(\Sigma_n^1)+S(\Sigma_n^2)-S(\Sigma_n^1 \cap \Sigma_n^2) \geq S(F^1)+S(F^2)-\epsilon
$$
\\
т.к.
$$
(\Sigma_n^1 \supset F_1, а F_2\subset\Sigma_n^2, то S(F_1) \leq S(\Sigma_n^1), S(F_2) \leq S(\Sigma_n^2))
$$

$$
S(\Sigma_n^1 \cup \Sigma_n^2) \leq S(\Sigma_n^1) + S(\Sigma_n^2) \leq (S(F^1)-\frac{\epsilon}{2})+(S(F^2)-\frac{\epsilon}{2})=S(F^1)+S(F^2)-\epsilon
$$
\\
Значит, $\lim_{n \to \infty}(\Sigma_n^1 \cup \Sigma_n^2)=S(F^1)+S(F^2)$
\\
Аналогично, и с $\sigma_n^1$, $\sigma_n^2$:
$$
\lim_{n \to \infty}(\sigma_n^1 \cup \sigma_n^2)=S(F^1)+S(F^2)
$$
Получается, фигура $F_1 \cup F_2$ по определению квадрируема и 
$$
S(F_1 \cup F_2) = S(F_1) + S(F_2)
$$
\dokno

\subsubsection{Упражнение}
Обобщить Лемму 1 на случай более двух фигур

\begin{opred}
Граничная точка фигуры - точка на плоскости (которая не может принадлежать фигуре) такая, что в круге любого радиуса с центром в этой точке содержатся как точки фигуры, так и точки дополнения фигуры до плоскости.
\end{opred}

\begin{opred}
Изолированная точка фигуры F - точка А, такая, что $\exists(\sigma > 0)[F \cap U_\sigma = A]$
\end{opred}

\begin{opred}
Внутренняя точка фигуры F - точка B, такая, что $\exists(\sigma > 0)[ U_\sigma \subset F]$
\end{opred}

\subsubsection{Пример}
Возьмем фигуру $F=\{(x;y):x^2+y^2<2\}\cup (4;0).$
Здесь: множество внутренних точек: $\{(x;y):x^2+y^2<2\}$
\\
множество граничных точек:
$\{(x;y):x^2+y^2=2\};$
\\
множество изолированных точек: (4;0).

\subsubsection{Лемма 2.}
Если граница фигуры F квадрируема, и $S(\delta F)=0$, то сама фигура квадрируема.

\begin{opred}
Граница фигуры - совокупность граничных точек ($\delta F$ - граница фигуры F)
\end{opred}

\dokvo
Пусть $\tilde{\Sigma_n}$ - фигура, составленная из квадратов ранга n, имеющих с $\delta F$ непустое пересечение. Тогда т.к. $S(\delta F) = 0,$ то $S(\tilde{\Sigma_n})<\epsilon$.
\\
Пусть $\sigma_n^0$ - фигура, составленная из квадратов ранга n, не вошедших в $\tilde{\Sigma_n}$ каждый из которых содержит внутренние точки из F.
\\
Очевидно $\sigma_n^0 \subseteq \sigma_n \subseteq F \subseteq \Sigma_n\subseteq (\tilde{\Sigma_n}\cup \sigma_n^0).$

$$
\lim_{n\to \infty}(S(\Sigma_n))=S(F), \lim_{n \to \infty}(S(\sigma))=S(F) \Rightarrow
$$
$\Rightarrow$ F-квадрируема.
\dokno

\subsubsection{Упражнение}
Доказать, что кривая, изображаюшая график непр. функции на отрезке, имеет нулевую S.

\begin{teorema}
Пусть f - непрерывная и неограниченная на [a;b] функция. Тогда криволинейная трапеция F, ограниченная сверху кривой y=f(x), снизу осью $O_x$ и с боков x=f,x=b - квадрируема и $S(F)=\int_{a}^{b} f(x) dx$
\end{teorema}

\dokvo
т.к. f(x) - непрерывная $\Rightarrow S(\delta F)=0 \Rightarrow$ (в силу Леммы 2) F-квадрируема.
$S(\alpha,\beta)=S(\beta)-S(\alpha)$ - в силу Леммы об аддитивности промежутка.
\\
Пусть m=min f(x), M=max f(x), то $m(\beta-\alpha) \leq S(\alpha,\beta) \leq M(\beta-\alpha),$ f(x) - непрерывна ($f\in R[a;b]$)
\\
Из теоремы об аддитивности функции ориентированного промежутка следует утверждение теоремы.
\dokno












\subsection{Объём тела вращения}
\begin{opred}
	Объем - это как площадь только в $\R^3$.
	Поэтому все определения объема эквивалентны определениям площади. Например:
	квадрируемость$\approx$кубируемость;
	граница$\approx$поверхность;
	фигура$\approx$тело, и т.д.
\end{opred}

\subsubsection{Лемма 3.}
Если $F_1,F_2$ - кубируемые тела такие, что $F_1\cap F_2$ - кубируемо и $V(F_1 \cap F_2)=0$, тогда $F_1\cup F_2$ - кубируема, и $V(F_1 \cup F_2)=V(F_1)+V(F_2)$
\dokvo
(см. Лемму 1)

\subsubsection{Лемма 4.}
Если поверхность тела F кубируема и $V(\delta F)=0)$, то F-кубируема.
\dokvo
(см. Лемму 2)

\subsubsection{Лемма 5.}
Граница тела вращения, определяемая непрерывной неотрицательной на [a;b] функции (т.е. тела, заметаемого вращением криволинейной трапеции x=a, x=b, y=f(x) вокруг $O_x$) имеет V=0.
\dokvo
(аналогично Упражнению после Леммы 2)

\subsubsection{Теорема 2.}
Объём тела вращения, определяемого непрерывной, неотрицательной на [a;b] функции y=f(x) вычисляется по формуле $V(F)=\pi\int_{a}^{b}f^2(x)dx$
\dokvo
Из Лемм 4,5 следует, что часть тела F, заключенного между плоскостями x=$\alpha;x=\beta$ кубируемо при $\forall(\alpha,\beta \in [a;b])$
\\
Пусть $V(\alpha,\beta)$ - объем этой части из Леммы 4 следует, что $V(\alpha,\beta)=V(\beta)-V(\alpha)$, где V($\alpha$)- объём тела вращения, ограниченного плоскостями x=a и x=b.
\\
Если $\alpha>\beta$, то получим, что $V(\alpha,\beta)$ - аддитивная функция промежутка. Очевидно, что:
$V_{мал.цилиндра}\leqslant V(\alpha,\beta) \leqslant V_{бол.цилиндра}$
\\
$\pi m^2(\beta-\alpha) \leqslant V(\alpha,\beta) \leqslant \pi M^2(\beta-\alpha)$,
где $m=min f(x), M=max f(x)$.
Тогда по теореме об аддитивной функции промежутка:
$V(F)=\pi\int_{a}^{b} f^2(x) dx$
\subsection{Понятие о несобственных интегралах}
Подобно тому, как мы распространяли понятие предела на случай, когда в выражении участвует бесконечность, можно распространить и понятие интеграла на бесконечные (неограниченные) криволинейные трапеции.
Такие интегралы называют несобственными.

\begin{opr}\label{opr_nesobstv_intl_1}
Пусть $y=f(x)$, $f:[a;+\infty]\to\R$, $\forall(b>a)[f\in R[a;b]]$.
Если 
\begin{equation}\label{lim_nesobstv_intl_1}
\exists \lim_{b\to\infty}\intl_a^b f(x)dx\neq \pm\infty
\end{equation}
то говорят, что интеграл
\begin{equation}\label{nesobstv_intl_1}
\intl_a^{+\infty}f(x)dx
\end{equation}
сходится и равен пределу (\ref{lim_nesobstv_intl_1}), в противном случае -- что интеграл расходится.
Интеграл по бесконечному промежутку называют несобственным интегралом первого рода.
\end{opr}

Запись
$$
\intl_a^{+\infty}f(x)dx=\infty
$$
не используют.

\begin{primer}
$$
\intl_0^{+\infty}\frac{dx}{1+x^2}=\lim_{b\to+\infty}\intl_0^{b}\frac{dx}{1+x^2}=
\lim_{b\to+\infty}(\arctg b - \arctg 0)=\frac{\pi}{2}
$$
\end{primer}

Несобственный интеграл для $-\infty$ в качестве предела вводится аналогично.

Рассмотри теперь другой тип несобственных интегралов - интегралы от неограниченных функций, называемые несобственными интегралами второго рода.

\begin{opr}
Пусть функция $f$ неограниченно возрастает при стремлении справа к точке $a$:
$$
\lim_{x\to a +}f(x)=\infty
$$
и
$$
\forall(\epsilon>0)[f\in R[a+\epsilon;b]]
$$
то полагают
$$
\intl_a^bf(x)dx=\lim_{\epsilon\to 0+}\intl_{a+\epsilon}^b f(x)dx
$$
если предел в правой части равенства существует.
В противном случае говорят, что интеграл расходится.
\end{opr}
Случай для стремления слева к правой границе определяется аналогично.

\begin{opr}
Точки $\pm\infty$ и точки, в которых подынтегральная функция неограниченно возрастает, если они принадлежат промежутку интегрирования, называются особенностями интеграла.
\end{opr}
Если в интеграле несколько особенностей, то его разбивают на сумму интегралов, каждый из которых имеет не более одной особенности.

\begin{opr}
Несобственный интеграл
$$
\intl_a^b f(x) dx
$$
(первого или второго рода) называется абсолютно сходящимся, если сходится интеграл
$$
\intl_a^b |f(x)| dx
$$

\end{opr}


