\begin{opred}
Пусть $F:[a;b]^2 \to \R$. $F$ называется аддитивной, если
\begin{equation}\label{opr_additive}
\forall(\{\alpha,\beta,\gamma\}\subset[a;b])[F(\alpha,\beta)=F(\alpha,\gamma)+F(\gamma,\beta)]
\end{equation}
\end{opred}
Заметим, что аддитивной функцией промежутка такую функцию называют потому, что часто удобно считать $(\alpha,\beta)$ промежутком.

\begin{zamech}\label{zamech_additive_func_promezh}
$f(\alpha,\alpha)=0$, так как $F(\alpha,\beta)=F(\alpha,\alpha)+F(\alpha,\beta)$.
Аналогично доказывается, что $F(\beta,\alpha)=-F(\alpha,\beta)$.
\end{zamech}

Покажем теперь, что с аддитивной функцией можно связать некоторую обычную функцию.
Это сделать очень легко -- достаточно зафиксировать $\alpha=a$:
$$
f(x)=F(a,x)
$$
Тогда приращение $f(\beta)-f(\alpha)$ запишется в виде:
$$
f(\beta)-f(\alpha)=F(a,\beta)-F(a,\alpha)=F(\alpha,\beta)
$$

Найденная связь обратима: аддитивную функцию можно определить через разность приращений.

\begin{primer}
Пусть $F(x)=\intl_a^x f(t)dt, f\in R[a;b]$ и 
$$
\forall(\{\alpha,\beta\}\subset[a;b])[\Phi(\alpha,\beta)=F(\beta)-F(\alpha)]
$$.
Тогда $\Phi(\alpha,\beta)=\intl_a^\beta f(t)dt-\intl_a^\alpha f(t)dt=\intl_\alpha^\beta f(t)dt$
\end{primer}

\begin{teorema}
Пусть дана аддитивная функция промежутка $[a;b]$ $F(\alpha,\beta)$, $\{\alpha,\beta\}\subset[a;b]$,
и функция $f\in R[a;b]$ такая, что
\begin{equation}\label{usl_teorema_ob_additive}
\forall(\{\alpha,\beta\}\subset[a,b]:\alpha<\beta)
[(\beta-\alpha)\inf\limits_{[\alpha;\beta]}f(x)\leq F(\alpha,\beta)\leq(\beta-\alpha)\sup\limits_{[\alpha;\beta]}f(x)]
\end{equation}
Тогда $F(\alpha,\beta)=\intl_\alpha^\beta f(t)dt$
\end{teorema}
\dokvo
Возьмём разбиение $T$ отрезка $[a;b]$ и обозначим $m_i=\inf\limits_{\Delta_i}f(x)$, $M_i=\sup\limits_{\Delta_i}f(x)$
Тогда из (\ref{usl_teorema_ob_additive}) следует, что
$$
m_i\Delta x_i\leq F(x_{i-1},x_i)\leq M_i\Delta x_i
$$
Суммируем:
$$
\sum_{i=1}^n m_i\Delta x_i\leq \sum_{i=1}^n F(x_{i-1},x_i)\leq \sum_{i=1}^n M_i\Delta x_i
$$
Слева и справа в этом равенстве -- нижняя и верхняя суммы Дарбу соответственно.
Так как $F$ -- аддитивная функция промежутка, то 
$$
\sum_{i=1}^n F(x_{i-1},x_i)=F(\alpha,\beta)
$$
Отсюда немедленно следует, что 
$$
F(\alpha,\beta)=\intl_\alpha^\beta f(t)dt
$$
\dokno

