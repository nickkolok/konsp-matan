Рассмотрим вопрос об интегрировании четырёх типов дробей, называемых элементарными.

\opred
Элементарной дробью I типа называется дробь вида
\begin{equation}\label{elem_drob_I}
\frac{a}{x+p}
\end{equation}

Такая дробь интегрируется очевидным образом:
$$
\int\frac{a}{x+p}dx=a\int\frac{d(x+p)}{x+p}=a\ln|x+p|+C
$$

\opred
Элементарной дробью II типа называется дробь вида
\begin{equation}\label{elem_drob_II}
\frac{a}{(x+p)^n},~n\geq 2
\end{equation}

Такая дробь тоже легко интегрируется:
$$
\int\frac{a}{(x+p)^n}dx=a\int\frac{d(x+p)}{(x+p)^n}=\frac{a}{1-n}(x+p)^{-n+1}+C
$$

\opred
Элементарной дробью III типа называется дробь вида
\begin{equation}\label{elem_drob_III}
\frac{ax+b}{x^2+px+q},~ D=p^2-4q<0
\end{equation}

Такая дробь интегрируется с помощью замены
\begin{equation}
t=x+\frac{p}{2},~ dt=dx, ~ \alpha^2=\frac{-D}{4}, ~\beta=b-\frac{ap}{2}
\end{equation}
Имеем:
$$
\int \frac{ax+b}{x^2+px+q}=\int\frac{a\left(x+\frac{p}{2}\right)+b-\frac{ap}{2}}{x^2+2\frac{p}{2}+\frac{p^2}{4}+q-\frac{p^2}{4}}dx=$$$$=
\int\frac{at+\beta}{t^2+\alpha^2}dt=\frac{a}{2}\int\frac{d(t^2)}{t^2+\alpha^2}+\beta\int\frac{dt}{t^2+\alpha^2}=$$$$=\frac{a}{2}\ln|t^2+\alpha^2|+\frac{\beta}{\alpha}\arctg\frac{t}{\alpha}+C
$$
Возвращение к исходным переменной и параметрам предоставляем читателю.
\opred
Элементарной дробью IV типа называется дробь вида
\begin{equation}\label{elem_drob_IV}
\frac{ax+b}{(x^2+px+q)^k},~ D=p^2-4q<0,~k\geq 2
\end{equation}

Такая дробь тоже интегрируется с помощью замены (которая, вообще говоря, часто применяется при интегрировании выражений, содержащих квадратный трёхчлен)
\begin{equation}
t=x+\frac{p}{2},~ dt=dx, ~ \alpha^2=\frac{-D}{4}, ~\beta=b-\frac{ap}{2}
\end{equation}
Имеем:
$$
\int \frac{ax+b}{(x^2+px+q)^k}=
\int
	\frac	{a\left(x+\frac{p}{2}\right)+b-\frac{ap}{2}}
		{\left(x^2+2\frac{p}{2}+\frac{p^2}{4}+q-\frac{p^2}{4}\right)^k}
dx=
$$$$=
\int\frac{at+\beta}{(t^2+\alpha^2)^k}dt=
\frac{a}{2}\int\frac{d(t^2)}{(t^2+\alpha^2)^k}+\beta\int\frac{dt}{(t^2+\alpha^2)^k}=
$$$$=
\frac{a}{2(1-k)}(t^2+\alpha^2)^{1-k}+\beta\int\frac{dt}{(t^2+\alpha^2)^k}
$$

Рассмотрим теперь интеграл
$$J_k=\int\frac{dt}{(t^2+\alpha^2)^k}$$
Преобразуем его:
$$J_k=\frac{1}{\alpha^2}\int\frac{t^2+\alpha^2-t^2}{(t^2+\alpha^2)^k}dt=
$$$$=
\frac{1}{\alpha^2}\int\frac{dt}{(t^2+\alpha^2)^{k-1}}-\frac{1}{\alpha^2}\int\frac{t^2}{(t^2+\alpha^2)^k}dt=
\frac{1}{\alpha^2}J_{k-1}-\frac{1}{\alpha^2}\int\frac{t^2}{(t^2+\alpha^2)^k}dt
$$
Первое слагаемое вычисляется рекуррентно (помним, что $J_1$ -- интеграл от элементарной дроби III типа), займёмся вторым слагаемым:
$$
\int\frac{t^2}{(t^2+\alpha^2)^k}dt=
$$$$
\begin{zamena}[c|c]
u=t & du = dt\\
dv=\frac{t dt}{(t^2+\alpha^2)^k} & v=\int \frac{tdt}{(t^2+\alpha^2)^k}=\frac{1}{2}\int\frac{d(t^2)}{(t^2+\alpha^2)^k}=\frac{1}{2}\cdot\frac{1}{1-k}(t^2+\alpha^2)^{1-k}
\end{zamena}
$$$$
\frac{t}{2}\cdot\frac{1}{1-k}(t^2+\alpha^2)^{1-k}-\int \frac{1}{2}\cdot\frac{1}{1-k}\cdot\frac{tdt}{(t^2+\alpha^2)^{k-1}}
$$
Как вычисляется последний интеграл, мы уже знаем.
Таким образом, интегрирование элементарной дроби IV типа со знаменателем степени $k$ рекуррентно сводится к интегрированию элементарной дроби IV типа со знаменателем степени $k-1$, а, значит, на некотором шаге к интегрированию элементарной дроби III типа.
