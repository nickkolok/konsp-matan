\subsubsection{Определение непрерывности функиции в точке по Коши.}

\opred

\fXR, $x_0 \in X$.
Функция $f$ непрерывна в точке $x_0$, если
$$
\forall(\epsilon>0)\exists(\delta>0)[|x-x_0|<\delta \Rightarrow |f(x)-f(x_0)|<\epsilon].
$$

Или, что то же самое, но с применением окрестностей:

$$
\forall(\epsilon>0)\exists(\delta>0) [f(U_{\delta}(x_0) \cap X) \subset U_{\epsilon}(f(x_0))]
$$

Или, что то же самое:

$$
\forall(\epsilon>0)\exists(\delta>0) [f(U_{\delta,X}(x_0)) \subset U_{\epsilon}(f(x_0))]
$$

И, наконец, полностью перейдя в термины окрестностей:

$$
\forall(U \in O(f(x_0)))\exists(V \in O_X(x_0)) [f(V) \subset U]
$$

\subsubsection{Замечание 1.}

Вдумчивый читатель легко заметит, что это опреденление похоже на определение предела в точке, в котором проколотые окрестности заменены на непроколотые. Несколькими строками ниже мы рассмотрим вопросч о связи непрерывности функции, её предела и её значения в данной точке.

\subsubsection{Замечание 2.}

Если $x_0$ - изолированная точка множества $X$, то
$$
 \exists(U \in O(x_0))[U \cap X = \{x_0\}] \Rightarrow f(U)=\{f(x_0)\}],
$$
т. е. найдётся окрестность точки $x_0$, образом которой явялется единственная точка, и функция $f$ в точке $x_0$ непрерывно. Однако никаких содержательных результатов этот случай не даёт, и потому в дальнейшем мы, как правило, будем рассматривать непрерывность функции, заданной на множестве точек, лишь в предельных точках этого множества.

\subsubsection{Критерий непрерывности функции в точке.}

\fXRx.
$f$ непрерывна в $x_0$ тогда и только тогда, когда 

$$
\lim_{x\to x_0}f(x)=f(x_0)
$$

\subsubsection{Следствие 1.}

\fXRx.
$f$ непрерывна в $x_0$ тогда и только тогда, когда знак предела и знак функции коммутируют, т. е. 
$$
\lim_{x \to x_0} f(x) = f(\lim_{x \to x_0}x)
$$

\subsubsection{Следствие 2.}

\fXRx, $f$ непрерывна в $x_0$, $\Delta y = f(x_0+\Delta x)-f(x_0)$.
$\Delta x \to 0$ тогда и только тогда, когда $\Delta y \to 0$

\subsubsection{Определение непрерывности в точке по Гейне.}

\fXRx.
$f$ непрерывна в $x_0$, если 
$$
\forall(\{x_n\}:x_n \in X \cap x_n \to x_0)[f(x_n)\to f(x_0)]
$$

Обозначив $\Delta x = x_n-x_0$, $\Delta x = f(x_n)-f(x_0)$, можем сформулировать:

$$
\Delta x \to 0 \Rightarrow \Delta y \to 0
$$

