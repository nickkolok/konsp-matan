\subsubsection{Теорема Кантора о равномерной непрерывности.}
\fXR, $X$ - компакт и $f$ непрерывна на $X$.
Тогда $f$ равномерно непрерывна на $X$.

\subsubsection{Следствие 1.}

Если $f:[a;b] \to \R$ непрерывна на отрезке $[a;b]$, то она равномерно непрерывна на этом отрезке.

\subsubsection{Следствие 2.}

Если $f:[a;b] \to \R$ непрерывна на отрезке $[a;b]$, то
$$
\forall(\epsilon > 0) \exists (\delta > 0) \exists(a_1, b_1 : a < a_1 < b_1 < b, b_1 - a_1<\delta)[\omega(f,[a_1,b_1]<\epsilon],
$$
или, что то же самое,
$$
\forall(\epsilon > 0) \exists (отрезок \Delta \subset [a;b])[\omega(f,\Delta)<\epsilon]
$$
т. е. найдётся подотрезок, на котором колебание функции меньше любого наперёд заданного.

\subsubsection{Замечание.}


