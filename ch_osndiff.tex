\section{Дифференциальное исчисление функции одной независимой переменной}
\subsection{Определение производной и дифференциала, связь между этими понятиями}
\subsection{Связь между понятиями дифференцируемости и непрерывности функций}
\subsection{Дифференцирование и арифметические операции}
\subsection{Теорема о производной сложной функции. Инвариантность формы первого дифференциала}
\subsection{Теорема о производной обратной функции}
\subsection{Производные основных элементарных функций. Доказательство}
\subsection{Касательная к кривой. Геометрический смысл производной и дифференциала}
\subsection{Физический смысл производной и дифференциала}
\subsection{Односторонние и бесконечные производные}
\subsection{Производные и дифференциалы высших порядков}
...

\section{Основные теоремы дифференциального исчисления}
\subsection{Понятие о локальном экстремуме функции}
\opred

\fXRx.
Точка $x_0$ называется точкой локального минимума, а значение в ней - локальным минимумом функции $f$, если
$$
\exists (U(x_0)) \forall(x \in U(x_0) \cap X)[f(x) \geq f(x_0)]
$$

\opred

\fXRx.
Точка $x_0$ называется точкой локального максимума, а значение в ней - локальным максимумом функции $f$, если
$$
\exists (U(x_0)) \forall(x \in U(x_0) \cap X)[f(x) \leq f(x_0)]
$$

\opred

\fXRx.
Точка $x_0$ называется точкой строгого локального минимума, а значение в ней - строгим локальным минимумом функции $f$, если
$$
\exists (\mathring{U}(x_0)) \forall(x \in \mathring{U}(x_0) \cap X)[f(x) > f(x_0)]
$$

\opred

\fXRx.
Точка $x_0$ называется точкой строгого локального максимума, а значение в ней - строгим локальным максимумом функции $f$, если
$$
\exists (\mathring{U}(x_0)) \forall(x \in \mathring{U}(x_0) \cap X)[f(x) < f(x_0)]
$$

\opred

Точками локального экстремума называются вместе точки локального минимума или максимума.

\opred

Локальными экстремумами называются вместе локальные минимумы или максимумы.

\opred

Точками строгого локального экстремума называются вместе точки строгого локального минимума или максимума.

\opred

Строгими локальными экстремумами называются вместе строгие локальные минимумы или максимумы.

\opred

\fXR, $x_0$ - двусторонняя предельная точка $X$.
Если $x_0$ - точка локального экстремума, то она называается точкой внутреннего локального экстремума.



\subsection{Теорема Ферма}
\subsubsection{Теорема Ферма о производной в точке локального экстремума.}

\fXR, $f$ дифференцируема в точке внутреннего локального экстремума $x_0$.
Тогда $f'(x_0)=0$.

\subsubsection{Замечание 1.}

В невнутренней точке локального экстремума производная может, вообще говоря, быть не равной нулю.
Пример: $f:[-1;1]\to \R$, невнутренний локальный максимум $x_0 = 1$, $f'(x_0)=2$.

\subsubsection{Замечание 2.}
Теорема Ферма необратима.
Пример: $f:\R\to\R$, $f(x)=x^3$, $f'(0)=0$, но $f$ не имеет локальных экстремумов.




\subsection{Теорема Ролля}
\subsubsection{Теорема.}

Если $f:[a;b]\to \R$ такова, что

1) $f$ непрерывна на $[a;b]$;

2) $f$ дифференцируема на $(a;b)$;

3) $f(a)=f(b)$,

то $\exists(c \in (a;b))[f'(c)=0]$.

\subsubsection{Замечание 1.}

Геометрическая интерпретация теоремы: пусть кривая задана функцей $y=f(x)$.
Тогда между любыми двумя точками с равными ординатами, лежащими на данной кривой, найдётся такая точка, в которой касательная к данной кривой параллельна оси абсцисс.

\subsubsection{Замечание 2.}

Условие (1) избыточно: т. к. уже требуется, чтобы $f$ была дифференцируема на $(a;b)$, достаточно потребовать непрерывности $f$ в $a$ и $b$. Остальные условия существенны.

\subsubsection{Следствие. Теорема о корнях производной.}

Между любых двух корней дифференцируемой функции лежит корень её производной.

\dokvo

Применим теорему Ролля к случаю, когда $f(a)=f(b)=0$.






\subsection{Теорема Лагранжа и следствия из нее}
\subsubsection{Теорема Лагранжа о промежуточном значении (о конечных приращениях).}

Если $f:[a;b]\to \R$ такова, что

1) $f$ непрерывна на $[a;b]$;

2) $f$ дифференцируема на $(a;b)$;

то $\exists(c \in (a;b))[f(b)-f(a)=f'(c)(b-a)]$.

\subsubsection{Замечание 1.}

Равенство $f(b)-f(a)=f'(c)(b-a)$ называют формулой Лагранжа или формулой конечных приращений.

\subsubsection{Замечание 2.}

Формулу Лагранжа можно записать и в другом виде, если положить $\theta=\frac{c-a}{b-a}$:

$$
f(b)-f(a)=f'(a+\theta(b-a))(b-a)
$$

Полагая $x=a, h=b-a$, имеем

$$
f(x+h)-f(x)=f'(x+\theta h)h
$$

\subsubsection{Следствие 1.}

Функция, имеющая на промежутке равную нулю производную, постоянная на нём.

\subsubsection{Следствие 2.}

Пусть на промежутке $X$ определены и дифференцируемы две функции $f$ и $g$, притом на концах промежутка, если они в него входят, $f$ и $g$ непрерывны.
Если $\forall(x \in X)[f'(x)=g'(x)]$, то $\forall(x \in X)[f(x)-g(x)=const]$.

\subsubsection{Следствие 3.}

Функция, имеющая на промежутке ограниченную производную, равномерно непрерывна на нём.

\subsubsection{Следствие 4.}

Пусть $f:[a;b]\to \R$, $f$ непрерывна, $f$ дифференцируема на $(x_0;x_0+h)\subset [a;b]$.
Тогда правая производная $f$ в $x_0$ непрерывна.



\subsection{Теорема Коши}
\subsubsection{Теорема Коши.}

Пусть $f:[a;b]\to \R$, $g:[a;b]\to \R$, причём:

1) $f$ и $g$ непрерывны на $[a;b]$;

2) $f$ и $g$ дифференцируемы на $(a;b)$;

3)$\nexists (x \in (a;b))[g(x)=0]$

Тогда

$$
\exists (c \in (a;b))\left[ \frac{f(b)-f(a)}{g(b)-g(a)}=\frac{f'(c)}{g'(c)}\right].
$$

\subsubsection{Замечание 1.}

Теорема Коши не является следствием из теоремы Лагранжа; наоборот, теорема Лагранжа - частный случай теоремы Коши для $g(x)=x$.

\subsubsection{Замечание 2.}

Равенство $ \frac{f(b)-f(a)}{g(b)-g(a)}=\frac{f'(c)}{g'(c)}$ называют формулой конечных приращений Коши.



\section{Формула Тейлора}
\subsection{Формула Тейлора для многочлена}
\subsection{Формула Тейлора для произвольной функции. Различные формы остаточного члена формулы Тейлора}
\subsection{Локальная формула Тейлора}
\subsection{Формула Маклорена. Разложение по формуле Маклорена некоторых элементарных функций}
\subsection{Применение формулы Тейлора}
...

\section{Правило Лопиталя}
\subsection{Неопределённость. Виды неопределённостей}
Пусть даны две непрерывные на интервале $(a; b)$ функции $f(x)$ и $g(x)$, где $\{a; b\} \subset \overline{\mathbb{R}}$. Неопределённостью типа $\left[\frac{0}{0}\right]$ в точке $a$ называется предел 
\[
\lim_{x \to a+}\frac{f(x)}{g(x)}
\]
в случае, когда
\[
\lim_{x \to a+}f(x) = \lim_{x \to a+}g(x) = 0
\]
Аналогично определяются неопределённости вида $\left[\frac{\infty}{\infty}\right]$ и в точке $b$.

Другие виды неопределённостей сводятся к этим двум. Вообще говоря, неопределённость типа $\left[\frac{\infty}{\infty}\right]$ может быть сведена к типу $\left[\frac{0}{0}\right]$. Действительно, пусть
$$\lim_{x \to a+}f(x) = \lim_{x \to a+}g(x) = \infty$$
тогда
\[
\frac{f(x)}{g(x)}=\frac{\frac{1}{g(x)}}{\frac{1}{f(x)}}
\]
Однако при раскрытии неопределённостей возникает необходимость расcматривать их отдельно.

Неопределённость-произведение сводится к неопределённостям-частным двумя способами:

$$
[0 \cdot \infty]=\lim_{x\to x_0}(f(x) \cdot g(x))=\lim_{x\to x_0}\frac{f(x)}{\frac{1}{g(x)}}=\left[\frac{0}{0}\right]
$$

$$
[0 \cdot \infty]=\lim_{x\to x_0}(f(x) \cdot g(x))=\lim_{x\to x_0}\frac{g(x)}{\frac{1}{f(x)}}=\left[\frac{\infty}{\infty}\right]
$$

Неопределённости-степени сводятся с неопределённостям-произведениям (а затем - к неопределённостям-частным) через равенство 
$$
f(x) ^ {g(x)}=e^{g(x) \cdot \ln f(x)}
$$

Заметим, что это равенство, как и сам предел, имеет смысл лишь при $f(x)>0$.
Покажем, как раскрываются неопределённости-степени:

$$
[\infty ^0]=\lim_{x\to x_0}(f(x) ^{g(x)})=\lim_{x\to x_0}e^{g(x) \cdot \ln f(x)}=e^{\lim_{x\to x_0}(g(x) \cdot \ln f(x))}=e^{[\infty \cdot 0]}
$$

$$
[0^0]=\lim_{x\to x_0}(f(x) ^{g(x)})=\lim_{x\to x_0}e^{g(x) \cdot \ln f(x)}=e^{\lim_{x\to x_0}(g(x) \cdot \ln f(x))}=e^{-[0 \cdot \infty]}
$$

$$
[1 ^\infty]=\lim_{x\to x_0}(f(x) ^{g(x)})=\lim_{x\to x_0}e^{g(x) \cdot \ln f(x)}=e^{\lim_{x\to x_0}(g(x) \cdot \ln f(x))}=e^{[0 \cdot \infty]}
$$

Наконец, рассмотри раскрытие неопределённости-разности:

$$
[\infty - \infty]=\lim_{x\to x_0}(f(x) - g(x))=\lim_{x\to x_0}\left(f(x) \cdot g(x)\left(\frac{1}{f(x)}-\frac{1}{g(x)}\right)\right)=[\infty \cdot 0]
$$

Таким образом, раскрытие неопределённостей сведено к раскрытию неопределённостей-частных.

\subsection{Теорема Лопиталя}
%Докажем теперь правило Лопиталя для неопределённостей вида $\left[\frac{\infty}{\infty}\right].$

\subsubsection{Лемма об обратном пределе.}

Пусть даны функции $f$ и $g$, такие, что $lim_{x \to x_0}f(x)=lim_{x \to x_0}g(x)=+\infty$ и 
существует предел $lim_{x \to x_0} \frac{f(x)}{g(x)}$, 
тогда существует и предел $lim_{x \to x_0} \frac{g(x)}{f(x)}$.

\dokvo

По определению бесконечно большой в точке $x_0$ функции $\exists(V=\mathring{U}_\delta(x_0))\forall(x \in V$

\subsubsection{Замечание.}

Очевидно, вынос знака "минус" из-под знака предела не составляет сложности и не влияет на применимость правила.

\subsubsection{Теорема.}

Пусть даны функции $f$ и $g$, такие, что:
1)$f$ и $g$ определены на полуинтервале $(a;b]$
2)$f$ и $g$ дифференцируемы на полуинтервале $(a;b]$

\subsection{Применение правила Лопиталя}

\section{Применение дифференциального исчисления к исследованию функции одной переменной}
\subsection{Монотонные функции}
\subsubsection{Теорема.}

\fXR.
Для того, чтобы функция $f$ была неубываюшей (невозрастающей) на $X$, необходимо и достаточно, чтобы
$\forall(x \in X) [f'(x) \geq 0] (f'(x) \leq 0])$.

\dokvo 

Докажем теорему для случая неубывающей функции. Доказательство для случая невозрастающей оставляем читателю ввиду его аналогичности.

\neobh

$f$ - неубывающая функция. Возьмём $x$ и $h \neq 0$ такие, что $x\in X, x+h \in X$.

Если $h>0$, то, так как $f$ - неубывающая, $f(x+h) \geq f(x)$.
Если $h<0$, то $f(x+h) \leq f(x)$.
Значит, 

$$
\frac{ f(x+h) - f(x) }{ h } \geq 0
$$

Переходя к пределу, имеем

\[
\lim_{h\to 0} { \frac{ f(x+h) - f(x) }{ h } } = f'(x) \geq 0
\]

\dost

$f'(x) \geq 0$. Пусть $\{x_1,x_2\} \subset X, x_1 < x_2$.
Тогда на отрезке $[x_1, x_2]$ функция $f$ дифференцируема. Применим теорему Лагранжа:

$$
\exists(c \in [ x_1, x_2 ]) [f(x_2) - f(x_1) = f'(c)(x_2 - x_1)]
$$

Но $f'(c) \geq 0$ и $x_2 - x_1 > 0$. Значит, и $f(x_2) - f(x_1) \geq 0$, т. е. функция $f$ - неубывающая.

\dokno


\subsubsection{Замечание}
\fXR, $\forall(x \in X) [f'(x) > 0] (f'(x) < 0])$.
Рассуждениями, аналогичными рассуждениями в части доказательства достаточности условия предыдущей теоремы, можно показать, что в таком случае функция $f$ - возрастающая (убывающая).
Обратное, вообще говоря, неверно.
Например, возрастающая функция $f(x)=x^3$ имеет в точке $x=0$ нулевую производную:
$f'(x)=(x^3)'=3x^2$, $f'(0)=0$.

\subsubsection{Теорема}
\fXR, $f$ дифференцируема на $X$.
Для того, чтобы $f$ была возрастающей (убывающей), необходимо и достаточно, чтобы:

1) $\forall(x \in X) [f'(x)\geq 0]$

2) $\forall([a;b] \subset X)[f'(x)\not\equiv 0]$, т. е. чтобы ни на каком отрезке внутри $X$ $f'(x)$ не обращалась в тождественный нуль.

\dokvo 

Докажем теорему для случая возрастающей функции. Доказательство для случая убывающей оставляем читателю ввиду его аналогичности.

\neobh

$f(x)$ - возрастающая. Тогда в силу предыдущей теоремы выполнено первое условие.
Установим, что второе условие также выполнено.
\pp, т. е. что $\exists([a;b] \subset X)\forall(x \in [a;b])[f'(x)=0]$.
Тогда $f(x)$ на $[a;b]$ постоянна, и $f(a)=f(b)$, следовательно, $f$ не является возрастающей. Получили противоречие.

\dost

Так как $f'(x) \geq 0$, то по предыдущей теореме $f$ - неубывающая, т. е.
$\forall(x_1\in X, x_2 \in X : x_1<x_2)[f(x_2) \geq f(x_1)]$.

Докажем теперь, что $f(x_2) > f(x_1)$.
\pp, т. е. что $\exists(x_1\in X, x_2 \in X : x_1<x_2)[f(x_2) = f(x_1)]$.
Тогда $\forall(x\in [x_1; x_2])[f(x)=f(x_1)=f(x_2)]$, т. е. $\forall(x\in(x_1;x_2))[f'(x)=0]$, что противоречит второму условию теоремы.

\dokno


\subsection{Экстремумы функций} 

\fXR, $f$ непрерывна на $X$. Из теоремы Ферма вытекает, что точки локального экстремума следует искать среди корней производной и точек, принадлежащих $X$, в которых не существует конечная производная (т. е. производная не определена или бесконечна).

\opred
Корни производной функции называются стационарными точками этой функции.

\opred

Стационарные точки и точки, в которых не существует конечной производной, называются критическими точками первого рода или точками, подозрительными на экстремум.

\subsubsection{Замечание}
Условие $f'(x)=0$, являясь необходимым условием внутреннего локального экстремума дифференцируемой функции, не является достаточным. Классический пример -- функция $f(x)=x^3$ в точке $x=0$ имеет нулевую производную, но не имеет экстремума.

\opred 

Говорят, что при переходе через $x_0$ производная функции $f$ меняет знак с + на -, если
$$
\exists(\delta>0)(\forall(x\in(x_0-\delta;x_0))[f'(x)>0] \cap \forall(x\in(x_0;x_0+\delta))[f'(x)<0])
$$

Определения смены знака производной с - на + и отсутствия смены знака производной аналогичны; сформулировать их оставляем читателю.

\subsubsection{Теорема о смене знака производной}

\fXR, $x_0$ - критическая точка первого рода функции $f$ и функция $f$ дифференцируема в любой внутренней точке $X$, кроме, быть может, точки $x_0$.

Если при переходе через $x_0$ производная меняет знак с + на -, то $x_0$ -- точка локального максимума $f$, если с - на +, то $x_0$ -- точка локального минимума $f$, а если смены знака нет, то в точке $x_0$ нет и экстремума.

\dokvo (Для случая смены знака с + на -; случай смены знака с - на + предоставляем читателю.)
Возьмём $\forall(x \in U_\delta(x_0))$ и рассмотрим отрезок $A$ с концами $x$ и $x_0$. По теореме Лагранжа 
$$
\exists(c\in A)[f(x)-f(x_0)=f'(c)(x-x_0)].
$$

Если $x<x_0$, то $f'(c)>0, x-x_0<0$, откуда $f(x)-f(x_0)<0$.
Если $x>x_0$, то $f'(c)<0, x-x_0>0$, откуда $f(x)-f(x_0)<0$.
Имеем:
$$
\exists(\delta>0)\forall(x\in \mathring{U}_\delta(x_0))[f(x)<f(x_0)].
$$

Это в точности определение локального максимума.

\dokvo (Для случая постоянства знака производной.)
Знак разности $f(x)-f(x_0)$ будет зависеть от знака разности $x-x_0$, т. е. положения точки $x$ слева или справа от точки $x_0$, следовательно, в $x_0$ экстремума нет.

\dokno

\subsubsection{Теорема}
\fXR и функция $f$ имеет в точке $x_0\in X$ производные до n-ого порядка включительно, причём $f'(x_0)=f''(x_0)=...=f^{(n-1)}(x_0)=0$, $f^{(n)}(x_0) \neq 0$.
Тогда:

1) Если $n$ чётно, то в точке $x_0$ функция $f$ имеет экстремум, причём если $f^{(n)}(x_0)<0$, то это максимум, а если $f^{(n)}(x_0)>0$, то минимум.

2) Если $n$ нечётно, то в $x_0$ экстремума функции $f$ нет.

\dokvo (Для случая $f^{(n)}(x_0)>0$; случай $f^{(n)}(x_0)<0$ предоставляем читателю.)

Разложим $f(x)$ по формуле Тейлора в $x_0$ с остаточным членом в форме Пеано:

$$
f(x)=f(x_0)+\frac{f'(x_0)}{1!}(x-x_0)+...+\frac{f^{(n)}(x_0)}{n!}(x-x_0)^n+o(|x-x_0|^n)
$$

Так как $f'(x_0)=f''(x_0)=...=f^{(n-1)}(x_0)$ по условию теоремы, имеем:

$$
f(x)=f(x_0)+\frac{f^{(n)}(x_0)}{n!}(x-x_0)^n+o(|x-x_0|^n)
$$


$$
f(x)-f(x_0)=\frac{f^{(n)}(x_0)}{n!}(x-x_0)^n+o(|x-x_0|^n)
$$

При $x$, достаточно близких к $x_0$,

$$
\sgn(f(x)-f(x_0))=\sgn(f^{(n)}(x_0)(x-x_0)^n).
$$

Так как $f^{(n)}(x_0)>0$, то

$$
\sgn(f(x)-f(x_0))=\sgn((x-x_0)^n).
$$

Если $n$ чётно, то $\sgn((x-x_0)^n)=1$, т. е. $f(x)-f(x_0)>0$, что означает, что $x_0$ - точка минимума.

Если $n$ нечётно, то из последнего равенства имеем

$$
\sgn(f(x)-f(x_0))=\sgn(x-x_0),
$$

т. е. в любой сколь угодно малой окрестности $x_0$ разность $(x)-f(x_0)$ меняет знак, и экстремума функции нет.

\subsubsection{Замечание}

Для того, чтобы найти наибольшее (или наименьшее) значение непрерывной функции $f:[a;b] \to \R$, нужно найти её локальные максимумы (или минимумы) и сравнить значения функции в них со значениями функции на концах отрезка.

Впрочем, иногда просто вычисляют значения функции во всех критических точках.

\subsection{Выпуклые функции}
\subsection{Точки перегиба}
\subsection{Асимптоты кривых}
Пусть $L$ -- кривая, заданная уравнением $y=f(x)$, $x \in X$, $y \in Y$.

\opred

Кривая $L$ имеет бесконечные ветви, если по крайней мере одно из множеств $X$ или $Y$ является неограниченным.

Рассмотрим функцию $\rho(x)=\sqrt{x^2+f^2(x)}$, $x \in X$. Для того, чтобы кривая $L$ имела бесконечные ветви, необходимо и достаточно, чтобы $\rho$ была неограниченна на $X$.

\opred

Прямая $x=x_0$ называется вертикальной асимптотой кривой $L$, заданной уравнением $y=f(x)$, если $f(x) \to \pm \infty$ при $x \to x_0 \pm$, т. е. один из односторонних пределов функции бесконечен.

Горизонтальная асимптота -- это частный случай наклонной.

\opred

Пусть $f$ задана на неограниченном промежутке $X$. Прямая $y=kx+b$ называется наклонной асимптотой кривой $y=f(x)$, если
\[
\lim_{x\to + \infty}(f(x)-kx-b)=0
\]
или
\[
\lim_{x\to - \infty}(f(x)-kx-b)=0
\]

Иногда говорят об асимптоте на бесконечности, не указывая знак. Это означает, что асимптоты на $+\infty$ и $-\infty$ совпадают.

Чтобы выяснить, имеет ли кривая асимптоты и найти $k$ и $b$, разделим равенство

$$
f(x)-kx-b=o(x)
$$

(на $\pm \infty$) на $x$. Получим

$$
k=\frac{f(x)}{x}-\frac{b}{x}-o(x)=\frac{f(x)}{x}-o(x)
$$

\[
k=\lim_{x \to \pm \infty}\frac{f(x)}{x}
\]

\[
b=\lim_{x \to \pm \infty}(f(x)-kx)
\]

Очевидно, что рассуждения верны и в обратную сторону, т. е. прямая $y=kx+b$ будет асимптотой рассматриваемой кривой.

\subsubsection{Замечание}

При $\rho(x) \to \infty$, т. е. при удалении по бесконечной ветви кривой, расстояние $d(M)$ от точки $M$ кривой с координатами $(x; f(x))$ до асимптоты стремится к нулю.

Действительно, пусть $x=x_0$ -- вертикальная асимптота. Тогда $d(M)=|x-x_0|$.
Пусть теперь $y=kx+b$ - наклонная асимптота. Опустим из точки $M$ перпендикуляр $MH$ на асимптоту и перпендикуляр $MB$ на ось $Ox$ и обозначим через $A$ точку пересечения $MB$ с асимптотой. Тогда треугольник $AMH$ - прямоугольный, и катет $MH=d(M)$ в нём меньше гипотенузы $MA$, стремящейся к нулю.

Отметим, что кривая может пересекать свою асимптоту.

\subsection{Схема исследования функции}
\newcounter{N} % для создания списков, маркированных со стилями, нужен счётчик
\begin{list}{\arabic{N}.}{\usecounter{N}}

\item Находят область определения функции.

\item Проверяют функцию на чётность, нечётность и периодичность.

\item Находят точки пересечения графика функции с осями координат, если такие точки есть.

\item Исследуют функцию на непрерывность, определяют точки разрыва и их род.

\item Исследуют поведение функции при стремлении независимой переменной $x$ к точкам разрыва и границам области определения функции, включая, если это необходимо, $\pm \infty$.

\item Находят асимптоты (вертикальные и наклонные) и точки пересечения графика функции с асимптотами.

\item Находят критические точки первого рода.

\item Находят экстремумы.

\item Определяют интервалы монотонности функции.

Предыдущие три пункта удобно осуществить с помощью первой производной, сведя результаты в таблицу, где в первой строке указываются значения аргумента $x$ - интервалы и точки, во второй -- знак производной $f'(x)$, в третьей наклонной стрелкой вверх-вправо \nearrow или вниз-вправо \searrow указывается характер монотонности функции.

\item

\end{list}



