\opred

\fXRx.
Точка $x_0$ называется точкой локального минимума, а значение в ней - локальным минимумом функции $f$, если
$$
\exists (U(x_0)) \forall(x \in U(x_0) \cap X)[f(x) \geq f(x_0)]
$$

\opred

\fXRx.
Точка $x_0$ называется точкой локального максимума, а значение в ней - локальным максимумом функции $f$, если
$$
\exists (U(x_0)) \forall(x \in U(x_0) \cap X)[f(x) \leq f(x_0)]
$$

\opred

\fXRx.
Точка $x_0$ называется точкой строгого локального минимума, а значение в ней - строгим локальным минимумом функции $f$, если
$$
\exists (\mathring{U}(x_0)) \forall(x \in \mathring{U}(x_0) \cap X)[f(x) > f(x_0)]
$$

\opred

\fXRx.
Точка $x_0$ называется точкой строгого локального максимума, а значение в ней - строгим локальным максимумом функции $f$, если
$$
\exists (\mathring{U}(x_0)) \forall(x \in \mathring{U}(x_0) \cap X)[f(x) < f(x_0)]
$$

\opred

Точками локального экстремума называются вместе точки локального минимума или максимума.

\opred

Локальными экстремумами называются вместе локальные минимумы или максимумы.

\opred

Точками строгого локального экстремума называются вместе точки строгого локального минимума или максимума.

\opred

Строгими локальными экстремумами называются вместе строгие локальные минимумы или максимумы.

\opred

\fXR, $x_0$ - двусторонняя предельная точка $X$.
Если $x_0$ - точка локального экстремума, то она называается точкой внутреннего локального экстремума.


