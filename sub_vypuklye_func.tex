\opred
\fXR. $f$ называется вогнутой (выпуклой вниз, вогнутой вверх) на $X$, если
$$
\forall(x_1,x_2 \in X)\forall(\alpha\in[0;1])
[f((1-\alpha)x_1+\alpha x_2) \leq (1-\alpha)f(x_1)+\alpha f(x_2)]
$$
Это определение, хотя, как мы увидим далее, весьма удобно для доказательств, может вызвать вполне объяснимое недоумение.
Поясним его геометрический смысл.

Очевидно, что $\forall(x\in[x_1;x_2])\exists(\alpha\in[0;1])[x=(1-\alpha)x_1+\alpha x_2]$, то есть любую точку отрезка $[x_1;x_2]$ можно записать в том виде, которого требует определение.

Запишем теперь уравнение прямой (хорды графика функции), проходящей через точки $(x_1,f(x_1))$ и $(x_2,f(x_2))$:
$$
\frac{y-f(x_1)}{f(x_2)-f(x_1)}=\frac{x-x_1}{x_2-x_1}
$$
Или, в явном виде:
$$
y=f(x_1)+\frac{x-x_1}{x_2-x_1}(f(x_2)-f(x_1))
$$

С учётом равенства $x=(1-\alpha)x_1+\alpha x_2$ имеем:
$$
y=f(x_1)+\frac{(1-\alpha)x_1+\alpha x_2-x_1}{x_2-x_1}(f(x_2)-f(x_1))=(1-\alpha)f(x_1)+\alpha f(x_2)
$$
(приведение подобных, раскрытие скобок и прочую арифметику оставляем читателю).
Мы получили в точности правую часть неравенства из определения.
То есть определение можно понимать так: ``Для любой точки значение функции лежит ниже хорды, стягивающей любой участок графика функции, содержащий эту точку.''.

Если добавить в определение требование строго неравенства при $\alpha\in(0;1)$, то мы получим определение функции, строго выпуклой вниз.
Аналогично формулируется определение функции, выпуклой вверх:

\opred
\fXR. $f$ называется выпуклой (выпуклой вверх, вогнутой вниз) на $X$, если
$$
\forall(x_1,x_2 \in X)\forall(\alpha\in[0;1])
[f((1-\alpha)x_1+\alpha x_2) \geq (1-\alpha)f(x_1)+\alpha f(x_2)]
$$

Аналогично же вводится строгость и даётся графическое истолкование. Два вышеизложенных определения называют определениями выпуклости функции через хорды; свяжем теперь характер выпуклости со знаком второй производной.

\begin{teorema}\label{vypukl_th_1}
Пусть функция $f$ дважды дифференцируема на $(a;b)$. Тогда для того, чтобы $f$ была выпуклой вниз/вверх, необходимо и достаточно, чтобы
$$
\forall(x\in(a;b))[f''(x)\geq 0]~/~[f''(x)\leq 0]
$$
\end{teorema}

\dokvo
Доказываем для случая выпуклости вверх; случай выпуклости вниз оставляем читателю.

\neobh
Пусть функция $f$ выпукла вверх.
\pp, т. е. что
$$
\exists(x_0\in(a;b))[f''(x_0)<0].
$$
Возьмём $\forall(h:x_0\pm h\in(a;b))$. Тогда из определения выпуклой функции при $\alpha=\frac{1}{2},~x=x_0,~x_1=x_0-h,~x_2=x_0+h$ имеем:
$$
(f(x_0+h)-f(x_0))+(f(x_0-h)-f(x_0))\geq 0
$$

Применим к каждой из этих разностей формулу Лагранжа:

\begin{multline}\label{vypukl_Lagranzh}
(f(x_0+h)-f(x_0))+(f(x_0-h)-f(x_0))=
f'(x_0+\theta_1 h)h+f'(x_0-\theta_2 h)(-h)=\\=
h^2\left(  \frac{f'(x_0+\theta_1 h)-f'(x_0)}{\theta_1 h} + \frac{f'(x_0-\theta_2 h)-f'(x_0)}{-\theta_2 h} \right)  \geq 0
\end{multline}

Напомним, что в теореме Лагранжа $\theta_1,\theta_2 \in [0;1]$.
Мы предполагали, что $f''(x_0)<0$, тогда из определения второй производной
$$
\exists(h\in(a;b))\left[
\frac{f'(x_0+\theta_1 h)-f'(x_0)}{\theta_1 h}<0
~\cap~
\frac{f'(x_0-\theta_2 h)-f'(x_0)}{-\theta_2 h}<0
\right]
$$

Получили противоречие с (\ref{vypukl_Lagranzh}).

\dost
Известно, что $\forall(x\in(a;b))[f''(x)\geq 0]$.
Возьмём $\forall(x_1,x_2 \in (a;b) : x_1 < x_2)$ и $\forall(x\in(x_1;x_2))$.
Тогда $\exists(\alpha\in[0;1])[x=(1-\alpha)x_1+\alpha x_2]$.
Применим формулу Тейлора с остаточным членом в форме Лагранжа к точкам $x_1$ и $x_2$:
$$
f(x_1)=f(x)+f'(x)(x_1-x)+\frac{f''(c_1)}{2!}(x_1-x)^2
$$
$$
f(x_2)=f(x)+f'(x)(x_2-x)+\frac{f''(c_2)}{2!}(x_2-x)^2
$$
Здесь, напомним, $c_1\in[x_1;x],~c_2\in[x;x_2]$. Умножив первое равенство на $(1-\alpha)$, а второе на $\alpha$ и сложив, имеем:
$$
(1-\alpha)f(x_1)+\alpha f(x_2)=f(x)+f'(x)(x_1+\alpha x_1 - x + \alpha x + \alpha x_2 - \alpha x)+c,
$$
где
$$
c=\frac{f''(c_1)}{2}(x_1-x)^2 (1-\alpha)+\frac{f''(c_2)}{2}(x_2-x)^2 \alpha
$$
Легко видеть, что, раз $f''(x)\geq 0$, то и $c\geq 0$.
Значит, с учётом того, что $x=(1-\alpha)x_1+\alpha x_2$, 
$$
(1-\alpha)f(x_1)+\alpha f (x_2)=f(x)+f'(x)(-\alpha x_2 + \alpha x_2)+c
$$
т. е.
$$
(1-\alpha)f(x_1)+\alpha f (x_2) \leq f(x) 
$$
Это и есть определение выпуклости.
\dokno

\begin{teorema}\label{vypukl_th_2}
Пусть $\forall(x\in(a;b))\exists(f''(x))$. Для выпуклости вниз необходимо, а в случае непрерывности $f''(x)$ и достаточно, чтобы график функции $f$ лежал не ниже касательной к графику функции $f$, проведённой в точке $(x_0;f(x_0))$ для $\forall (x_0\in(a;b))$.
\end{teorema}

\dokvo

\neobh
Запишем уравнение касательной:
$$
y_K=f(x_0)+f'(x_0)(x-x_0)
$$
Обозначив $y=f(x)$ и применив формулу Тейлора с остаточным членом в форме Лагранжа, имеем:
$$
y-y_K=f(x)-f(x_0)-f'(x_0)(x-x_0)=f''(c)\frac{(x-x_0)^2}{2}
$$
Здесь $c$ лежит между $x$ и $x_0$.
По теореме \ref{vypukl_th_1} $f''(c)\geq 0$, значит, $y\geq y_K$.

\dost
\pp, т. е. что $f$ не выпукла вниз.
Тогда по теореме \ref{vypukl_th_1} $\exists(x_0\in(a;b))[f''(x_0)<0]$.
Т. к. $f''$ непрерывна, то $$\exists(\delta>0)\forall(x\in U_\delta(x_0)[f''(x)<0]$$
Но $y-y_K=f''(c)\frac{(x-x_0)^2}{2}$, т. е. $$\forall(x\in U_\delta(x_0))[y-y_K<0],$$ т. е. график функции лежит ниже касательной.
Получили противоречие.
\dokno Случай выпуклости вниз оставляем читателю.


