\section{Предварительные сведения}
\subsection{Равномерное стремление к пределу функции двух переменных}
\subsection{Непрерывность, дифференцируемость, интегрируемость предельной функции}
...

\section{Свойства интегралов, зависящих от параметра}
\subsection{Собственные интегралы, зависящие от параметра. Дифференцируемость интеграла по параметру в случае постоянных и переменных пределов интегрирования}
\subsection{Интегрируемость по параметру собственных интегралов, зависящих от параметра}
\subsection{Несобственные интегралы, зависящие от параметра. Равномерная сходимость несобственного интеграла, зависящего от параметра, признаки Дирихле и Абеля}
\subsection{Непрерывность, интегрируемость по конечному и бесконечному промежутку изменения параметра, дифференцируемость по параметру несобсенных интегралов, зависящих от параметра}
\subsection{Несобственные интегралы от неограниченных функций, зависящие от параметра}
...

\section{Интегралы Эйлера}
\subsection{$\Gamma$-функция, её дифференцируемость, свойства, формула приведения}
\subsection{$\Beta$-функция, её дифференцируемость, свойства, формула приведения}
\subsection{Связь $\Gamma$- и $\Beta$-функций}
\subsection{Формула Стирлинга}
...

\section{Примеры вычислительных приложений интегралов, зависящих от параметра}
\subsection{Вычисление интегралов $\int_0^\infty \frac{\sin x}{x} dx$ и $\int_0^\infty \frac{\sin nt}{t} dt$}
\subsection{Вычисление интеграла Пуассона $\int_0^\infty e^{-x^2} dx$}
...
