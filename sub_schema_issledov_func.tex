\newcounter{N} % для создания списков, маркированных со стилями, нужен счётчик
\begin{list}{\arabic{N}.}{\usecounter{N}}

\item Находят область определения функции.

\item Проверяют функцию на чётность, нечётность и периодичность.

\item Находят точки пересечения графика функции с осями координат, если такие точки есть.

\item Исследуют функцию на непрерывность, определяют точки разрыва и их род.

\item Исследуют поведение функции при стремлении независимой переменной $x$ к точкам разрыва и границам области определения функции, включая, если это необходимо, $\pm \infty$.

\item Находят асимптоты (вертикальные и наклонные) и точки пересечения графика функции с асимптотами.

\item Находят критические точки первого рода.

\item Находят экстремумы.

\item Определяют интервалы монотонности функции.

Предыдущие три пункта удобно осуществить с помощью первой производной, сведя результаты в таблицу, где в первой строке указываются значения аргумента $x$ - интервалы и точки, во второй -- знак производной $f'(x)$, в третьей наклонной стрелкой вверх-вправо \nearrow или вниз-вправо \searrow указывается характер монотонности функции.

\item

\end{list}
