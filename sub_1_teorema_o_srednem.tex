\begin{teorema}
Пусть $\{f,\varphi\}\subset R[a;b]$, $\varphi$ сохраняет знак на $[a;b]$, $m=\inf\limits_{[a;b]}f(x)$, $M=\sup\limits_{[a;b]}f(x)$.
Тогда 
$$
\exists(\mu\in[m;M])\left[
\intl_a^b f(x)\varphi(x) dx=\mu\intl_a^b \varphi(x) dx
\right]
$$
\end{teorema}

\dokvo
Не теряя общности, будем доказывать для случая, когда $\varphi(x)$ положительна на $[a;b]$.
(В противном случае -- просто вынести минус единицу за знак интеграла.)

Так как
$$
\forall(x\in[a;b])[m\leq f(x)\leq M]
$$

умножив на $\varphi(x)$, имеем

$$
m\varphi(x)\leq f(x)\varphi(x) \leq M\varphi(x)
$$

Интегрируем (помним свойства интеграла Римана!)

$$
\intl_a^b m\varphi(x) dx \leq \intl_a^b f(x)\varphi(x) dx \leq \intl_a^b M\varphi(x)
$$

Если $\intl_a^b\varphi(x) dx =0$, то $\varphi(x)\equiv 0$ на $[a;b]$, следовательно, $\intl_a^b\varphi(x)f(x) dx=0$ и $\mu$ можно брать любым.

В противном случае на $\intl_a^b\varphi(x)f(x)dx$ можно разделить:
$$
m\leq\frac{\intl_a^b\varphi(x)dx }{\intl_a^b\varphi(x)f(x)dx }\leq M
$$

Условию теоремы удовлетворяет
$$
\mu=\frac{\intl_a^b\varphi(x)dx}{\intl_a^b\varphi(x)f(x)dx}
$$

\dokno

Как следствие, если $f$ непрерывна на $[a;b]$, то, в силу теоремы о промежуточном значении, 
$$
\exists(\xi\in[a;b])[f(\xi)=\mu]
$$
то есть 

$$
\exists(\xi\in[a;b])\left[
\intl_a^b f(x)\varphi(x) dx=f(\xi)\intl_a^b \varphi(x) dx
\right]
$$

Если пойти дальше и положить $\varphi(x)\equiv 1$, получим

$$
\exists(\xi\in[a;b])\left[
\intl_a^b f(x) dx=f(\xi)(b-a)
\right]
$$

