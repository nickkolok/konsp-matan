\subsubsection{Теорема.}

Функция, непрерывная на отрезке, ограничена на нём.

\opred

Компактом (компактным множеством) называется такое множество $X$, что
$$
\forall(\{x_n\}:x_n \in X)\exists(\{x_{n_k}\})[\{x_{n_k}\}\to x_0 \in X],
$$ 
т. е. в любой последовательности точек этого множества можно выделить подпоследовательность, сходящуюся к точке этого множества.

\subsubsection{Замечание.}
Конечный или бесконечный интервал $(a;b)$, где $\{a;b\}\subset\overline\R$, не является компактом, т. к. любая подпоследовательность любой последовательности его точек, сходящейся к $a$ или $b$, сходится к не принадлежащей интервалу точке $a$ или $b$ соответственно.

Полуинтервал также не является компактом.
Предоставляем читателю доказать это самостоятельно.

\subsubsection{Обобщение первой теоремы Вейерштрасса.}

Функция, непрерывная на компакте, ограничена на нём.

\subsubsection{Замечание.}

Функция, определённая на некомпактном множестве, может быть на нём неограничена. Пример - тождественная функция $f(x)=x$ на некомпактом множестве $(-\infty;+\infty)$.

