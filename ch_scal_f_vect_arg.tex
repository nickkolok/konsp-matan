\section{Скалярные функции векторного аргумента}
\subsection{Пространство \Rn}
\begin{opr}
Пространство \Rn -- множество упорядоченных наборов из $n$ вещественных чисел:
$$
x\in \R^n \Leftrightarrow x=(x^1, ... , x^n),~~x^i\in\R,~~i=1,...,n
$$
\end{opr}
\begin{zamech}
\Rn -- линейное пространство.
Оно более детально изучается в курсе линейной алгебры.
\end{zamech}

\begin{zamech}
Индекс (номер) координаты вектора пишется вверху, т. к. нижний индекс необходим в выкладках, содержащих последовательности.
Как правило, такие обозначения не приводят к недоразумению и путанице с обонзанчением степени.
\end{zamech}

Выпишем определения операций в \Rn -- сложения и внешнего умножения:

$$
\forall(x=(x^1,...,x^n)\in\R^n,y=(y^1,...,y^n)\in\R^n)[x+y=(x^1+y^1,...,x^n+y^n)]
$$

$$
\forall(\lambda\in\R)\forall(x=(x^1,...,x^n)\in\R^n)[\lambda x=(\lambda x^1,...,\lambda x^n)]
$$

Нулевой вектор, как и скалярный нуль, и нулевой оператор, и т. д., будем обозначать символом $0$. Опять же, в большинстве случаев к недоразумению такое обозначение не приводит.

Все выкладки будем давать в стандартном базисе $e$:
$$\begin{aligned}
e_1=(1,0,...,0)\\
e_n=(0,...,0,1)
\end{aligned}$$

Напомним также тот факт, что любой вектор разложим по базису:
$$
\forall(x\in\R^n)\exists(\alpha_1,...,\alpha_n\in\R)[x=\alpha_1 e_1+...+\alpha_n e_n]
$$

Примеры пространств:

$\R^1=\R$

$\R^2$ - точки плоскости.

$\R^3$ - точки пространства.


\subsection{Нормированное пространство \Rn}
\begin{opr}
\Rn -- нормировано, если каждому вектору $x\in\R^n$ сопоставлено вещественное число $\|x\|$, так, что:

\begin{equation}\label{aks1normy}
\|x\|=0 \Leftrightarrow x=0
\end{equation}

\begin{equation}\label{aks2normy}
\forall(\lambda\in\R)[\|\lambda x\|=|\lambda|\|x\|]
\end{equation}

\begin{equation}\label{aks3normy}
\forall(y\in\R^n)[\|x+y\|\leq\|x\|+\|y\|]
\end{equation}

\end{opr}

Эти три формулы называют аксиомами нормы.
Заметим, что неотрицательность нормы нет необходимости вводить как аксиому:
$$
2\|x\|=\|x\|+|-1|\|x\|=\|x\|+\|-x\|\geq\|x+(-x)\|=\|0\|=0
$$

Норма, вообще говоря, является скалярной функцией векторного аргумента, но определение такой функции будет дано далее.

\begin{opr}
Евклидовой нормой называют норму, введённую равенством
\begin{equation}
|x|=\sqrt{\sum_{i=1}^n (x^i)^2}
\end{equation}
\end{opr}

Евклидову норму обозначают не двойными вертикальными чертами, а одинарными.
Пространство \Rn, в котором введена евклидова норма, называт евклидовым.
В $\R^1$ евклидова норма -- не что иное, как модуль.

Аксиома (\ref{aks3normy}) приводит к неравенству Буняковского-Шварца:
\begin{equation}\label{nervoBunSchvarz}
\sum_{i=1}^n|a^i b^i|\leq\sqrt{\sum(a^i)^2}+\sqrt{\sum(b^i)^2}
\end{equation}

Заметим, однако, что это неравенство возможно доказать и без применения методов математического анализа или линейной алгебры.

Другим следствием аксиомы (\ref{aks3normy}) является неравенство Коши - Миньковского:
\begin{equation}\label{nervoKoshiMink}
\sqrt{\sum_{i=1}^n(x^i+y^i)^2}\leq\sqrt{\sum_{i=1}^n(x^i)^2}+\sqrt{\sum_{i=1}^n(y^i)^2}
\end{equation}

Заметим, что можно вводить и неевклидовы нормы, например:
\begin{equation}
\|x^1,...,x^n\|=\max\limits_{1...n} x^i
\end{equation}

\begin{equation}
\|x^1,...,x^n\|=\sum_{i=1}^n x^i
\end{equation}

\begin{opr}
Две нормы $\|x\|_1$ и $\|x\|_2$ в \Rn называются эквивалентными, если
\begin{equation}\label{opred_eqiv_norm}
\exists(c_1>0, c_2>0)\forall(x\in\R^n)[c_1\|x\|_1\leq\|x\|_2\leq c_2\|x\|_1]
\end{equation}
\end{opr}

Примем пока без доказательств утверждение, что в конечномерных \Rn любые две нормы эквивалентны.
Позже оно будет доказано.

\begin{opr}
Множество $G\subset\R^n$ -- ограниченно, если
\begin{equation}\label{opr_pgran_mn}
\exists(c>0)\forall(x\in G)[\|x\|\leq c]
\end{equation}
\end{opr}

\begin{opr}
Функция $\rho(x,y)$, где $x\in\R^n$, $y\in\R^n$, называется метрикой, если выполнены следующие аксиомы (аксиомы метрики):
\begin{equation}\label{aks1metr}
\rho(x,y)=0 \Leftrightarrow x=y
\end{equation}

\begin{equation}\label{aks2metr}
\rho(x,y)=\rho(y,x)
\end{equation}

\begin{equation}\label{aks3metr}
\rho(x,y)\leq\rho(x,z)+\rho(z,y)
\end{equation}

\end{opr}

Заметим, что неотрицательность метрики следует из третьей аксиомы метрики при $y=x$.

\begin{opr}

Евклидово пространство, в котором введена метрика 
\begin{equation}\label{evkl_metrika}
\rho(x,y)=\|x-y\|
\end{equation}
называют метрическим.

\end{opr}

Заметим вскользь, что метрику можно ввести и без использования понятия нормы.

\subsection{Последовательность в \Rn.Сходимость последовательностей. Эквивалентность покоординатной сходимости}
\begin{opr}
Последовательностью в \Rn называется отображение $f:\N\to\R^n$.
\end{opr}
Это означает, что $\forall(k\in\N)\exists(x_k\in\R^n)[f(k)=x_k]$.

\begin{primer}
\begin{multline*}
\left\{x_k=\left( \frac{1}{k};k^2+1;2^k; \frac{k}{3k+1}\right)\right\}
\\
x_1=\left(1;2;2;\frac{1}{4}\right)
\\
x_2=\left(\frac{1}{2};5;4;\frac{2}{7}\right)
\end{multline*}
и т. д.
\end{primer}

\begin{opr}
Пусть $\{x_k\}\subset\R^n$ - последовательность.
Если $$\exists(x_0\in\R^n)[\{\|x_k-x_0\|\}\to0]$$ (здесь $\{\|x_k-x_0\|\}$ -- числовая последовательность), то говорят, что $\{x_k\}$ сходится к $x_0$ и пишут:
$$
\{x_k\}\to x_0
$$
или
$$
\lim x_k =  x_0
$$
или
$$
\lim_{k\to\infty} x_k =  x_0
$$
\end{opr}
Иначе говоря,
\begin{equation*}
\{x_k\}\to x_0 \Leftrightarrow \forall(\varepsilon>0)\exists(k_0\in\N)\forall(k>k_0)[\|x_k-x_0\|<\varepsilon]
\end{equation*}

Легко доказать, что если две нормы эквивалентны, то сходимость по первой из этих норм равносильна сходимости по второй.

\begin{teorema}
Сходимость по норме эквивалентна покоординатной сходимости, т. е.
$$
\{x_k\}\to x_0 \Leftrightarrow \forall(i\in\Z\cap[1;n])[x_k^i\to x_0^i]
$$
\end{teorema}
\dokvo
Так как все нормы эквивалентны, то докажем утверждение только для евклидовой нормы (\ref{evklidova_norma}):
$$|x_k-x_0|=\sqrt{\sum_{i=1}^{n}(x_k^i-x_0^i)^2} \to 0
\Rightarrow \forall(i\in\Z\cap[1;n])[x_k^i- x_0^i\to0]
$$
\dokno

\begin{sledstvie}
\begin{multline*}
\forall(\{x_k\}\to x_0:\{x_k\}\subset \R^n,\{y_k\}\to y_0:\{y_k\}\subset \R^n,\{\lambda_k\}\to \lambda_0:\{\lambda_k\}\subset \R)
\\
[\{x_k+y_k\}\to x_0+y_0 ~\cap~ \{\lambda_k x_k\}\to \lambda_0 x_0]
\end{multline*}
\end{sledstvie}

\begin{sledstvie}
Некоторое множество $G\subset\R^n$ ограничено тогда и только тогда, когда ограничено множество, состоящее из вещественных чисел, являющихся координатами элементов $G$.
\end{sledstvie}

\begin{teorema}[Больцано-Вейерштрасса для \Rn]
Из любой ограниченной последовательности можно выделить сходящуюся подпоследовательность.
\end{teorema}
\dokvo
Пусть $\{x_k\}\subset\R^n$ --- последовательность.
Выделим из неё сначала подпоследовательность $\{x_{k_1}\}$ так, что последовательность первых координат $\{x_{k_1}^1\}$ сходится;
(это возможно по теореме Больцано-Вейерштрасса для $\R$, так как множество значений первых координат ограничено)
затем выделим из $\{x_{k_1}\}$ подпоследовательность $\{x_{k_2}\}$, такую, что последовательность вторых координат $\{x_{k_1}^2\}$ сходится.
Продолжая действовать подобным образом, получим требуемую последовательность $\{x_{k_n}\}$, сходящуюся покоординатно.
\dokno

\subsection{Замкнутые, открытые, компактные множества в \Rn}
\subsection{Функции многих переменных. Предел. Непрерывность }
...

\section{Дифференцирование скалярных функций векторного аргумента}
\subsection{Линейные функционалы в \Rn}
\subsection{Определение дифференциала скалярной функции векторного аргумента. Связь между понятиями дифференцируемости и непрерывности}
\subsection{Простейшие свойства операции дифференцирования}
\subsection{Определение производной по направлению. Связь между понятиями дифференцируемости функции по Фреше и Гато}
\subsection{Теорема Лагранжа}
\subsection{Частные производные скалярных функций векторного аргумента. Связь между существованием частных производных и дифференцируемостью функции по Фреше и Гато}
\subsection{Теорема о дифференцируемости сложной функции и следствие из неё}
\subsection{Инвариантность формы первого дифференциала}
\subsection{Частные производные высших порядков}
\subsection{Дифференциалы высших порядков}
\subsection{Формула Тейлора для скалярной функции векторного аргумента}
\begin{teorema}
Пусть $f:E\to\R^n$, $E\subset\R^n$, $E$ - открыто, $f$ дифференцируема на $E$ до $(k+1)$-го порядка включительно.
Тогда
\begin{multline}
\forall(x\in E)
\end{multline}
\end{teorema}
\dokvo
\dokno

...

\section{Локальные экстремумы скалярных функций векторного аргумента}
\subsection{Необходимое условие локального экстремума}
\subsection{Достаточные условия локального экстремума}
\begin{teorema}
Пусть $A\subset\R^n$, $f:A \to \R$ и $f$ дважды непрерывно дифференцируема в $x_0\in A$, $x_0$ - стационарная точка $f$,
второй дифференциал $f$ в точке $x_0$, т. е. $d^2 f(x_0,h)$ является невырожденной квадратичной формой.
Тогда $d^2 f(x_0,h)$ определяет наличие в точке $x_0$ локального экстремума, причём если $d^2 f$ --- положительно определённая квадратичная форма (напомним, от $h$), то функция $f$ имеет в точке $x_0$ локальный минимум, если $d^2 f$ --- отрицательно определённая квадратичная форма, то функция $f$ имеет в точке $x_0$ локальный максимум, если же $d^2 f$ --- неопределённая квадратичная форма, то локального экстремума в точке $x_0$ у функции $f$ нет.
\end{teorema}
\dokvo


\dokno

...

\section{Теорема о неявной функции (теорема Юнга)}
\subsection{Лемма о неявной функции}
\subsection{Теорема Юнга}
\subsection{Следствие о непрерывной дифференцируемости $k$-го порядка}
\subsection{Теорема о неявной функции для скалярной функции векторного аргумента}
...

