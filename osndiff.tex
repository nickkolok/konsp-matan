\section{Дифференциальное исчисление функции одной независимой переменной}
\subsection{Определение производной и дифференциала, связь между этими понятиями}
\subsection{Связь между понятиями дифференцируемости и непрерывности функций}
\subsection{Дифференцирование и арифметические операции}
\subsection{Теорема о производной сложной функции. Инвариантность формы первого дифференциала}
\subsection{Теорема о производной обратной функции}
\subsection{Производные основных элементарных функций. Доказательство}
\subsection{Касательная к кривой. Геометрический смысл производной и дифференциала}
\subsection{Физический смысл производной и дифференциала}
\subsection{Односторонние и бесконечные производные}
\subsection{Производные и дифференциалы высших порядков}

\section{Основные теоремы дифференциального исчисления}
\subsection{Теорема Ферма}
\subsection{Теорема Ролля}
\subsection{Теорема Лагранжа и следствия из нее}
\subsection{Теорема Коши}

\section{Формула Тейлора}
\subsection{Формула Тейлора для многочлена}
\subsection{Формула Тейлора для произвольной функции. Различные формы остаточного члена формулы Тейлора}
\subsection{Локальная формула Тейлора}
\subsection{Формула Маклорена. Разложение по формуле Маклорена некоторых элементарных функций}

\section{Правило Лопиталя}
