\section{Ряды Фурье}
\subsection{Линейные пространства со скалярным произведением. Ортогональные и ортонормированные системы}
\subsection{Теорема о существовании и единственности проекции на $X_n$. Неравенство Бесселя}
\subsection{Ряд Фурье по произвольной ортонормированной последовательности элементов. Минимальное свойство частичных сумм ряда Фурье}
\subsection{Критерий замкнутости ортонормированной системы}
\subsection{Полные ортонормированные системы. Связь между замкнутостью и полнотой}
\subsection{Тригонометрическая система функций, её ортонормированность. Тригонометрический ряд Фурье. Минимальное свойство. Неравенство Бесселя}
\subsection{Лемма Римана}
\subsection{Интегральное представление для частичных сумм ряда Фурье. Интеграл Дирихле}
\subsection{Принцип локализации Римана}
\subsection{Теорема о поточечной сходимости тригонометрического ряда Фурье}
\subsection{Теорема о равномерной сходимости тригонометрического ряда Фурье}
\subsection{Почленное дифференцирование и интегрирование рядов Фурье}
\subsection{Ряд Фурье для функции, определённой и интегрируемой на $[-l;l]$}
\subsection{Ряды Фурье для чётных и нечётных функций}
\subsection{Разложение в ряд Фурье функции, заданной на $[0;l]$}
\subsection{Комплексная форма ряда Фурье}
\subsection{Понятие о ряде Фурье для функции нескольких переменных}
\subsection{Понятие об интеграле Фурье}
\subsection{Понятие о преобразовании Фурье и его применении}
\subsection{Замкнутость в $L_2^1 [-\pi;\pi]$ системы тригонометрических функций (теорема Дирихле-Ляпунова) }
...

\section{Равномерная аппроксимация функций}
\subsection{Теорема Вейерштрасса об аппроксимации непрерывных функций с помощью тригонометрических многочленов}
\subsection{Теорема Вейерштрасса об аппроксимации непрерывных функций с помощью алгебраических многочленов}
...

