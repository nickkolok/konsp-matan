Продолжая перенос результатов, полученных при изучении векторной функции скалярного аргумента, на случай векторной функции векторного аргумента, рассмотрим теорему Лагранжа.

\begin{teorema}
Пусть $G\subset\R^n$, $G$ открыто, $f:G\to\R$, $f$ - дифференцируема на $G$.
Тогда
\begin{equation}
\forall([x;;y]\subset G)\exists(\xi\in(x;;y))[f(x)-f(y)=f'(\xi)(x-y)]
\end{equation}
\end{teorema}

Покажем, что в форме равенства теорему Лагранжа перенести на случай $f:\R^n\to\R^m$ нельзя.
Рассмотрим $f(x)=(\sin x; \cos x)$. 
Тогда $f'(x)=(\cos x; -\sin x)$.
Выпишем опровергаемое утверждение для отрезка $[0;\frac{\pi}{2}]$:
\begin{equation}\label{teorema_Lagranzha_vect_f_vect_arg_kontr}
(\sin 0; \cos 0)-(\sin \frac{\pi}{2}; \cos \frac{\pi}{2}) = (\cos \xi; -\sin \xi)\cdot\frac{\pi}{2}
\end{equation}
Распишем покоординатно:
$$
\sin 0-\sin \frac{\pi}{2} = \cos \xi \cdot\frac{\pi}{2}
$$
$$
\cos 0-\cos \frac{\pi}{2} = -\sin \xi\cdot\frac{\pi}{2}
$$
То есть:
$$
0-1 = \cos \xi \cdot\frac{\pi}{2}
$$
$$
1-0 = -\sin \xi\cdot\frac{\pi}{2}
$$
Отсюда имеем $\cos \xi = \sin \xi = - \frac{2}{\pi}$, то есть $\cos^2 \xi + \sin^2 \xi \neq 1$, что невозможно.
Следовательно, ни при каком $\xi$ равенство \ref{teorema_Lagranzha_vect_f_vect_arg_kontr} выполнено быть не может.

