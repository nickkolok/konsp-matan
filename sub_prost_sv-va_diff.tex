\begin{teorema}
Если $f,g:E\to\R^1$ - дифференцируемы в $x\in E$, то:
\\
a) $\lambda f+\mu g$ (где $\lambda,\mu\in\R^1)$ - дифференцируема в х, и $[(\lambda f+ \mu g)'(x)=(\lambda f'+\mu g')(x)]$
\\
б) $f\cdot g$ - дифференцируема в x, и 
$[(f\cdot g)'(x)=(f'\cdot g)(x)+(f\cdot g')(x)]$
\\
в) $f/g$ (если $g(x)\ne 0$) - дифференцируема в х, и
\\
$[(f/g)'(x)=(\frac{f'\cdot g - g'\cdot f}{g^2})(x)]$
\end{teorema}

\dokvo
а) Возьмем $\forall$ приращение: $\forall(h\in\R^n:x+h\in E):$
\\
$$(\lambda f+\mu g)(x+h)-(\lambda f+\mu g)(x)=$$
$$=(\lambda\cdot f(x+h)-\lambda\cdot f(x))+(\mu\cdot g(x+h)-\mu\cdot g(x))=$$
$$=\lambda(f(x+h)-f(x))+\mu(g(x+h)-g(x))=$$
$$=\lambda(f'(x,h)+\omega_1(x,h))+\mu(g'(x,h)+\omega_2(x,h))=$$
$$=\lambda f'(x,h)+\mu g'(x,h)+(\lambda\omega_1(x,h)+\mu\omega_2(x,h))=$$
$$=(\lambda f'(x)+\mu g'(x))(h)+\omega(x,h).$$
\\
$\lim_{h\to 0}\omega(x,h)=\lim_{h\to 0}(\lambda\omega_1(x,h)+\mu\omega_2(x,h))=0\Rightarrow$
\\
$\Rightarrow \omega(x,h)=o(||h||)$
\\
Значит, $\lambda f+ \mu g$ - дифференцируема в х, и $(\lambda f+ \mu g)'=\lambda f'+ \mu g'$ 
\\
\dokno
\\
б) аналогично пункту а)
\\
в) достаточно будет доказать, что $\frac{1}{g}$ - дифференцируема в х, $(\frac{1}{g})'(x) = -\frac{g'(x)}{g^2(x)}$, и воспользовать пунктом б) данной теоремы:
\\
$(\frac{f}{g})'(x) = (f\cdot\frac{1}{g})'(x) = (f'\cdot\frac{1}{g})(x)+(f\cdot\frac{-g'}{g^2})(x) = (\frac{f'g-g'f}{g^2})(x)$
\\
докажем вышесказанное:
$\forall(h\in\R^n:x+h\in\R^2)$
\\
$$
[\frac{1}{g(x+h)}-\frac{1}{g(x)}=-\frac{g(x+h)-g(x)}{g(x+h)\cdot g(x)} = -\frac{g'(x)+\omega(x,h)}{g(x+h)\cdot g(x)}]
$$
где $\omega(x,h)=o(||h||)$, при $h\to 0$.
\\
Функция g - дифференцируема $\Rightarrow$ непрерывна в х.
\\
Поэтому $\frac{1}{g(x+h)\cdot g(x)}\to\frac{1}{g^2(x)}$ при $h\to 0$.
\\
$\frac{1}{g(x+h)\cdot g(x)}\to\frac{1}{g^2(x)}+\alpha(x,h)$, где $\alpha(x,h)\to 0, h\to 0$
\\
Отсюда:
\\
$\frac{1}{g(x+h)}-\frac{1}{g(x)}=-((g'(x)+\omega(x,h))\cdot\frac{1}{g(x+h)\cdot g(x)})=$
\\
$=-(g'(x)h+\omega(x,h))\cdot(\frac{1}{g^2(x)}+\alpha(x,h))=-\frac{g'(x)h}{g^2(x)}+\omega_1(x,h)$
\\
где $\lim_{h\to 0}\omega_1(x,h)=\lim_{h\to 0}(g'(x)h\cdot\alpha(x,h)+\omega(x,h)(\frac{1}{g^2(x)}+\alpha(x,h)))=0$
\\
$\lim_{h\to 0}\frac{\omega_1(x,h)}{||h||}=\lim_{h\to 0}(g'(x)\cdot\frac{h}{||h||}\cdot\alpha(x,h)+\frac{\omega(x,h)}{||h||}(\frac{1}{g^2(x)}+\alpha(x,h)))=0$
\\
Значит, $\omega_1(x,h)=o(||h||)$, при $h\to 0\Rightarrow$
\\
$\Rightarrow\frac{1}{g(x)}$ - дифференцируема в х, и $(\frac{1}{g})'(x)=-\frac{g'(x)}{g^2(x)}$
\\
\dokno








