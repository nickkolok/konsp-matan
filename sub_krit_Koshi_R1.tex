\opred
Числовая последовательность $\{x_k\}$ называется фундаментальной,
если
\begin{equation}\label{opr_fund_posled}
\forall (\varepsilon >0) \exists (n_0\in \N ) \forall (n,m\geq n_0) [|x_n-x_m| < \varepsilon ]
\end{equation}

\begin{teorema}[Критерий Коши.]
Числовая последовательность $\{x_n\}$ имеет предел тогда и только тогда, когда она фундаментальна.
\end{teorema}
\dokvo

\neobh
Пусть $$\lim_{n\to\infty} x_n = a.$$
Согласно определению предела,
$$
\forall(\varepsilon > 0) \exists(n_0 \in \N) \forall(n \geq n_0)\left[ |x_n -a| < \frac{\varepsilon}{2}\right].
$$
Аналогично
$$
\forall(\varepsilon > 0) \exists(n_0 \in \N) \forall(m \geq n_0)\left[ |x_m -a| < \frac{\varepsilon}{2}\right].
$$
Известно, что
$$
|x_n - x_m| \leq |x_n - a| + |x_m - a|.
$$
Итого имеем
$$
\forall(\varepsilon > 0) \exists(n_0 \in \N) \forall(m, n \geq n_0)\left[|x_n - x_m| \leq  |x_n -a| + |x_m -a| < \varepsilon\right].
$$

\dost

Докажем сначала, что $\{x_n\}$ ограничена.
Зафиксируем $\varepsilon_0 > 0$.
В силу фундаментальности $\{x_n\}$
$$
	exists (n_0\in \N ) \forall (n,m \geq n_0) [|x_n-x_m| < \varepsilon_0 ],
$$
в частности,
$$
	exists (n_0\in \N ) \forall (n \geq n_0) [|x_n-x_{n_0}| < \varepsilon_0 ].
$$
Оценим:
$$
	|x_n| = |x_n - x_{n_0} + x_{n_0} | \leq |x_n - x_{n_0}| + |x_{n_0} | < \varepsilon_0 + |x_{n_0}|.
$$

Таким образом, последовательность $\{x_n\}$, начиная с некотрого члена, ограничена, а значит, и ограничена вообще.

Ещё раз бросим взгляд на проделанное: мы зафиксировали $\varepsilon_0$, нашли по нему $n_0$ и,
положив в определении фундаментальной последовательности $m = n_0$,
доказали ограниченность фундаментальной последовательности.

Теперь воспользуемся следствием из теоремы Больцано-Вейерштрасса \ref{teorema_Bolcano-Veyershtrassa},
гласящим, что из любой ограниченной последовательности можно выделить сходящуюся подпоследовательность:
$$
	\exists(x_{n_k})\left[  \lim_{k\to\infty} x_{n_k} = a \ne\pm\infty \right].
$$
Покажем, что и $x_n \to a$:
$$
	|x_n - a| \leq |x_n - x_{n_k}| + |x_{n_k}-a|.
$$
Первое слагаемое стремится к нулю (при достаточно больших $n$ и $n_k$) в силу фундаментальности,
второе --- в силу сходимости подпоследовательности.
%TODO: поподробнее
\dokno

