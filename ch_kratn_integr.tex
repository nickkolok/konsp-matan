\section{Двойные интегралы}
\subsection{Определение двойного интеграла и условия его существования}
\subsection{Необходимое условие интегрируемости}
\subsection{Критерии Дарбу и Римана}
\subsection{Классы интегрируемых функций}
\subsection{Свойства двойных интегралов}
\subsection{Интегрирование по прямоугольнику и специальной области}
\subsection{Замена переменной в двойном интеграле}
\subsection{Переход к полярным координатам}
\subsection{Вычисление площади поверхности с помощью двойного интеграла}
\subsection{Вычисление объёма тела с помощью двойного интеграла}
\subsection{Вычисление объёма тела, ограниченного явно заданной поверхностью, с помощью двойного интеграла}
...

\section{$n$-кратные интегралы}
\subsection{Определение тройного интеграла}
\subsection{Объём $n$-мерного параллелепипеда}
\subsection{Общее определение $n$-кратного интеграла}
\subsection{Интегрирование по специальной области}
\subsection{Замена переменной: связь между элементом площади в пространстве $(x,y)$ и элементом площади в пространстве $(u,v)$}
\section{Замена переменной в двойном интеграле}
\section{Переход к цилиндрическим координатам}
\section{Переход к сферическим координатам}
...

