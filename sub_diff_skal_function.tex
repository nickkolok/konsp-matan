\begin{opred}
Функция $f:E\to\R_1$ дифференцируема в $x\in E$ (дифференцируема по Фреше), если $\exists$(линейный функционал 
\\$e(x):\R^n\to\R^1)\forall(h\in\R^n:x+h\in E)[f(x+h)-f(x)=l(x)(h)+\omega(x,h)]$, где $\omega(x,h)=0 (||h||)$ при $h\to 0$, т.е. $\frac{\omega (x,h)}{||h||}\to 0$ при $h\to 0$.
\end{opred}

\begin{opred}
	Производная функции f в т. x (f'(x)) - это линейный функционал $e(x):\R^n \to \R^1$ 
\end{opred}

\begin{opred}
	Дифференциал функции f в т. x - значение линейного функционала на элементе h. df(x,h). Отсюда:
	$df(x,h)=f'(x)h$
\end{opred}

$\vartriangle x(h)=(x+h)-x=h; \vartriangle f(x,h) = f(x+h)-f(x)$
\\
Подставим эти обозначения в определение дифференцируемости по Фреше:
\\
$\vartriangle f(x,h) = df(x,h) + \omega (x,h)$, где $\omega (x,h) = o(||h||), h\to 0$
\\
Пусть $E \subset \R^n$ - открытое подмножество $\R^n$. Тогда: 
\begin{teorema}
Единственность производной по Фреше.
\end{teorema}

\dokvo
Пусть $f:E\to\R^1$ дифференцируема в $x\in E$ и  имеет 2 производные. Тогда по определению производной по Фреше:
$$
\exists (f_1'(x),f_2'(x))[f(x+h)-f(x)=f_1'(x)h+\omega_1 (x,h)] (1)
$$

$$
[f(x+h)-f(x)=f'(x)h+\omega_2 (x,h)] (2)
$$
\\
где $\omega\omega_1(x,h)=o(||h||)$ и $\omega_2(x,h)=o(||h||), h\to 0$
\\
$(1)-(2):o=f_1'(x)h+\omega_1(x,h)-f_2'(x)h-\omega_2 (x,h).$
\\
$(f_1'(x)-f_2'(x))h=\omega_2(x,h)-\omega_1(x,h)$
\\
$|\frac{f_1'(x)-f_2'(x)}{||h||}\cdot h|=|\frac{\omega_2(x,h)-\omega_1(x,h)}{||h||}\le|\frac{\omega_2(x,h)}{||h||}|+|\frac{\omega_1(x,h)}{||h||}|$
\\
$\frac{\omega_2(x,h)}{||h||} \to 0$ и $\frac{\omega_1(x,h)}{||h||} \to 0$, при $h\to 0$
\\
Возьмем $\forall (h_0\in\R^n, h=t\cdot h_0, t\in\R^1, h_0$ - фиксирована) 
\\
Тогда $\frac{|(f_1'(x)-f_2'(x))t h_0|}{|t|||h_0||}\to 0$ при $t\to 0$
\\
$\frac{(f_1'(x)-f_2'(x))h_0|}{||h_0||}\to 0$ при $t\to 0$, но т.к. это выражение от t не зависит, то это выражение = const.
\\
если $h_0 = 0$, то $f_1'(x(o))=f_2'(x(o))=0$
\\
если $h_0 \neq 0$, то 
\\
$f_1'(x)-f_2'(x)=0\Rightarrow f_1'(x)=f_2'(x)$
\\
\dokno

\begin{teorema}
Если $f:E\to\R^2$ - дифференцируема по Фреше в $x\in E$, то f - непрерывна в x.
\end{teorema}
\dokvo
$f(x+h)-f(x)=f'(x)h+\omega(x,h)$, где $\omega(x,h) = o(||h||), h\to 0$
\\
Пусть $h\to 0$. Тогда, в силу линейности:
\\
$f'(x)h+\omega(x,h)\to 0\Rightarrow f(x+h)\to f(x) \Rightarrow$ функция непрерывна в т. х.
\\
\dokno

\subsubsection{Следствие}
Если f(x) дифференцируема во всех точках Е, то она непрерывна на Е.

\begin{opred}
f(x) дифференцируема на множестве E, если $f:E\to\R^1$ - дифференцируема в любой точке E.
\end{opred}
\\
Примеры:
\\
$1.f:E\to\R^1$, где $E \subset\R^2, f(x^1,x^2)=(x^1)^2-(x^2)$
\\
$E \subset \R^2 \Rightarrow h = (h^1,h^2)$
\\
$f(x+h)-f(x)=(x^1+h^1)^2 - (x_2 + h_2) - (x^1)^2 + (x^2) = 2xh^1 - h_1^2 + (h^1)^2$
\\
$\frac{(h^1)^2}{||h||}=\frac{(h^1)^2}{\sqrt{(h^1)^2+(h^2)^2}}\le\frac{(h^1)^2}{|h^1|}\to 0$, при $h\to 0$, т.е.
\\
$(h^1)^2=o(||h||)=\omega(x,h)\Rightarrow$
\\
$\Rightarrow$ выполняется определение дифференцируемости по Фреше $\Rightarrow f(x_1,x_2)$ - дифференцируема на $\R^2$ и $d(x,h) = 2x^1h^1-h^2$
\\
2.Пусть l-линейный функционал, определенный на пр-ве $\R^n$. Тогда 
\\
$\forall(x,h\in\R^n)[l(x+h)-l(x)=l(h)=lh]$
\\
значит определение дифференцируемост по Фреше выполняется.
\\
$(\omega(x,h) = 0)\Rightarrow$ линейный функционал дифференцирован на $\R^n$ и $l'(x)h=lh\Rightarrow l'(x)=l$.
\\
Отсюда: производная функционала в любой точке х есть тот же самый функционал.
\\
3.Рассмотрим $f:\R^n\to\R^1, f(x)=||x||$.
\\
Докажем, что в точке х=0 f(x) - не дифференцируема.
\\
\dokvo
$f(o+h)-f(o)=lh+\omega(h)$, где $\omega(h)=o(||h||), h\to 0$,
\\
но $f(o+h)-f(o)=||o+h||-||o||=||h||$
\\
Пусть $\forall(h_0\in\R^n,h_0 \ne 0)\exists(t\in\R^1)[h=th_0].$
\\
Тогда $|t|\cdot ||h_0||=l(th_0)+\omega(th_0)=t\cdot l(h_0)+\omega(th_0)$
\\
$||h_0||=\frac{t}{|t|}l(h_0)+\frac{\omega(th_0)}{|t|}$
\\
$||h_0||-\frac{\omega(th_0)}{|t|}=\frac{t}{|t|}l(h_0)$
\\
$\lim_{t\to 0} (||h_0||-\frac{\omega (th_0)||h_0||}{|t|||h_0||})=\lim_{t\to 0}||h_0||=||h_0||$
\\
т.е. $\exists(\lim_{t\to 0} (||h_0||-\frac{\omega (th_0)||h_0||}{|t|||h_0||})=||h_0||)$, но $\urcorner\exists(\lim_{t\to 0} \frac{t}{|t|}l(h_0))$ Противоречие.
\\
\dokno









