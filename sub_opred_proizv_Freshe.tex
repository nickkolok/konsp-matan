\begin{opr}\label{oped_proizv_Freshe_Rn_Rm}
Пусть $A\subset\R^n$, $x_0$ --- внутренняя точка $A$, $f:A\to\R^m$.
Отображение $f$ называется дифференцируемым по Фреше в точке $A$, если
\begin{multline}
\exists(L_{x_0}\in L(\R^n,\R^m))\forall(h\in \R^n: x_0+h\in A)
\\
[f(x_0+h)-f(x_0)=L_{x_0}h+\omega(x_0,h), \omega(x_0,h)=o(\|h\|)]
\end{multline}
\end{opr}

\begin{opr}
Введённое выше линейное отображение $L_{x_0}$ называется производной по Фреше функции $f$ в точке $x_0$ и обозначается $f'(x_0)$ или $Df(x_0)$. Иногда его также называют производным отображением или касательным отображением.
Значение $f'(x_0)$ на элементе $h$ называется дифференциалом функции $f$, соответствующим приращению $h$: $df(x_0,h)=f'(x_0)h$.
\end{opr}

\begin{opr}
Функцию, дифференцируемую в каждой точке некоторого множества, называют дифференцируемой на этом множестве.
\end{opr}

\begin{zamech}
В отличие от случая скалярной функции скалярного аргумента, функция и производная представляют собой объекты разной природы:
$$
f(x)\in\R^m
$$
$$
f'(x)\in L(\R^n,\R^m)
$$
Иногда понятия дифференциала и производной не различают.
\end{zamech}

\begin{opr}
Введённая в определении \ref{oped_proizv_Freshe_Rn_Rm} функцию $\omega(x_0,h)$ называется остатком приращения.
\end{opr}

\begin{primer}
Если $f:A\to\R^m$, $f=const$, т. е. постоянно, то $\forall(x\in A)[f'(x)=0]$.
\end{primer}

\begin{primer}
Если $f:A\to\R^m$, $f\in L(\R^n,\R^m)$, т. е. линейно, то $\forall(x\in A)[f'(x)=f]$.
\end{primer}

Покажем теперь, как вычислять производную Фреше по определению, заодно убедив читателя, что это не вполне удобно.
\begin{primer}
$f:\R^2 \to \R^2$,ж
$$f(x^1,x^2)=\left(\frac{1}{2}\left((x^1)^2-(x^2)^2\right); x^1 x^2\right)$$

Зафиксируем $h=(h^1, h^2)$.
Выписываем приращение функции:
\begin{multline}
f(x+h)-f(x)
=\\=
\left(\frac{1}{2}\left((x^1+h^1)^2-(x^2+h^2)^2\right); (x^1+h^1)( x^2+h^2) \right)-\left(\frac{1}{2}\left((x^1)^2-(x^2)^2\right); x^1 x^2\right)
=\\=
(x^1h^1-x^2h^2; x^1h^2-x^2h^1)+\left( \frac{1}{2}\left( (h^1)^2+(h^2)^2 \right); h^1 h^2 \right)
\end{multline}
Видно, что первое слагаемое линейно по $h$, второе --- нет.
Значит,
$$
f'(x)(h^1,h^2)=(x^1h^1-x^2h^2; x^1h^2-x^2h^1)=
\begin{pmatrix}
x^1 & -x^2\\
x^2 & x^1
\end{pmatrix}
\begin{pmatrix}
h^1\\
h^2
\end{pmatrix}
$$
Таким образом, производная имеет вид
$$
f'(x)=
\begin{pmatrix}
x^1 & -x^2\\
x^2 & x^1
\end{pmatrix}
$$
\end{primer}

