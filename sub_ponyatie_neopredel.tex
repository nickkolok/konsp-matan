Пусть даны две непрерывные на интервале $(a; b)$ функции $f(x)$ и $g(x)$, где $\{a; b\} \subset \overline{\mathbb{R}}$. Неопределённостью типа $\left[\frac{0}{0}\right]$ в точке $a$ называется предел 
\[
\lim_{x \to a+}\frac{f(x)}{g(x)}
\]
в случае, когда
\[
\lim_{x \to a+}f(x) = \lim_{x \to a+}g(x) = 0
\]
Аналогично определяются неопределённости вида $\left[\frac{\infty}{\infty}\right]$ и в точке $b$.

Другие виды неопределённостей сводятся к этим двум. Вообще говоря, неопределённость типа $\left[\frac{\infty}{\infty}\right]$ может быть сведена к типу $\left[\frac{0}{0}\right]$. Действительно, пусть
$$\lim_{x \to a+}f(x) = \lim_{x \to a+}g(x) = \infty$$
тогда
\[
\frac{f(x)}{g(x)}=\frac{\frac{1}{g(x)}}{\frac{1}{f(x)}}
\]
Однако при раскрытии неопределённостей возникает необходимость расcматривать их отдельно.

Неопределённость-произведение сводится к неопределённостям-частным двумя способами:

$$
[0 \cdot \infty]=\lim_{x\to x_0}(f(x) \cdot g(x))=\lim_{x\to x_0}\frac{f(x)}{\frac{1}{g(x)}}=\left[\frac{0}{0}\right]
$$

$$
[0 \cdot \infty]=\lim_{x\to x_0}(f(x) \cdot g(x))=\lim_{x\to x_0}\frac{g(x)}{\frac{1}{f(x)}}=\left[\frac{\infty}{\infty}\right]
$$

Неопределённости-степени сводятся с неопределённостям-произведениям (а затем - к неопределённостям-частным) через равенство 
$$
f(x) ^ {g(x)}=e^{g(x) \cdot \ln f(x)}
$$

Заметим, что это равенство, как и сам предел, имеет смысл лишь при $f(x)>0$.
Покажем, как раскрываются неопределённости-степени:

$$
[\infty ^0]=\lim_{x\to x_0}(f(x) ^{g(x)})=\lim_{x\to x_0}e^{g(x) \cdot \ln f(x)}=e^{\lim_{x\to x_0}(g(x) \cdot \ln f(x))}=e^{[\infty \cdot 0]}
$$

$$
[0^0]=\lim_{x\to x_0}(f(x) ^{g(x)})=\lim_{x\to x_0}e^{g(x) \cdot \ln f(x)}=e^{\lim_{x\to x_0}(g(x) \cdot \ln f(x))}=e^{-[0 \cdot \infty]}
$$

$$
[1 ^\infty]=\lim_{x\to x_0}(f(x) ^{g(x)})=\lim_{x\to x_0}e^{g(x) \cdot \ln f(x)}=e^{\lim_{x\to x_0}(g(x) \cdot \ln f(x))}=e^{[0 \cdot \infty]}
$$

Наконец, рассмотри раскрытие неопределённости-разности:

$$
[\infty - \infty]=\lim_{x\to x_0}(f(x) - g(x))=\lim_{x\to x_0}\left(f(x) \cdot g(x)\left(\frac{1}{f(x)}-\frac{1}{g(x)}\right)\right)=[\infty \cdot 0]
$$

Таким образом, раскрытие неопределённостей сведено к раскрытию неопределённостей-частных.
