\section{Криволинейные интегралы первого рода}
\subsection{Задача о вычислении массы нити}
\subsection{Определение криволинейного интеграла первого рода}
\subsection{Сведение криволинейного интеграла первого рода к обыкновенному определённому интегралу}
...

\section{Криволинейные интегралы второго рода}
\subsection{Задача о вычислении работы}
\subsection{Определение криволинейного интеграла второго рода}
\subsection{Сведение криволинейного интеграла второго рода к обыкновенному определённому интегралу}
\subsection{Обобщение на $n$-мерный случай}
\subsection{Связь между криволинейными интегралами первого и второго рода}
\subsection{Ориентация кривой}
\subsection{Формула Грина. Нахождение площади плоской фигуры}
\subsection{Криволинейные интегралы второго рода, не зависящие от пути интегрирования}
...

\section{Элементы теории поверхностей}
\subsection{Понятие поверхности}
\subsection{Касательная плоскость и нормаль к поверхности}
\subsection{Ориентация поверхности}
...

\section{Площадь поверхности}
\subsection{Вантуз (сапог) Шварца}
\subsection{Определение площади поверхности}
\subsection{Преобразование элемента площади}
...

\section{Поверхностные интегралы первого рода}
\subsection{Определение поверхностного интеграла первого рода}
\subsection{Существование поверхностного интеграла первого рода и сведение его кобыкновенному двойному интегралу}
\subsection{Приложения поверхностных интегралов первого рода}
...

\section{Поверхностные интегралы второго рода}
\subsection{Определение и свойства}
\subsection{Сведение к обыкновенному двойному интегралу}
...


