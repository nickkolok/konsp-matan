\subsubsection{Теорема.}

\fXR.
Для того, чтобы функция $f$ была неубываюшей (невозрастающей) на $X$, необходимо и достаточно, чтобы
$\forall(x \in X) [f'(x) \geq 0] (f'(x) \leq 0])$.

\dokvo 

Докажем теорему для случая неубывающей функции. Доказательство для случая невозрастающей оставляем читателю ввиду его аналогичности.

\neobh

$f$ - неубывающая функция. Возьмём $x$ и $h \neq 0$ такие, что $x\in X, x+h \in X$.

Если $h>0$, то, так как $f$ - неубывающая, $f(x+h) \geq f(x)$.
Если $h<0$, то $f(x+h) \leq f(x)$.
Значит, 

$$
\frac{ f(x+h) - f(x) }{ h } \geq 0
$$

Переходя к пределу, имеем

\[
\lim_{h\to 0} { \frac{ f(x+h) - f(x) }{ h } } = f'(x) \geq 0
\]

\dost

$f'(x) \geq 0$. Пусть $\{x_1,x_2\} \subset X, x_1 < x_2$.
Тогда на отрезке $[x_1, x_2]$ функция $f$ дифференцируема. Применим теорему Лагранжа:

$$
\exists(c \in [ x_1, x_2 ]) [f(x_2) - f(x_1) = f'(c)(x_2 - x_1)]
$$

Но $f'(c) \geq 0$ и $x_2 - x_1 > 0$. Значит, и $f(x_2) - f(x_1) \geq 0$, т. е. функция $f$ - неубывающая.

\dokno


\subsubsection{Замечание}
\fXR, $\forall(x \in X) [f'(x) > 0] (f'(x) < 0])$.
Рассуждениями, аналогичными рассуждениями в части доказательства достаточности условия предыдущей теоремы, можно показать, что в таком случае функция $f$ -- возрастающая (убывающая).
Обратное, вообще говоря, неверно.
Например, возрастающая функция $f(x)=x^3$ имеет в точке $x=0$ нулевую производную:
$f'(x)=(x^3)'=3x^2$, $f'(0)=0$.

\subsubsection{Теорема}
\fXR, $f$ дифференцируема на $X$.
Для того, чтобы $f$ была возрастающей (убывающей), необходимо и достаточно, чтобы:

1) $\forall(x \in X) [f'(x)\geq 0]$

2) $\forall([a;b] \subset X)[f'(x)\not\equiv 0]$, т. е. чтобы ни на каком отрезке внутри $X$ $f'(x)$ не обращалась в тождественный нуль.

\dokvo 

Докажем теорему для случая возрастающей функции. Доказательство для случая убывающей оставляем читателю ввиду его аналогичности.

\neobh

$f(x)$ -- возрастающая. Тогда в силу предыдущей теоремы выполнено первое условие.
Установим, что второе условие также выполнено.
\pp, т. е. что $\exists([a;b] \subset X)\forall(x \in [a;b])[f'(x)=0]$.
Тогда $f(x)$ на $[a;b]$ постоянна, и $f(a)=f(b)$, следовательно, $f$ не является возрастающей. Получили противоречие.

\dost

Так как $f'(x) \geq 0$, то по предыдущей теореме $f$ -- неубывающая, т. е.
$\forall(x_1\in X, x_2 \in X : x_1<x_2)[f(x_2) \geq f(x_1)]$.

Докажем теперь, что $f(x_2) > f(x_1)$.
\pp, т. е. что $\exists(x_1\in X, x_2 \in X : x_1<x_2)[f(x_2) = f(x_1)]$.
Тогда $\forall(x\in [x_1; x_2])[f(x)=f(x_1)=f(x_2)]$, т. е. $\forall(x\in(x_1;x_2))[f'(x)=0]$, что противоречит второму условию теоремы.

\dokno

