\section{Предел отображения из \Rn в \Rm}
\subsection{Эквивалентность определений по Коши и по Гейне}
\subsection{Эквивалентость покоординатной сходимости}
\subsection{Предел линейной комбинации функций}
\subsection{Повторные пределы}
...

\section{Непрерывность отображения из \Rn в \Rm}
\subsection{Непрерывность координатных функций}
Определение непрерывности для случая векторной функции векторного аргумента мы не будем давать таким же образом, как давали определения для скалярных функций скалярного или векторного аргумента.
Вместо этого дадим сначала определение через непрерывность координатных функций, а затем докажем эквивалентность ему классических определений по Коши и по Гейне.

\begin{opr}
Если $f:A\subset\R^n\to\R^m$ и $\forall(x\in A)[f(x)=(f^1(x), ... , f^m(x))]$, то скалярные функции векторного аргумента $f^i:A\to\R$ называются координатными функциями исходной функции $f$.
\end{opr}

\begin{opr}\label{opred_predela_Rn_Rm_nepr}
Если $f:A\subset\R^n\to\R^m$ и $\forall(x\in A)[f(x)=(f^1(x), ... , f^m(x))]$, то функция $f$ называется непрерывной в точке $x_0\in A$, если в этой точке непрерывны все её координатные функции.
\end{opr}

\begin{opr}
Если функция $f$ непрерывна в каждой точке множества $A$, то она непрерывна на это множестве.
\end{opr}

Заметим, что из данного таким образом определения следует непрерывность функции в изолированной точке её области определения.


\subsection{Различные определения непрерывности}
Здесь и далее мы будем иногда без особого предупреждения использовать различные нормы в $\R^n$.
Внимательному читателю не составит труда вспомнить, что все они эквивалентны в силу конечномерности $\R^n$ (а невнимательному мы только что напомнили).

Дадим классическое определение непрерывности по Коши:
\begin{opr}
Пусть $f:A\subset\R^n\to\R^m$.
$f$ непрерывно в точке $x_0\in A$, если
$$
\forall(\varepsilon>0)\exists(\delta>0)\forall(x:0<|x-x_0|<\delta)[|f(x_0)-f(x)|<\varepsilon]
$$
\end{opr}

\begin{utverzhd}
Определение через непрерывность координатных функций (определение \ref{opred_predela_Rn_Rm_nepr}) и приведённое выше определения эквивалентны.
\end{utverzhd}

\dokvo
Положим для $y=(y^1, ... , y^m) |y|=\max\limits_{i}|y^i|$. 
Тогда
\begin{multline*}
\forall(\varepsilon>0)\exists(\delta>0)\forall(x:0<|x-x_0|<\delta)[|f(x_0)-f(x)|<\varepsilon]
\\\Leftrightarrow
\forall(\varepsilon>0)\exists(\delta>0)\forall(x:0<|x-x_0|<\delta)[\max\limits_{i}|f^i(x_0)-f^i(x)|<\varepsilon]
\Leftrightarrow\\
\forall(i\in\{1,...,m\})\forall(\varepsilon>0)\exists(\delta>0)\forall(x:0<|x-x_0|<\delta)[|f^i(x_0)-f^i(x)|<\varepsilon]
\end{multline*}
Предыдущая строка и означает непрерывность координатных функций.
\dokno

Теперь, следуя порядку изложения предыдущих разделов, дадим определения через окрестности:

\begin{opr}
Пусть $f:A\subset\R^n\to\R^m$.
$f$ непрерывно в точке $x_0\in A$, если
$$
\forall(\varepsilon>0)\exists(\delta>0)[x\in\mathring{U}_\delta(x_0)\cap A \Rightarrow f(x)\in U_\varepsilon(f(x_0))]
$$
\end{opr}

\begin{opr}
Пусть $f:A\subset\R^n\to\R^m$.
$f$ непрерывно в точке $x_0\in A$, если
$$
\forall(V(f(x_0))\exists(\mathring{U}(x_0))[f(\mathring{U}(x_0)\cap A)\in V(f(x_0))]
$$
\end{opr}

Наконец, перепишем то же самое в предельной форме:

\begin{opr}
Пусть $f:A\subset\R^n\to\R^m$.
$f$ непрерывно в точке $x_0\in A$, если
$$
\lim\limits_{x\to x_0, x\in A} f(x) = f(x_0)
$$
\end{opr}

Теперь сформулируем определение непрерывности по Гейне:

\begin{opr}
Пусть $f:A\subset\R^n\to\R^m$.
$f$ непрерывно в точке $x_0\in A$, если
$$
\forall(\{x_k\}\subset A\\\{x_0\}:\{x_k\}\to x_0)[\{f(x_k)\}\to f(x_0)]
$$
\end{opr}

Его эквивалентность определению \ref{opred_predela_Rn_Rm_nepr} доказывается аналогично --- через переход к координатным функциям.


%\subsection{Ограниченность образа окрестности}
\subsection{Непрерывность линейной комбинации функций}
\subsection{Непрерывность сложного отображения}
\subsection{Теорема Вейерштрасса}
\subsection{Линейная связность образа}
\subsection{Теорема Кантора}
\subsection{Открытость прообраза}
...

\section{Линейные отображения из \Rn в \Rm}
\subsection{Определение линейного отображения}
Перед тем, как познакомить читателя с понятием производной векторной функции векторного аргумента, не лишним будет напомнить ему некоторые базовые сведения о линейных отображениях, изученные ранее в курсе алгебры.

\begin{opr}
	Отображение $L:\R^n\to\R^m$ назывется линейным, если:
	\begin{equation}\label{additivnost_lin_otobr}
		\forall(x,y\in\R^n)[L(x+y)=L(x)+L(y)]
	\end{equation}
	\begin{equation}\label{odnorodnost_lin_otobr}
		\forall(x\in\R^n)\forall(\lambda\in\R)[L(\lambda x)=\lambda L(x)]
	\end{equation}
\end{opr}

Свойства \ref{additivnost_lin_otobr}, называемое аддитивностью, и \ref{odnorodnost_lin_otobr}, называемое однородностью, иногда объединяют в свойство, называемое линейностью:

\begin{equation}\label{lineynost_otobr}
	\forall(x,y\in\R^n)\forall(\lambda,\mu\in\R)[L(\lambda x + \mu y)=\lambda L(x)+\mu L(y)]
\end{equation}

По индукции легко установить, что для линейного отображения $L$ верно следующее:

\begin{equation}\label{lineynost_otobr}
	\forall(x_1, ... , x_n \in\R^n)\forall(\lambda_1, ... , \lambda_n \in\R)\left[L\left( \sum_{i=1}^n \lambda_i x_i\right)=\sum_{i=1}^n L(\lambda_i x_i)\right]
\end{equation}

Отдельно обратим внимание читателя на следующие утверждения:
\begin{utverzhd}
	Линейное отображение является непрерывным.
\end{utverzhd}
\begin{utverzhd}
	Если $L_1$ и $L_2$ --- линейные отображения из $\R^n$ в $\R^m$, то отображение $L:\R^n\to\R^m$,
	задаваемое формулой $L(x)=\lambda_1 L_1 (x)+\lambda_2 L_2 (x)$, где $\{\lambda_1, \lambda_2\}\subset\R$, также является линейным. 
\end{utverzhd}
\begin{utverzhd}
	Суперпозиция линейных отображений есть линейное отображение.
\end{utverzhd}

Напомним, что в курсе алгебры вводилось понятие матрицы линейного отображения.
Излагая дальнейший материал, будем считать, что в пространстве \Rn задан стандартный базис.
Учитывая это, условимся сокращать для линейного отображения запись $L(x)$ до $Lx$.

\begin{utverzhd}
	Матрица суперпозиции линейных отображений есть произведение матриц соответствующих линейных отображений, притом матрица внешнего отображения ставится слева (напомним, что произведение матриц некоммутативно).
\end{utverzhd}



\subsection{Норма линейного отображения}
Из курса алгебры читателю известно, что линейные отображения из \Rn в $\R^m$ образуют линейное пространство $L(\R^n,\R^m)$ размерности $nm$.
Ввести норму на этом пространстве можно различными способами --- например, как сумму элементов матрицы в некотором фиксированном базисе или как максимальный элемент такой матрицы.
Подобное разнообразие широко используется, например, в курсе дифференциальных уравнений;
напомним, что в конечномерном пространстве все нормы эквивалентны.
Нам же было бы удобно условиться называть нормой некоторый фиксированный функционал, действующий из пространства линейных отображений в $\R$.

Рассмотрим функцию $n(L)=\sup\limits_{|x|\leq 1}|L(x)|$, где $x\in\R^n$, $L:\R^n\to\R^m$. 
Заметим, что $\phi(x)=|L(x)|$ --- сккалярная функция векторного аргумента.
Значит, на компакте, задаваемом неравенством $|x|\leq 1$, т. е. на замкнутом единичном шаре в $\R^n$, она ограничена и достигает своего супремума.
Таким образом, мы можем назвать функцию $n$ нормой.
Итак, в дальнейшем будем считать, что
$$
\|L\|=\sup\limits_{|x|\leq 1}|L(x)|
$$

\begin{utverzhd}
$\forall(x: |x| \leq 1)[|Lx|\leq\|L\|]$
\end{utverzhd}

\begin{sledstvie}
$\forall(x\in\R^n)[|Lx|\leq\|L\|\cdot|x|]$
\end{sledstvie}

\dokvo
Для $x=0$ утверждение очевидно.

Для $x \neq 0$ имеем
$$
|Lx|=\left|L\left(\frac{x}{|x|}\right)\cdot|x|\right|=|x|\cdot\left|L\left(\frac{x}{|x|}\right)\right|\leq|x|\cdot\|L\|
$$
\dokno



...

\section{Дифференцируемость отображения из \Rn в \Rm}
\subsection{Определение производной Фреше}
\begin{opr}\label{oped_proizv_Freshe_Rn_Rm}
Пусть $A\subset\R^n$, $x_0$ --- внутренняя точка $A$, $f:A\to\R^m$.
Отображение $f$ называется дифференцируемым по Фреше в точке $A$, если
\begin{multline}
\exists(L_{x_0}\in L(\R^n,\R^m))\forall(h\in \R^n: x_0+h\in A)
\\
[f(x_0+h)-f(x_0)=L_{x_0}h+\omega(x_0,h), \omega(x_0,h)=o(\|h\|)]
\end{multline}
\end{opr}

\begin{opr}
Введённое выше линейное отображение $L_{x_0}$ называется производной по Фреше функции $f$ в точке $x_0$ и обозначается $f'(x_0)$ или $Df(x_0)$. Иногда его также называют производным отображением или касательным отображением.
Значение $f'(x_0)$ на элементе $h$ называется дифференциалом функции $f$, соответствующим приращению $h$: $df(x_0,h)=f'(x_0)h$.
\end{opr}

\begin{opr}
Функцию, дифференцируемую в каждой точке некоторого множества, называют дифференцируемой на этом множестве.
\end{opr}

\begin{zamech}
В отличие от случая скалярной функции скалярного аргумента, функция и производная представляют собой объекты разной природы:
$$
f(x)\in\R^m
$$
$$
f'(x)\in L(\R^n,\R^m)
$$
Иногда понятия дифференциала и производной не различают.
\end{zamech}

\begin{opr}
Введённая в определении \ref{oped_proizv_Freshe_Rn_Rm} функцию $\omega(x_0,h)$ называется остатком приращения.
\end{opr}

\begin{primer}
Если $f:A\to\R^m$, $f=const$, т. е. постоянно, то $\forall(x\in A)[f'(x)=0]$.
\end{primer}

\begin{primer}
Если $f:A\to\R^m$, $f\in L(\R^n,\R^m)$, т. е. линейно, то $\forall(x\in A)[f'(x)=f]$.
\end{primer}

Покажем теперь, как вычислять производную Фреше по определению, заодно убедив читателя, что это не вполне удобно.
\begin{primer}
$f:\R^2 \to \R^2$,ж
$$f(x^1,x^2)=\left(\frac{1}{2}\left((x^1)^2-(x^2)^2\right); x^1 x^2\right)$$

Зафиксируем $h=(h^1, h^2)$.
Выписываем приращение функции:
\begin{multline}
f(x+h)-f(x)
=\\=
\left(\frac{1}{2}\left((x^1+h^1)^2-(x^2+h^2)^2\right); (x^1+h^1)( x^2+h^2) \right)-\left(\frac{1}{2}\left((x^1)^2-(x^2)^2\right); x^1 x^2\right)
=\\=
(x^1h^1-x^2h^2; x^1h^2-x^2h^1)+\left( \frac{1}{2}\left( (h^1)^2+(h^2)^2 \right); h^1 h^2 \right)
\end{multline}
Видно, что первое слагаемое линейно по $h$, второе --- нет.
Значит,
$$
f'(x)(h^1,h^2)=(x^1h^1-x^2h^2; x^1h^2-x^2h^1)=
\begin{pmatrix}
x^1 & -x^2\\
x^2 & x^1
\end{pmatrix}
\begin{pmatrix}
h^1\\
h^2
\end{pmatrix}
$$
Таким образом, производная имеет вид
$$
f'(x)=
\begin{pmatrix}
x^1 & -x^2\\
x^2 & x^1
\end{pmatrix}
$$
\end{primer}


\subsection{Свойства производной}
\subsection{Теорема о дифференцируемости сложного отображения}
\subsection{Три следствия из теоремы о дифференцируемости сложного отображения}
\subsection{Матрица Якоби. Якобиан}
...


\section{Принцип сжимающих отображений}
\subsection{Необходимые определения}
\subsection{Принцип сжимающих отображений}
\subsection{Оценка погрешности при использовании метода последовательных приближений}
...

\section{Теорема о конечных приращениях}
\subsection{Пример невозможности дословного переноса теорема Лагранжа со случая скалярной функции векторного аргумента}
Продолжая перенос результатов, полученных при изучении векторной функции скалярного аргумента, на случай векторной функции векторного аргумента, рассмотрим теорему Лагранжа.

\begin{teorema}
Пусть $G\subset\R^n$, $G$ открыто, $f:G\to\R$, $f$ - дифференцируема на $G$.
Тогда
\begin{equation}
\forall([x;;y]\subset G)\exists(\xi\in(x;;y))[f(x)-f(y)=f'(\xi)(x-y)]
\end{equation}
\end{teorema}

Покажем, что в форме равенства теорему Лагранжа перенести на случай $f:\R^n\to\R^m$ нельзя.
Рассмотрим $f(x)=(\sin x; \cos x)$. 
Тогда $f'(x)=(\cos x; -\sin x)$.
Выпишем опровергаемое утверждение для отрезка $[0;\frac{\pi}{2}]$:
\begin{equation}\label{teorema_Lagranzha_vect_f_vect_arg_kontr}
(\sin 0; \cos 0)-(\sin \frac{\pi}{2}; \cos \frac{\pi}{2}) = (\cos \xi; -\sin \xi)\cdot\frac{\pi}{2}
\end{equation}
Распишем покоординатно:
$$
\sin 0-\sin \frac{\pi}{2} = \cos \xi \cdot\frac{\pi}{2}
$$
$$
\cos 0-\cos \frac{\pi}{2} = -\sin \xi\cdot\frac{\pi}{2}
$$
То есть:
$$
0-1 = \cos \xi \cdot\frac{\pi}{2}
$$
$$
1-0 = -\sin \xi\cdot\frac{\pi}{2}
$$
Отсюда имеем $\cos \xi = \sin \xi = - \frac{2}{\pi}$, то есть $\cos^2 \xi + \sin^2 \xi \neq 1$, что невозможно.
Следовательно, ни при каком $\xi$ равенство \ref{teorema_Lagranzha_vect_f_vect_arg_kontr} выполнено быть не может.


\subsection{Лемма о системе стягивающихся отрезков}
\subsection{Теорема о конечных приращениях}
...

\section{Другие теоремы}
\subsection{Теорема об обратном отображении}
\subsection{Теорема о неявном отображении}
\subsection{Условный экстремум}
...
