\section{Предел отображения из \Rn в \Rm}
\subsection{Эквивалентность определений по Коши и по Гейне}
\subsection{Эквивалентость покоординатной сходимости}
\subsection{Предел линейной комбинации функций}
\subsection{Повторные пределы}
...

\section{Непрерывность отображения из \Rn в \Rm}
\subsection{Различные определения непрерывности}
\subsection{Непрерывность координатных функций}
\subsection{Ограниченность образа окрестности}
\subsection{Непрерывность линейной комбинации функций}
\subsection{Непрерывность сложного отображения}
\subsection{Теорема Вейерштрасса}
\subsection{Линейная связность образа}
\subsection{Теорема Кантора}
\subsection{Открытость прообраза}
...

\section{Линейные отображения из \Rn в \Rm}
\subsection{Определение линейного отображения}
Перед тем, как познакомить читателя с понятием производной векторной функции векторного аргумента, не лишним будет напомнить ему некоторые базовые сведения о линейных отображениях, изученные ранее в курсе алгебры.

\begin{opr}
	Отображение $L:\R^n\to\R^m$ назывется линейным, если:
	\begin{equation}\label{additivnost_lin_otobr}
		\forall(x,y\in\R^n)[L(x+y)=L(x)+L(y)]
	\end{equation}
	\begin{equation}\label{odnorodnost_lin_otobr}
		\forall(x\in\R^n)\forall(\lambda\in\R)[L(\lambda x)=\lambda L(x)]
	\end{equation}
\end{opr}

Свойства \ref{additivnost_lin_otobr}, называемое аддитивностью, и \ref{odnorodnost_lin_otobr}, называемое однородностью, иногда объединяют в свойство, называемое линейностью:

\begin{equation}\label{lineynost_otobr}
	\forall(x,y\in\R^n)\forall(\lambda,\mu\in\R)[L(\lambda x + \mu y)=\lambda L(x)+\mu L(y)]
\end{equation}

По индукции легко установить, что для линейного отображения $L$ верно следующее:

\begin{equation}\label{lineynost_otobr}
	\forall(x_1, ... , x_n \in\R^n)\forall(\lambda_1, ... , \lambda_n \in\R)\left[L\left( \sum_{i=1}^n \lambda_i x_i\right)=\sum_{i=1}^n L(\lambda_i x_i)\right]
\end{equation}

Отдельно обратим внимание читателя на следующие утверждения:
\begin{utverzhd}
	Линейное отображение является непрерывным.
\end{utverzhd}
\begin{utverzhd}
	Если $L_1$ и $L_2$ --- линейные отображения из $\R^n$ в $\R^m$, то отображение $L:\R^n\to\R^m$,
	задаваемое формулой $L(x)=\lambda_1 L_1 (x)+\lambda_2 L_2 (x)$, где $\{\lambda_1, \lambda_2\}\subset\R$, также является линейным. 
\end{utverzhd}
\begin{utverzhd}
	Суперпозиция линейных отображений есть линейное отображение.
\end{utverzhd}

Напомним, что в курсе алгебры вводилось понятие матрицы линейного отображения.
Излагая дальнейший материал, будем считать, что в пространстве \Rn задан стандартный базис.
Учитывая это, условимся сокращать для линейного отображения запись $L(x)$ до $Lx$.

\begin{utverzhd}
	Матрица суперпозиции линейных отображений есть произведение матриц соответствующих линейных отображений, притом матрица внешнего отображения ставится слева (напомним, что произведение матриц некоммутативно).
\end{utverzhd}



\subsection{Норма линейного отображения}
...

\section{Дифференцируемость отображения из \Rn в \Rm}
\subsection{Определение производной Фреше}
\subsection{Свойства производной}
\subsection{Теорема о дифференцируемости сложного отображения}
\subsection{Три следствия из теоремы о дифференцируемости сложного отображения}
\subsection{Матрица Якоби. Якобиан}
...


\section{Принцип сжимающих отображений}
\subsection{Необходимые определения}
\subsection{Принцип сжимающих отображений}
\subsection{Оценка погрешности при использовании метода последовательных приближений}
...

\section{Теорема о конечных приращениях}
\subsection{Пример невозможности дословного переноса теорема Лагранжа со случая скалярной функции векторного аргумента}
Продолжая перенос результатов, полученных при изучении векторной функции скалярного аргумента, на случай векторной функции векторного аргумента, рассмотрим теорему Лагранжа.

\begin{teorema}
Пусть $G\subset\R^n$, $G$ открыто, $f:G\to\R$, $f$ - дифференцируема на $G$.
Тогда
\begin{equation}
\forall([x;;y]\subset G)\exists(\xi\in(x;;y))[f(x)-f(y)=f'(\xi)(x-y)]
\end{equation}
\end{teorema}

Покажем, что в форме равенства теорему Лагранжа перенести на случай $f:\R^n\to\R^m$ нельзя.
Рассмотрим $f(x)=(\sin x; \cos x)$. 
Тогда $f'(x)=(\cos x; -\sin x)$.
Выпишем опровергаемое утверждение для отрезка $[0;\frac{\pi}{2}]$:
\begin{equation}\label{teorema_Lagranzha_vect_f_vect_arg_kontr}
(\sin 0; \cos 0)-(\sin \frac{\pi}{2}; \cos \frac{\pi}{2}) = (\cos \xi; -\sin \xi)\cdot\frac{\pi}{2}
\end{equation}
Распишем покоординатно:
$$
\sin 0-\sin \frac{\pi}{2} = \cos \xi \cdot\frac{\pi}{2}
$$
$$
\cos 0-\cos \frac{\pi}{2} = -\sin \xi\cdot\frac{\pi}{2}
$$
То есть:
$$
0-1 = \cos \xi \cdot\frac{\pi}{2}
$$
$$
1-0 = -\sin \xi\cdot\frac{\pi}{2}
$$
Отсюда имеем $\cos \xi = \sin \xi = - \frac{2}{\pi}$, то есть $\cos^2 \xi + \sin^2 \xi \neq 1$, что невозможно.
Следовательно, ни при каком $\xi$ равенство \ref{teorema_Lagranzha_vect_f_vect_arg_kontr} выполнено быть не может.


\subsection{Лемма о системе стягивающихся отрезков}
\subsection{Теорема о конечных приращениях}
...

\section{Другие теоремы}
\subsection{Теорема об обратном отображении}
\subsection{Теорема о неявном отображении}
\subsection{Условный экстремум}
...
