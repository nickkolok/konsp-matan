\section{Предел отображения из \Rn в \Rm}
\subsection{Эквивалентность определений по Коши и по Гейне}
\subsection{Эквивалентость покоординатной сходимости}
\subsection{Предел линейной комбинации функций}
\subsection{Повторные пределы}
...

\section{Непрерывность отображения из \Rn в \Rm}
\subsection{Различные определения непрерывности}
\subsection{Непрерывность координатных функций}
\subsection{Ограниченность образа окрестности}
\subsection{Непрерывность линейной комбинации функций}
\subsection{Непрерывность сложного отображения}
\subsection{Теорема Вейерштрасса}
\subsection{Линейная связность образа}
\subsection{Теорема Кантора}
\subsection{Открытость прообраза}
...

\section{Линейные отображения из \Rn в \Rm}
\subsection{Определение линейного отображения}
\subsection{Норма линейного отображения}
...

\section{Дифференцируемость отображения из \Rn в \Rm}
\subsection{Определение производной Фреше}
\subsection{Свойства производной}
\subsection{Теорема о дифференцируемости сложного отображения}
\subsection{Три следствия из теоремы о дифференцируемости сложного отображения}
\subsection{Матрица Якоби. Якобиан}
...


\section{Принцип сжимающих отображений}
\subsection{Необходимые определения}
\subsection{Принцип сжимающих отображений}
\subsection{Оценка погрешности при использовании метода последовательных приближений}
...

\section{Теорема о конечных приращениях}
\subsection{Пример невозможности дословного переноса теорема Лагранжа со случая скалярной функции векторного аргумента}
\subsection{Лемма о системе стягивающихся отрезков}
\subsection{Теорема о конечных приращениях}
...

