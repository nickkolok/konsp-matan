\subsubsection{Теорема.}

Для того, чтобы число $a \in \R$ было верхним пределом последовательности $\{x_n\}$, необходимо и достаточно выполнения следующих двух условий:

1)$\forall(\epsilon >0 )\exists(n_0 \in \N)\forall(n \geq n_0)[x_n < a + \epsilon]$

2)$\forall(\epsilon > 0)\forall(m   \in \N)\exists(n \geq m  )[x_n > a - \epsilon]$

\subsubsection{Замечание.}
Условие (1) означает, что количество членов последовательности, б\`{о}льших $a+\epsilon$, конечно.

Условие (2) означает, что количество членов подпоследовательности, б\`{о}льших $a-\epsilon$, бесконечно.

\par

Аналогично формулируется характеристическое свойство нижнего предела:

\subsubsection{Теорема.}

Для того, чтобы число $a \in \R$ было нижним пределом последовательности $\{x_n\}$, необходимо и достаточно выполнения следующих двух условий:

1)$\forall(\epsilon >0 )\exists(n_0 \in \N)\forall(n \geq n_0)[x_n > a - \epsilon]$

2)$\forall(\epsilon > 0)\forall(m   \in \N)\exists(n \geq m  )[x_n < a + \epsilon]$

\subsubsection{Замечание.}
Условие (1) означает, что количество членов последовательности, меньших $a-\epsilon$, конечно.

Условие (2) означает, что количество членов подпоследовательности, меньших $a+\epsilon$, бесконечно.


