Из курса алгебры читателю известно, что линейные отображения из \Rn в $\R^m$ образуют линейное пространство размерности $nm$.
Ввести норму на этом пространстве можно различными способами --- например, как сумму элементов матрицы в некотором фиксированном базисе или как максимальный элемент такой матрицы.
Подобное разнообразие широко используется, например, в курсе дифференциальных уравнений;
напомним, что в конечномерном пространстве все нормы эквивалентны.
Нам же было бы удобно условиться называть нормой некоторый фиксированный функционал, действующий из пространства линейных отображений в $\R$.

Рассмотрим функцию $n(L)=\sup\limits_{|x|\leq 1}|L(x)|$, где $x\in\R^n$, $L:\R^n\to\R^m$. 
Заметим, что $\phi(x)=|L(x)|$ --- сккалярная функция векторного аргумента.
Значит, на компакте, задаваемом неравенством $|x|\leq 1$, т. е. на замкнутом единичном шаре в $\R^n$, она ограничена и достигает своего супремума.
Таким образом, мы можем назвать функцию $n$ нормой.
Итак, в дальнейшем будем считать, что
$$
\|L\|=\sup\limits_{|x|\leq 1}|L(x)|
$$

\begin{utverzhd}
$\forall(x: |x| \leq 1)[|Lx|\leq\|L\|]$
\end{utverzhd}

\begin{sledstvie}
$\forall(x\in\R^n)[|Lx|\leq\|L\|\cdot|x|]$
\end{sledstvie}

\dokvo
Для $x=0$ утверждение очевидно.

Для $x \neq 0$ имеем
$$
|Lx|=\left|L\left(\frac{x}{|x|}\right)\cdot|x|\right|=|x|\cdot\left|L\left(\frac{x}{|x|}\right)\right|\leq|x|\cdot\|L\|
$$
\dokno


